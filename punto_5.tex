\documentclass[11pt]{article}

    \usepackage[breakable]{tcolorbox}
    \usepackage{parskip} % Stop auto-indenting (to mimic markdown behaviour)
    

    % Basic figure setup, for now with no caption control since it's done
    % automatically by Pandoc (which extracts ![](path) syntax from Markdown).
    \usepackage{graphicx}
    % Maintain compatibility with old templates. Remove in nbconvert 6.0
    \let\Oldincludegraphics\includegraphics
    % Ensure that by default, figures have no caption (until we provide a
    % proper Figure object with a Caption API and a way to capture that
    % in the conversion process - todo).
    \usepackage{caption}
    \DeclareCaptionFormat{nocaption}{}
    \captionsetup{format=nocaption,aboveskip=0pt,belowskip=0pt}

    \usepackage{float}
    \floatplacement{figure}{H} % forces figures to be placed at the correct location
    \usepackage{xcolor} % Allow colors to be defined
    \usepackage{enumerate} % Needed for markdown enumerations to work
    \usepackage{geometry} % Used to adjust the document margins
    \usepackage{amsmath} % Equations
    \usepackage{amssymb} % Equations
    \usepackage{textcomp} % defines textquotesingle
    % Hack from http://tex.stackexchange.com/a/47451/13684:
    \AtBeginDocument{%
        \def\PYZsq{\textquotesingle}% Upright quotes in Pygmentized code
    }
    \usepackage{upquote} % Upright quotes for verbatim code
    \usepackage{eurosym} % defines \euro

    \usepackage{iftex}
    \ifPDFTeX
        \usepackage[T1]{fontenc}
        \IfFileExists{alphabeta.sty}{
              \usepackage{alphabeta}
          }{
              \usepackage[mathletters]{ucs}
              \usepackage[utf8x]{inputenc}
          }
    \else
        \usepackage{fontspec}
        \usepackage{unicode-math}
    \fi

    \usepackage{fancyvrb} % verbatim replacement that allows latex
    \usepackage{grffile} % extends the file name processing of package graphics
                         % to support a larger range
    \makeatletter % fix for old versions of grffile with XeLaTeX
    \@ifpackagelater{grffile}{2019/11/01}
    {
      % Do nothing on new versions
    }
    {
      \def\Gread@@xetex#1{%
        \IfFileExists{"\Gin@base".bb}%
        {\Gread@eps{\Gin@base.bb}}%
        {\Gread@@xetex@aux#1}%
      }
    }
    \makeatother
    \usepackage[Export]{adjustbox} % Used to constrain images to a maximum size
    \adjustboxset{max size={0.9\linewidth}{0.9\paperheight}}

    % The hyperref package gives us a pdf with properly built
    % internal navigation ('pdf bookmarks' for the table of contents,
    % internal cross-reference links, web links for URLs, etc.)
    \usepackage{hyperref}
    % The default LaTeX title has an obnoxious amount of whitespace. By default,
    % titling removes some of it. It also provides customization options.
    \usepackage{titling}
    \usepackage{longtable} % longtable support required by pandoc >1.10
    \usepackage{booktabs}  % table support for pandoc > 1.12.2
    \usepackage{array}     % table support for pandoc >= 2.11.3
    \usepackage{calc}      % table minipage width calculation for pandoc >= 2.11.1
    \usepackage[inline]{enumitem} % IRkernel/repr support (it uses the enumerate* environment)
    \usepackage[normalem]{ulem} % ulem is needed to support strikethroughs (\sout)
                                % normalem makes italics be italics, not underlines
    \usepackage{soul}      % strikethrough (\st) support for pandoc >= 3.0.0
    \usepackage{mathrsfs}
    

    
    % Colors for the hyperref package
    \definecolor{urlcolor}{rgb}{0,.145,.698}
    \definecolor{linkcolor}{rgb}{.71,0.21,0.01}
    \definecolor{citecolor}{rgb}{.12,.54,.11}

    % ANSI colors
    \definecolor{ansi-black}{HTML}{3E424D}
    \definecolor{ansi-black-intense}{HTML}{282C36}
    \definecolor{ansi-red}{HTML}{E75C58}
    \definecolor{ansi-red-intense}{HTML}{B22B31}
    \definecolor{ansi-green}{HTML}{00A250}
    \definecolor{ansi-green-intense}{HTML}{007427}
    \definecolor{ansi-yellow}{HTML}{DDB62B}
    \definecolor{ansi-yellow-intense}{HTML}{B27D12}
    \definecolor{ansi-blue}{HTML}{208FFB}
    \definecolor{ansi-blue-intense}{HTML}{0065CA}
    \definecolor{ansi-magenta}{HTML}{D160C4}
    \definecolor{ansi-magenta-intense}{HTML}{A03196}
    \definecolor{ansi-cyan}{HTML}{60C6C8}
    \definecolor{ansi-cyan-intense}{HTML}{258F8F}
    \definecolor{ansi-white}{HTML}{C5C1B4}
    \definecolor{ansi-white-intense}{HTML}{A1A6B2}
    \definecolor{ansi-default-inverse-fg}{HTML}{FFFFFF}
    \definecolor{ansi-default-inverse-bg}{HTML}{000000}

    % common color for the border for error outputs.
    \definecolor{outerrorbackground}{HTML}{FFDFDF}

    % commands and environments needed by pandoc snippets
    % extracted from the output of `pandoc -s`
    \providecommand{\tightlist}{%
      \setlength{\itemsep}{0pt}\setlength{\parskip}{0pt}}
    \DefineVerbatimEnvironment{Highlighting}{Verbatim}{commandchars=\\\{\}}
    % Add ',fontsize=\small' for more characters per line
    \newenvironment{Shaded}{}{}
    \newcommand{\KeywordTok}[1]{\textcolor[rgb]{0.00,0.44,0.13}{\textbf{{#1}}}}
    \newcommand{\DataTypeTok}[1]{\textcolor[rgb]{0.56,0.13,0.00}{{#1}}}
    \newcommand{\DecValTok}[1]{\textcolor[rgb]{0.25,0.63,0.44}{{#1}}}
    \newcommand{\BaseNTok}[1]{\textcolor[rgb]{0.25,0.63,0.44}{{#1}}}
    \newcommand{\FloatTok}[1]{\textcolor[rgb]{0.25,0.63,0.44}{{#1}}}
    \newcommand{\CharTok}[1]{\textcolor[rgb]{0.25,0.44,0.63}{{#1}}}
    \newcommand{\StringTok}[1]{\textcolor[rgb]{0.25,0.44,0.63}{{#1}}}
    \newcommand{\CommentTok}[1]{\textcolor[rgb]{0.38,0.63,0.69}{\textit{{#1}}}}
    \newcommand{\OtherTok}[1]{\textcolor[rgb]{0.00,0.44,0.13}{{#1}}}
    \newcommand{\AlertTok}[1]{\textcolor[rgb]{1.00,0.00,0.00}{\textbf{{#1}}}}
    \newcommand{\FunctionTok}[1]{\textcolor[rgb]{0.02,0.16,0.49}{{#1}}}
    \newcommand{\RegionMarkerTok}[1]{{#1}}
    \newcommand{\ErrorTok}[1]{\textcolor[rgb]{1.00,0.00,0.00}{\textbf{{#1}}}}
    \newcommand{\NormalTok}[1]{{#1}}

    % Additional commands for more recent versions of Pandoc
    \newcommand{\ConstantTok}[1]{\textcolor[rgb]{0.53,0.00,0.00}{{#1}}}
    \newcommand{\SpecialCharTok}[1]{\textcolor[rgb]{0.25,0.44,0.63}{{#1}}}
    \newcommand{\VerbatimStringTok}[1]{\textcolor[rgb]{0.25,0.44,0.63}{{#1}}}
    \newcommand{\SpecialStringTok}[1]{\textcolor[rgb]{0.73,0.40,0.53}{{#1}}}
    \newcommand{\ImportTok}[1]{{#1}}
    \newcommand{\DocumentationTok}[1]{\textcolor[rgb]{0.73,0.13,0.13}{\textit{{#1}}}}
    \newcommand{\AnnotationTok}[1]{\textcolor[rgb]{0.38,0.63,0.69}{\textbf{\textit{{#1}}}}}
    \newcommand{\CommentVarTok}[1]{\textcolor[rgb]{0.38,0.63,0.69}{\textbf{\textit{{#1}}}}}
    \newcommand{\VariableTok}[1]{\textcolor[rgb]{0.10,0.09,0.49}{{#1}}}
    \newcommand{\ControlFlowTok}[1]{\textcolor[rgb]{0.00,0.44,0.13}{\textbf{{#1}}}}
    \newcommand{\OperatorTok}[1]{\textcolor[rgb]{0.40,0.40,0.40}{{#1}}}
    \newcommand{\BuiltInTok}[1]{{#1}}
    \newcommand{\ExtensionTok}[1]{{#1}}
    \newcommand{\PreprocessorTok}[1]{\textcolor[rgb]{0.74,0.48,0.00}{{#1}}}
    \newcommand{\AttributeTok}[1]{\textcolor[rgb]{0.49,0.56,0.16}{{#1}}}
    \newcommand{\InformationTok}[1]{\textcolor[rgb]{0.38,0.63,0.69}{\textbf{\textit{{#1}}}}}
    \newcommand{\WarningTok}[1]{\textcolor[rgb]{0.38,0.63,0.69}{\textbf{\textit{{#1}}}}}


    % Define a nice break command that doesn't care if a line doesn't already
    % exist.
    \def\br{\hspace*{\fill} \\* }
    % Math Jax compatibility definitions
    \def\gt{>}
    \def\lt{<}
    \let\Oldtex\TeX
    \let\Oldlatex\LaTeX
    \renewcommand{\TeX}{\textrm{\Oldtex}}
    \renewcommand{\LaTeX}{\textrm{\Oldlatex}}
    % Document parameters
    % Document title
    \title{}
    \author{}
    \date{}
    
    
    
    
    
    
    
% Pygments definitions
\makeatletter
\def\PY@reset{\let\PY@it=\relax \let\PY@bf=\relax%
    \let\PY@ul=\relax \let\PY@tc=\relax%
    \let\PY@bc=\relax \let\PY@ff=\relax}
\def\PY@tok#1{\csname PY@tok@#1\endcsname}
\def\PY@toks#1+{\ifx\relax#1\empty\else%
    \PY@tok{#1}\expandafter\PY@toks\fi}
\def\PY@do#1{\PY@bc{\PY@tc{\PY@ul{%
    \PY@it{\PY@bf{\PY@ff{#1}}}}}}}
\def\PY#1#2{\PY@reset\PY@toks#1+\relax+\PY@do{#2}}

\@namedef{PY@tok@w}{\def\PY@tc##1{\textcolor[rgb]{0.73,0.73,0.73}{##1}}}
\@namedef{PY@tok@c}{\let\PY@it=\textit\def\PY@tc##1{\textcolor[rgb]{0.24,0.48,0.48}{##1}}}
\@namedef{PY@tok@cp}{\def\PY@tc##1{\textcolor[rgb]{0.61,0.40,0.00}{##1}}}
\@namedef{PY@tok@k}{\let\PY@bf=\textbf\def\PY@tc##1{\textcolor[rgb]{0.00,0.50,0.00}{##1}}}
\@namedef{PY@tok@kp}{\def\PY@tc##1{\textcolor[rgb]{0.00,0.50,0.00}{##1}}}
\@namedef{PY@tok@kt}{\def\PY@tc##1{\textcolor[rgb]{0.69,0.00,0.25}{##1}}}
\@namedef{PY@tok@o}{\def\PY@tc##1{\textcolor[rgb]{0.40,0.40,0.40}{##1}}}
\@namedef{PY@tok@ow}{\let\PY@bf=\textbf\def\PY@tc##1{\textcolor[rgb]{0.67,0.13,1.00}{##1}}}
\@namedef{PY@tok@nb}{\def\PY@tc##1{\textcolor[rgb]{0.00,0.50,0.00}{##1}}}
\@namedef{PY@tok@nf}{\def\PY@tc##1{\textcolor[rgb]{0.00,0.00,1.00}{##1}}}
\@namedef{PY@tok@nc}{\let\PY@bf=\textbf\def\PY@tc##1{\textcolor[rgb]{0.00,0.00,1.00}{##1}}}
\@namedef{PY@tok@nn}{\let\PY@bf=\textbf\def\PY@tc##1{\textcolor[rgb]{0.00,0.00,1.00}{##1}}}
\@namedef{PY@tok@ne}{\let\PY@bf=\textbf\def\PY@tc##1{\textcolor[rgb]{0.80,0.25,0.22}{##1}}}
\@namedef{PY@tok@nv}{\def\PY@tc##1{\textcolor[rgb]{0.10,0.09,0.49}{##1}}}
\@namedef{PY@tok@no}{\def\PY@tc##1{\textcolor[rgb]{0.53,0.00,0.00}{##1}}}
\@namedef{PY@tok@nl}{\def\PY@tc##1{\textcolor[rgb]{0.46,0.46,0.00}{##1}}}
\@namedef{PY@tok@ni}{\let\PY@bf=\textbf\def\PY@tc##1{\textcolor[rgb]{0.44,0.44,0.44}{##1}}}
\@namedef{PY@tok@na}{\def\PY@tc##1{\textcolor[rgb]{0.41,0.47,0.13}{##1}}}
\@namedef{PY@tok@nt}{\let\PY@bf=\textbf\def\PY@tc##1{\textcolor[rgb]{0.00,0.50,0.00}{##1}}}
\@namedef{PY@tok@nd}{\def\PY@tc##1{\textcolor[rgb]{0.67,0.13,1.00}{##1}}}
\@namedef{PY@tok@s}{\def\PY@tc##1{\textcolor[rgb]{0.73,0.13,0.13}{##1}}}
\@namedef{PY@tok@sd}{\let\PY@it=\textit\def\PY@tc##1{\textcolor[rgb]{0.73,0.13,0.13}{##1}}}
\@namedef{PY@tok@si}{\let\PY@bf=\textbf\def\PY@tc##1{\textcolor[rgb]{0.64,0.35,0.47}{##1}}}
\@namedef{PY@tok@se}{\let\PY@bf=\textbf\def\PY@tc##1{\textcolor[rgb]{0.67,0.36,0.12}{##1}}}
\@namedef{PY@tok@sr}{\def\PY@tc##1{\textcolor[rgb]{0.64,0.35,0.47}{##1}}}
\@namedef{PY@tok@ss}{\def\PY@tc##1{\textcolor[rgb]{0.10,0.09,0.49}{##1}}}
\@namedef{PY@tok@sx}{\def\PY@tc##1{\textcolor[rgb]{0.00,0.50,0.00}{##1}}}
\@namedef{PY@tok@m}{\def\PY@tc##1{\textcolor[rgb]{0.40,0.40,0.40}{##1}}}
\@namedef{PY@tok@gh}{\let\PY@bf=\textbf\def\PY@tc##1{\textcolor[rgb]{0.00,0.00,0.50}{##1}}}
\@namedef{PY@tok@gu}{\let\PY@bf=\textbf\def\PY@tc##1{\textcolor[rgb]{0.50,0.00,0.50}{##1}}}
\@namedef{PY@tok@gd}{\def\PY@tc##1{\textcolor[rgb]{0.63,0.00,0.00}{##1}}}
\@namedef{PY@tok@gi}{\def\PY@tc##1{\textcolor[rgb]{0.00,0.52,0.00}{##1}}}
\@namedef{PY@tok@gr}{\def\PY@tc##1{\textcolor[rgb]{0.89,0.00,0.00}{##1}}}
\@namedef{PY@tok@ge}{\let\PY@it=\textit}
\@namedef{PY@tok@gs}{\let\PY@bf=\textbf}
\@namedef{PY@tok@gp}{\let\PY@bf=\textbf\def\PY@tc##1{\textcolor[rgb]{0.00,0.00,0.50}{##1}}}
\@namedef{PY@tok@go}{\def\PY@tc##1{\textcolor[rgb]{0.44,0.44,0.44}{##1}}}
\@namedef{PY@tok@gt}{\def\PY@tc##1{\textcolor[rgb]{0.00,0.27,0.87}{##1}}}
\@namedef{PY@tok@err}{\def\PY@bc##1{{\setlength{\fboxsep}{\string -\fboxrule}\fcolorbox[rgb]{1.00,0.00,0.00}{1,1,1}{\strut ##1}}}}
\@namedef{PY@tok@kc}{\let\PY@bf=\textbf\def\PY@tc##1{\textcolor[rgb]{0.00,0.50,0.00}{##1}}}
\@namedef{PY@tok@kd}{\let\PY@bf=\textbf\def\PY@tc##1{\textcolor[rgb]{0.00,0.50,0.00}{##1}}}
\@namedef{PY@tok@kn}{\let\PY@bf=\textbf\def\PY@tc##1{\textcolor[rgb]{0.00,0.50,0.00}{##1}}}
\@namedef{PY@tok@kr}{\let\PY@bf=\textbf\def\PY@tc##1{\textcolor[rgb]{0.00,0.50,0.00}{##1}}}
\@namedef{PY@tok@bp}{\def\PY@tc##1{\textcolor[rgb]{0.00,0.50,0.00}{##1}}}
\@namedef{PY@tok@fm}{\def\PY@tc##1{\textcolor[rgb]{0.00,0.00,1.00}{##1}}}
\@namedef{PY@tok@vc}{\def\PY@tc##1{\textcolor[rgb]{0.10,0.09,0.49}{##1}}}
\@namedef{PY@tok@vg}{\def\PY@tc##1{\textcolor[rgb]{0.10,0.09,0.49}{##1}}}
\@namedef{PY@tok@vi}{\def\PY@tc##1{\textcolor[rgb]{0.10,0.09,0.49}{##1}}}
\@namedef{PY@tok@vm}{\def\PY@tc##1{\textcolor[rgb]{0.10,0.09,0.49}{##1}}}
\@namedef{PY@tok@sa}{\def\PY@tc##1{\textcolor[rgb]{0.73,0.13,0.13}{##1}}}
\@namedef{PY@tok@sb}{\def\PY@tc##1{\textcolor[rgb]{0.73,0.13,0.13}{##1}}}
\@namedef{PY@tok@sc}{\def\PY@tc##1{\textcolor[rgb]{0.73,0.13,0.13}{##1}}}
\@namedef{PY@tok@dl}{\def\PY@tc##1{\textcolor[rgb]{0.73,0.13,0.13}{##1}}}
\@namedef{PY@tok@s2}{\def\PY@tc##1{\textcolor[rgb]{0.73,0.13,0.13}{##1}}}
\@namedef{PY@tok@sh}{\def\PY@tc##1{\textcolor[rgb]{0.73,0.13,0.13}{##1}}}
\@namedef{PY@tok@s1}{\def\PY@tc##1{\textcolor[rgb]{0.73,0.13,0.13}{##1}}}
\@namedef{PY@tok@mb}{\def\PY@tc##1{\textcolor[rgb]{0.40,0.40,0.40}{##1}}}
\@namedef{PY@tok@mf}{\def\PY@tc##1{\textcolor[rgb]{0.40,0.40,0.40}{##1}}}
\@namedef{PY@tok@mh}{\def\PY@tc##1{\textcolor[rgb]{0.40,0.40,0.40}{##1}}}
\@namedef{PY@tok@mi}{\def\PY@tc##1{\textcolor[rgb]{0.40,0.40,0.40}{##1}}}
\@namedef{PY@tok@il}{\def\PY@tc##1{\textcolor[rgb]{0.40,0.40,0.40}{##1}}}
\@namedef{PY@tok@mo}{\def\PY@tc##1{\textcolor[rgb]{0.40,0.40,0.40}{##1}}}
\@namedef{PY@tok@ch}{\let\PY@it=\textit\def\PY@tc##1{\textcolor[rgb]{0.24,0.48,0.48}{##1}}}
\@namedef{PY@tok@cm}{\let\PY@it=\textit\def\PY@tc##1{\textcolor[rgb]{0.24,0.48,0.48}{##1}}}
\@namedef{PY@tok@cpf}{\let\PY@it=\textit\def\PY@tc##1{\textcolor[rgb]{0.24,0.48,0.48}{##1}}}
\@namedef{PY@tok@c1}{\let\PY@it=\textit\def\PY@tc##1{\textcolor[rgb]{0.24,0.48,0.48}{##1}}}
\@namedef{PY@tok@cs}{\let\PY@it=\textit\def\PY@tc##1{\textcolor[rgb]{0.24,0.48,0.48}{##1}}}

\def\PYZbs{\char`\\}
\def\PYZus{\char`\_}
\def\PYZob{\char`\{}
\def\PYZcb{\char`\}}
\def\PYZca{\char`\^}
\def\PYZam{\char`\&}
\def\PYZlt{\char`\<}
\def\PYZgt{\char`\>}
\def\PYZsh{\char`\#}
\def\PYZpc{\char`\%}
\def\PYZdl{\char`\$}
\def\PYZhy{\char`\-}
\def\PYZsq{\char`\'}
\def\PYZdq{\char`\"}
\def\PYZti{\char`\~}
% for compatibility with earlier versions
\def\PYZat{@}
\def\PYZlb{[}
\def\PYZrb{]}
\makeatother


    % For linebreaks inside Verbatim environment from package fancyvrb.
    \makeatletter
        \newbox\Wrappedcontinuationbox
        \newbox\Wrappedvisiblespacebox
        \newcommand*\Wrappedvisiblespace {\textcolor{red}{\textvisiblespace}}
        \newcommand*\Wrappedcontinuationsymbol {\textcolor{red}{\llap{\tiny$\m@th\hookrightarrow$}}}
        \newcommand*\Wrappedcontinuationindent {3ex }
        \newcommand*\Wrappedafterbreak {\kern\Wrappedcontinuationindent\copy\Wrappedcontinuationbox}
        % Take advantage of the already applied Pygments mark-up to insert
        % potential linebreaks for TeX processing.
        %        {, <, #, %, $, ' and ": go to next line.
        %        _, }, ^, &, >, - and ~: stay at end of broken line.
        % Use of \textquotesingle for straight quote.
        \newcommand*\Wrappedbreaksatspecials {%
            \def\PYGZus{\discretionary{\char`\_}{\Wrappedafterbreak}{\char`\_}}%
            \def\PYGZob{\discretionary{}{\Wrappedafterbreak\char`\{}{\char`\{}}%
            \def\PYGZcb{\discretionary{\char`\}}{\Wrappedafterbreak}{\char`\}}}%
            \def\PYGZca{\discretionary{\char`\^}{\Wrappedafterbreak}{\char`\^}}%
            \def\PYGZam{\discretionary{\char`\&}{\Wrappedafterbreak}{\char`\&}}%
            \def\PYGZlt{\discretionary{}{\Wrappedafterbreak\char`\<}{\char`\<}}%
            \def\PYGZgt{\discretionary{\char`\>}{\Wrappedafterbreak}{\char`\>}}%
            \def\PYGZsh{\discretionary{}{\Wrappedafterbreak\char`\#}{\char`\#}}%
            \def\PYGZpc{\discretionary{}{\Wrappedafterbreak\char`\%}{\char`\%}}%
            \def\PYGZdl{\discretionary{}{\Wrappedafterbreak\char`\$}{\char`\$}}%
            \def\PYGZhy{\discretionary{\char`\-}{\Wrappedafterbreak}{\char`\-}}%
            \def\PYGZsq{\discretionary{}{\Wrappedafterbreak\textquotesingle}{\textquotesingle}}%
            \def\PYGZdq{\discretionary{}{\Wrappedafterbreak\char`\"}{\char`\"}}%
            \def\PYGZti{\discretionary{\char`\~}{\Wrappedafterbreak}{\char`\~}}%
        }
        % Some characters . , ; ? ! / are not pygmentized.
        % This macro makes them "active" and they will insert potential linebreaks
        \newcommand*\Wrappedbreaksatpunct {%
            \lccode`\~`\.\lowercase{\def~}{\discretionary{\hbox{\char`\.}}{\Wrappedafterbreak}{\hbox{\char`\.}}}%
            \lccode`\~`\,\lowercase{\def~}{\discretionary{\hbox{\char`\,}}{\Wrappedafterbreak}{\hbox{\char`\,}}}%
            \lccode`\~`\;\lowercase{\def~}{\discretionary{\hbox{\char`\;}}{\Wrappedafterbreak}{\hbox{\char`\;}}}%
            \lccode`\~`\:\lowercase{\def~}{\discretionary{\hbox{\char`\:}}{\Wrappedafterbreak}{\hbox{\char`\:}}}%
            \lccode`\~`\?\lowercase{\def~}{\discretionary{\hbox{\char`\?}}{\Wrappedafterbreak}{\hbox{\char`\?}}}%
            \lccode`\~`\!\lowercase{\def~}{\discretionary{\hbox{\char`\!}}{\Wrappedafterbreak}{\hbox{\char`\!}}}%
            \lccode`\~`\/\lowercase{\def~}{\discretionary{\hbox{\char`\/}}{\Wrappedafterbreak}{\hbox{\char`\/}}}%
            \catcode`\.\active
            \catcode`\,\active
            \catcode`\;\active
            \catcode`\:\active
            \catcode`\?\active
            \catcode`\!\active
            \catcode`\/\active
            \lccode`\~`\~
        }
    \makeatother

    \let\OriginalVerbatim=\Verbatim
    \makeatletter
    \renewcommand{\Verbatim}[1][1]{%
        %\parskip\z@skip
        \sbox\Wrappedcontinuationbox {\Wrappedcontinuationsymbol}%
        \sbox\Wrappedvisiblespacebox {\FV@SetupFont\Wrappedvisiblespace}%
        \def\FancyVerbFormatLine ##1{\hsize\linewidth
            \vtop{\raggedright\hyphenpenalty\z@\exhyphenpenalty\z@
                \doublehyphendemerits\z@\finalhyphendemerits\z@
                \strut ##1\strut}%
        }%
        % If the linebreak is at a space, the latter will be displayed as visible
        % space at end of first line, and a continuation symbol starts next line.
        % Stretch/shrink are however usually zero for typewriter font.
        \def\FV@Space {%
            \nobreak\hskip\z@ plus\fontdimen3\font minus\fontdimen4\font
            \discretionary{\copy\Wrappedvisiblespacebox}{\Wrappedafterbreak}
            {\kern\fontdimen2\font}%
        }%

        % Allow breaks at special characters using \PYG... macros.
        \Wrappedbreaksatspecials
        % Breaks at punctuation characters . , ; ? ! and / need catcode=\active
        \OriginalVerbatim[#1,codes*=\Wrappedbreaksatpunct]%
    }
    \makeatother

    % Exact colors from NB
    \definecolor{incolor}{HTML}{303F9F}
    \definecolor{outcolor}{HTML}{D84315}
    \definecolor{cellborder}{HTML}{CFCFCF}
    \definecolor{cellbackground}{HTML}{F7F7F7}

    % prompt
    \makeatletter
    \newcommand{\boxspacing}{\kern\kvtcb@left@rule\kern\kvtcb@boxsep}
    \makeatother
    \newcommand{\prompt}[4]{
        {\ttfamily\llap{{\color{#2}[#3]:\hspace{3pt}#4}}\vspace{-\baselineskip}}
    }
    

    
    % Prevent overflowing lines due to hard-to-break entities
    \sloppy
    % Setup hyperref package
    \hypersetup{
      breaklinks=true,  % so long urls are correctly broken across lines
      colorlinks=true,
      urlcolor=urlcolor,
      linkcolor=linkcolor,
      citecolor=citecolor,
      }
    % Slightly bigger margins than the latex defaults
    
    \geometry{verbose,tmargin=1in,bmargin=1in,lmargin=1in,rmargin=1in}
    
    

\begin{document}
    
    \maketitle
    
    

    
    \subsection{\texorpdfstring{\textbf{Examen 1 - Análisis
Cuantitativo}}{Examen 1 - Análisis Cuantitativo}}\label{examen-1---anuxe1lisis-cuantitativo}

    \subsection{\texorpdfstring{\textbf{{ Solución Pregunta 5
}}}{ Solución Pregunta 5 }}\label{soluciuxf3n-pregunta-5}

\paragraph{Se desea predecir la resistencia a la compresión del concreto
(Concrete compressive strength) en función de diferentes variables
predictoras como el cemento (Cement), la escoria (Slag), la ceniza
volante (Fly ash), el agua (Water), el superplastificante
(Superplasticizer), el agregado grueso (Coarse aggregate) y el agregado
fino (Fine aggregate). Para ello se dispone de un conjunto de datos con
1030 observaciones. Se desea construir un modelo de regresión lineal
múltiple para predecir la resistencia a la compresión del concreto en
función de las variables
predictoras.}\label{se-desea-predecir-la-resistencia-a-la-compresiuxf3n-del-concreto-concrete-compressive-strength-en-funciuxf3n-de-diferentes-variables-predictoras-como-el-cemento-cement-la-escoria-slag-la-ceniza-volante-fly-ash-el-agua-water-el-superplastificante-superplasticizer-el-agregado-grueso-coarse-aggregate-y-el-agregado-fino-fine-aggregate.-para-ello-se-dispone-de-un-conjunto-de-datos-con-1030-observaciones.-se-desea-construir-un-modelo-de-regresiuxf3n-lineal-muxfaltiple-para-predecir-la-resistencia-a-la-compresiuxf3n-del-concreto-en-funciuxf3n-de-las-variables-predictoras.}

    5.1. Cargar los datos del archivo ``Concrete\_Data.xls'' y examinar las
características del conjunto de datos.

\textless/span

    \begin{tcolorbox}[breakable, size=fbox, boxrule=1pt, pad at break*=1mm,colback=cellbackground, colframe=cellborder]
\prompt{In}{incolor}{1}{\boxspacing}
\begin{Verbatim}[commandchars=\\\{\}]
\PY{c+c1}{\PYZsh{} Carga de las librerías necesarias para llevar a cabo los procedimientos}

\PY{c+c1}{\PYZsh{}Librerías}

\PY{k+kn}{import} \PY{n+nn}{pandas} \PY{k}{as} \PY{n+nn}{pd}
\PY{k+kn}{import} \PY{n+nn}{numpy} \PY{k}{as} \PY{n+nn}{np}

\PY{c+c1}{\PYZsh{} Librerías de visualización}

\PY{k+kn}{import} \PY{n+nn}{matplotlib}\PY{n+nn}{.}\PY{n+nn}{pyplot} \PY{k}{as} \PY{n+nn}{plt}
\PY{k+kn}{import} \PY{n+nn}{seaborn} \PY{k}{as} \PY{n+nn}{sns}
\PY{k+kn}{import} \PY{n+nn}{plotly}\PY{n+nn}{.}\PY{n+nn}{express} \PY{k}{as} \PY{n+nn}{px}
\PY{k+kn}{import} \PY{n+nn}{plotly}\PY{n+nn}{.}\PY{n+nn}{graph\PYZus{}objects} \PY{k}{as} \PY{n+nn}{go}


\PY{c+c1}{\PYZsh{} Matplotlib y  seaborn para gráficos}
\PY{k+kn}{import} \PY{n+nn}{matplotlib}\PY{n+nn}{.}\PY{n+nn}{pyplot} \PY{k}{as} \PY{n+nn}{plt}
\PY{k+kn}{import} \PY{n+nn}{matplotlib}


\PY{k+kn}{import} \PY{n+nn}{seaborn} \PY{k}{as} \PY{n+nn}{sns}

\PY{c+c1}{\PYZsh{} No muestra warnings que no son determinantes para el proceso}

\PY{k+kn}{import} \PY{n+nn}{warnings}
\PY{n}{warnings}\PY{o}{.}\PY{n}{filterwarnings}\PY{p}{(}\PY{l+s+s2}{\PYZdq{}}\PY{l+s+s2}{ignore}\PY{l+s+s2}{\PYZdq{}}\PY{p}{)}
\end{Verbatim}
\end{tcolorbox}

    \begin{tcolorbox}[breakable, size=fbox, boxrule=1pt, pad at break*=1mm,colback=cellbackground, colframe=cellborder]
\prompt{In}{incolor}{2}{\boxspacing}
\begin{Verbatim}[commandchars=\\\{\}]
\PY{c+c1}{\PYZsh{} Se lleva a cabo la carga de la información necesaria para trabajar el caso propuesto}

\PY{n}{data\PYZus{}concrete} \PY{o}{=} \PY{n}{pd}\PY{o}{.}\PY{n}{read\PYZus{}excel}\PY{p}{(}\PY{l+s+s2}{\PYZdq{}}\PY{l+s+s2}{Concrete\PYZus{}Data.xls}\PY{l+s+s2}{\PYZdq{}}\PY{p}{,} \PY{n}{sheet\PYZus{}name}\PY{o}{=}\PY{l+s+s2}{\PYZdq{}}\PY{l+s+s2}{Sheet1}\PY{l+s+s2}{\PYZdq{}}\PY{p}{)}
\end{Verbatim}
\end{tcolorbox}

    5.2. Realizar un análisis exploratorio de los datos para entender la
relación entre las variables predictoras y la variable respuesta.

\textless/span

    Inicialmente, se lleva a cabo una identificación de los \textbf{tipos de
variables} del DataSet.

    \begin{longtable}[]{@{}
  >{\centering\arraybackslash}p{(\columnwidth - 6\tabcolsep) * \real{0.2500}}
  >{\centering\arraybackslash}p{(\columnwidth - 6\tabcolsep) * \real{0.2500}}
  >{\centering\arraybackslash}p{(\columnwidth - 6\tabcolsep) * \real{0.2500}}
  >{\centering\arraybackslash}p{(\columnwidth - 6\tabcolsep) * \real{0.2500}}@{}}
\toprule\noalign{}
\begin{minipage}[b]{\linewidth}\centering
Columna
\end{minipage} & \begin{minipage}[b]{\linewidth}\centering
Descripción
\end{minipage} & \begin{minipage}[b]{\linewidth}\centering
Tipo
\end{minipage} & \begin{minipage}[b]{\linewidth}\centering
Rango / No.~Categorías
\end{minipage} \\
\midrule\noalign{}
\endhead
\bottomrule\noalign{}
\endlastfoot
Cement & Cantidad de cemento en kg/m3 de la mezcla & Cuantitativa
Continua & Rango: {[}102,0 - 540,0{]} \\
Blast Furnace Slag & Cantidad de escoria de alto horno en kg/m3 de la
mezcla & Cuantitativa Continua & Rango: {[}0,0 - 359,4{]} \\
Fly Ash & Cantidad de ceniza volante en kg/m3 de la mezcla &
Cuantitativa Continua & Rango: {[}0,0 - 200,1{]} \\
Water & Cantidad de agua en kg/m3 de la mezcla & Cuantitativa Continua &
Rango: {[}121,8 - 247,0{]} \\
Superplasticizer & Cantidad de superplastificante en kg/m3 de la mezcla
& Cuantitativa Continua & Rango: {[}0,0 - 32,2{]} \\
Coarse Aggregate & Cantidad de agregado grueso en kg/m3 de la mezcla &
Cuantitativa Continua & Rango: {[}801,0 - 1145,0{]} \\
Fine Aggregate & Cantidad de agregado fino en kg/m3 de la mezcla &
Cuantitativa Continua & Rango: {[}594,0 - 992,6{]} \\
Age & Edad del concreto en días & Cuantitativa Continua & Rango:
{[}594,0 - 992,6{]} \\
Concrete compressive strength & Resistencia a la comprensión del
concreto en MPa & Cuantitativa Continua & Rango: {[}2,33 - 82,60{]} \\
\end{longtable}

    \textbf{Dominio de los datos:} teniendo en cuenta la información
proporcionada, el contexto de los datos hace alusión a los materiales o
estructuras de concreto que se utilizan para elaborar proyectos de
construcción como edificaciones, viviendas, entre otros.

Por su parte, las variables como el cemento, la escoria, ceniza volante,
agua, superplastificante, agregado grueso y fino, corresponden a
aquellos componentes que se usan para llevar a cabo mezclas de concreto,
que son indispensables para asegurar caracterísitcas de calidad,
durabilidad y resistencia en los contextos en los que dichos materiales
se utilicen.

    \paragraph{Revisión de la estructura del
dataset:}\label{revisiuxf3n-de-la-estructura-del-dataset}

    \begin{tcolorbox}[breakable, size=fbox, boxrule=1pt, pad at break*=1mm,colback=cellbackground, colframe=cellborder]
\prompt{In}{incolor}{3}{\boxspacing}
\begin{Verbatim}[commandchars=\\\{\}]
\PY{n}{data\PYZus{}concrete}\PY{o}{.}\PY{n}{info}\PY{p}{(}\PY{p}{)}
\end{Verbatim}
\end{tcolorbox}

    \begin{Verbatim}[commandchars=\\\{\}]
<class 'pandas.core.frame.DataFrame'>
RangeIndex: 1030 entries, 0 to 1029
Data columns (total 9 columns):
 \#   Column                                                 Non-Null Count
Dtype
---  ------                                                 --------------
-----
 0   Cement (component 1)(kg in a m\^{}3 mixture)              1030 non-null
float64
 1   Blast Furnace Slag (component 2)(kg in a m\^{}3 mixture)  1030 non-null
float64
 2   Fly Ash (component 3)(kg in a m\^{}3 mixture)             1030 non-null
float64
 3   Water  (component 4)(kg in a m\^{}3 mixture)              1030 non-null
float64
 4   Superplasticizer (component 5)(kg in a m\^{}3 mixture)    1030 non-null
float64
 5   Coarse Aggregate  (component 6)(kg in a m\^{}3 mixture)   1030 non-null
float64
 6   Fine Aggregate (component 7)(kg in a m\^{}3 mixture)      1030 non-null
float64
 7   Age (day)                                              1030 non-null
int64
 8   Concrete compressive strength(MPa, megapascals)        1030 non-null
float64
dtypes: float64(8), int64(1)
memory usage: 72.6 KB
    \end{Verbatim}

    \textbf{Análisis:} se identificó que en el dataset otorgado existen un
total de 1.030 registros que recogen diferentes composiciones de los
materiales mencionados anteriormente.

Por su parte, se encontraron un total de 9 columnas que corresponden a
las 8 variables que posteriormente se utilizaran como posibles variables
predictoras de la novena variable del conjunto de datos: resistencia a
la comprensión del concreto.

Finalmente, se destaca como se vio en el cuadro anterior que todas las
variables son cuantitativas continuas y no se identificó la presencia de
datos nulos.

    \begin{tcolorbox}[breakable, size=fbox, boxrule=1pt, pad at break*=1mm,colback=cellbackground, colframe=cellborder]
\prompt{In}{incolor}{4}{\boxspacing}
\begin{Verbatim}[commandchars=\\\{\}]
\PY{c+c1}{\PYZsh{}Buscando optimizar la visualización de la información, renombramos las variables del dataset}

\PY{n}{new\PYZus{}columns} \PY{o}{=} \PY{p}{\PYZob{}}

    \PY{l+s+s1}{\PYZsq{}}\PY{l+s+s1}{Cement (component 1)(kg in a m\PYZca{}3 mixture)}\PY{l+s+s1}{\PYZsq{}}\PY{p}{:} \PY{l+s+s1}{\PYZsq{}}\PY{l+s+s1}{Cement}\PY{l+s+s1}{\PYZsq{}}\PY{p}{,}
    \PY{l+s+s1}{\PYZsq{}}\PY{l+s+s1}{Blast Furnace Slag (component 2)(kg in a m\PYZca{}3 mixture)}\PY{l+s+s1}{\PYZsq{}}\PY{p}{:} \PY{l+s+s1}{\PYZsq{}}\PY{l+s+s1}{Slag}\PY{l+s+s1}{\PYZsq{}}\PY{p}{,}
    \PY{l+s+s1}{\PYZsq{}}\PY{l+s+s1}{Fly Ash (component 3)(kg in a m\PYZca{}3 mixture)}\PY{l+s+s1}{\PYZsq{}}\PY{p}{:} \PY{l+s+s1}{\PYZsq{}}\PY{l+s+s1}{Fly\PYZus{}Ash}\PY{l+s+s1}{\PYZsq{}}\PY{p}{,}
    \PY{l+s+s1}{\PYZsq{}}\PY{l+s+s1}{Water  (component 4)(kg in a m\PYZca{}3 mixture)}\PY{l+s+s1}{\PYZsq{}}\PY{p}{:} \PY{l+s+s1}{\PYZsq{}}\PY{l+s+s1}{Water}\PY{l+s+s1}{\PYZsq{}}\PY{p}{,}
    \PY{l+s+s1}{\PYZsq{}}\PY{l+s+s1}{Superplasticizer (component 5)(kg in a m\PYZca{}3 mixture)}\PY{l+s+s1}{\PYZsq{}}\PY{p}{:} \PY{l+s+s1}{\PYZsq{}}\PY{l+s+s1}{Superplasticizer}\PY{l+s+s1}{\PYZsq{}}\PY{p}{,}
    \PY{l+s+s1}{\PYZsq{}}\PY{l+s+s1}{Coarse Aggregate  (component 6)(kg in a m\PYZca{}3 mixture)}\PY{l+s+s1}{\PYZsq{}}\PY{p}{:} \PY{l+s+s1}{\PYZsq{}}\PY{l+s+s1}{Coarse}\PY{l+s+s1}{\PYZsq{}}\PY{p}{,}
    \PY{l+s+s1}{\PYZsq{}}\PY{l+s+s1}{Fine Aggregate (component 7)(kg in a m\PYZca{}3 mixture)}\PY{l+s+s1}{\PYZsq{}}\PY{p}{:} \PY{l+s+s1}{\PYZsq{}}\PY{l+s+s1}{Fine}\PY{l+s+s1}{\PYZsq{}}\PY{p}{,}
    \PY{l+s+s1}{\PYZsq{}}\PY{l+s+s1}{Age (day)}\PY{l+s+s1}{\PYZsq{}}\PY{p}{:} \PY{l+s+s1}{\PYZsq{}}\PY{l+s+s1}{Age}\PY{l+s+s1}{\PYZsq{}}\PY{p}{,}
    \PY{l+s+s1}{\PYZsq{}}\PY{l+s+s1}{Concrete compressive strength(MPa, megapascals) }\PY{l+s+s1}{\PYZsq{}}\PY{p}{:} \PY{l+s+s1}{\PYZsq{}}\PY{l+s+s1}{Compressive strength}\PY{l+s+s1}{\PYZsq{}}\PY{p}{,}
\PY{p}{\PYZcb{}}

\PY{c+c1}{\PYZsh{} Ejecutando las nuevas etiquetas}

\PY{n}{df\PYZus{}concrete} \PY{o}{=} \PY{n}{data\PYZus{}concrete}\PY{o}{.}\PY{n}{rename}\PY{p}{(}\PY{n}{columns}\PY{o}{=}\PY{n}{new\PYZus{}columns}\PY{p}{)}

\PY{n}{df\PYZus{}concrete}\PY{o}{.}\PY{n}{head}\PY{p}{(}\PY{l+m+mi}{5}\PY{p}{)}
\end{Verbatim}
\end{tcolorbox}

            \begin{tcolorbox}[breakable, size=fbox, boxrule=.5pt, pad at break*=1mm, opacityfill=0]
\prompt{Out}{outcolor}{4}{\boxspacing}
\begin{Verbatim}[commandchars=\\\{\}]
   Cement   Slag  Fly\_Ash  Water  Superplasticizer  Coarse   Fine  Age  \textbackslash{}
0   540.0    0.0      0.0  162.0               2.5  1040.0  676.0   28
1   540.0    0.0      0.0  162.0               2.5  1055.0  676.0   28
2   332.5  142.5      0.0  228.0               0.0   932.0  594.0  270
3   332.5  142.5      0.0  228.0               0.0   932.0  594.0  365
4   198.6  132.4      0.0  192.0               0.0   978.4  825.5  360

   Compressive strength
0             79.986111
1             61.887366
2             40.269535
3             41.052780
4             44.296075
\end{Verbatim}
\end{tcolorbox}
        
    \paragraph{Estadísticas Descriptivas (se analizan algunas variables del
dataset):}\label{estaduxedsticas-descriptivas-se-analizan-algunas-variables-del-dataset}

    \begin{tcolorbox}[breakable, size=fbox, boxrule=1pt, pad at break*=1mm,colback=cellbackground, colframe=cellborder]
\prompt{In}{incolor}{5}{\boxspacing}
\begin{Verbatim}[commandchars=\\\{\}]
\PY{c+c1}{\PYZsh{} Construcción de una función que nos permita calcular las estadísticas descriptivas:}

\PY{k+kn}{from} \PY{n+nn}{scipy}\PY{n+nn}{.}\PY{n+nn}{stats} \PY{k+kn}{import} \PY{n}{skew}\PY{p}{,} \PY{n}{kurtosis} 

\PY{k}{def} \PY{n+nf}{tabla\PYZus{}descriptivas\PYZus{}completa} \PY{p}{(}\PY{n}{columnas}\PY{p}{)}\PY{p}{:}

    \PY{n}{tabla\PYZus{}descriptivas\PYZus{}completa}\PY{o}{=}\PY{n}{pd}\PY{o}{.}\PY{n}{DataFrame}\PY{p}{(}\PY{n}{columnas}\PY{o}{.}\PY{n}{describe}\PY{p}{(}\PY{p}{)}\PY{p}{)}
    \PY{n}{tabla\PYZus{}descriptivas\PYZus{}completa}\PY{o}{.}\PY{n}{loc}\PY{p}{[}\PY{l+s+s1}{\PYZsq{}}\PY{l+s+s1}{coef. variation}\PY{l+s+s1}{\PYZsq{}}\PY{p}{]}\PY{o}{=}\PY{n}{columnas}\PY{o}{.}\PY{n}{std}\PY{p}{(}\PY{p}{)}\PY{o}{/}\PY{n}{columnas}\PY{o}{.}\PY{n}{mean}\PY{p}{(}\PY{p}{)}
    \PY{n}{tabla\PYZus{}descriptivas\PYZus{}completa}\PY{o}{.}\PY{n}{loc}\PY{p}{[}\PY{l+s+s1}{\PYZsq{}}\PY{l+s+s1}{skew}\PY{l+s+s1}{\PYZsq{}}\PY{p}{]}\PY{o}{=}\PY{n}{skew}\PY{p}{(}\PY{n}{columnas}\PY{p}{)}
    \PY{n}{tabla\PYZus{}descriptivas\PYZus{}completa}\PY{o}{.}\PY{n}{loc}\PY{p}{[}\PY{l+s+s1}{\PYZsq{}}\PY{l+s+s1}{kurtosis}\PY{l+s+s1}{\PYZsq{}}\PY{p}{]}\PY{o}{=}\PY{n}{kurtosis}\PY{p}{(}\PY{n}{columnas}\PY{p}{)}

    \PY{k}{return} \PY{n}{tabla\PYZus{}descriptivas\PYZus{}completa}
\end{Verbatim}
\end{tcolorbox}

    \begin{tcolorbox}[breakable, size=fbox, boxrule=1pt, pad at break*=1mm,colback=cellbackground, colframe=cellborder]
\prompt{In}{incolor}{6}{\boxspacing}
\begin{Verbatim}[commandchars=\\\{\}]
\PY{n}{tabla\PYZus{}descriptivas\PYZus{}completa}\PY{p}{(}\PY{n}{df\PYZus{}concrete}\PY{p}{)}
\end{Verbatim}
\end{tcolorbox}

            \begin{tcolorbox}[breakable, size=fbox, boxrule=.5pt, pad at break*=1mm, opacityfill=0]
\prompt{Out}{outcolor}{6}{\boxspacing}
\begin{Verbatim}[commandchars=\\\{\}]
                      Cement         Slag      Fly\_Ash        Water  \textbackslash{}
count            1030.000000  1030.000000  1030.000000  1030.000000
mean              281.165631    73.895485    54.187136   181.566359
std               104.507142    86.279104    63.996469    21.355567
min               102.000000     0.000000     0.000000   121.750000
25\%               192.375000     0.000000     0.000000   164.900000
50\%               272.900000    22.000000     0.000000   185.000000
75\%               350.000000   142.950000   118.270000   192.000000
max               540.000000   359.400000   200.100000   247.000000
coef. variation     0.371692     1.167583     1.181027     0.117619
skew                0.508775     0.799571     0.536662     0.074216
kurtosis           -0.523959    -0.511495    -1.327884     0.116262

                 Superplasticizer       Coarse         Fine          Age  \textbackslash{}
count                 1030.000000  1030.000000  1030.000000  1030.000000
mean                     6.203112   972.918592   773.578883    45.662136
std                      5.973492    77.753818    80.175427    63.169912
min                      0.000000   801.000000   594.000000     1.000000
25\%                      0.000000   932.000000   730.950000     7.000000
50\%                      6.350000   968.000000   779.510000    28.000000
75\%                     10.160000  1029.400000   824.000000    56.000000
max                     32.200000  1145.000000   992.600000   365.000000
coef. variation          0.962983     0.079918     0.103642     1.383420
skew                     0.906790    -0.040148    -0.252611     3.264415
kurtosis                 1.400515    -0.601916    -0.107489    12.104177

                 Compressive strength
count                     1030.000000
mean                        35.817836
std                         16.705679
min                          2.331808
25\%                         23.707115
50\%                         34.442774
75\%                         46.136287
max                         82.599225
coef. variation              0.466407
skew                         0.416315
kurtosis                    -0.318142
\end{Verbatim}
\end{tcolorbox}
        
    \textbf{Análisis:}

En el caso del cemento, se observa que la cantidad media necesaria que
expone este primer componente para la mezcla de concreto corresponde a
281.1 kg/m3. Por su parte, el coeficiente de variación correspondiente a
37,1\%, sugiere una dispersión moderada de los datos.

Con respecto a la escoria, se evidencia que la cantidad promedio en
términos de kg/m3 es de 73,9. En este caso, se encontró una alta
dispersión en los datos, explicada por un coeficiente de variación de
116,8\%. Finalmente, se destaca que el sesgo de 0,80 sugiere una
asimetría hacia la derecha (positiva).

Teniendo en cuenta la tabla, el 75\% de las mezclas identificadas en el
dataset vinculan entre sus componentes una cantidad de ceniza volante
igual o menor a 118,3 kg/m3, por su parte, el coefieciente de variación
(118,1\%) evidencia la presencia de alta disperisón de los datos.

En el caso del agua, se observa que la cantidad media utilizada en las
diferentes mezclas de concreto evienciadas en el dataset corresponde
181,6 kilogramos por metro cúbico. Aquí, se destaca que la curtósis
correspondiente a 0,11 sugiere una forma de distribución puntiaguda en
el centro de los datos.

Dado el contexto, la edad del concreto actúa como una variable
interesante para estudiar la capacidad que tiene el mismo para soportar
cargas. En ese sentido, se encontró que la edad media del concreto en
este dataset corresponde a 45,7 días. No obstante, se observó también
que hay una alta dispersión de los datos explicada por un coeficiente de
variación de 138,3\%. Finalmente, el 75\% de las observaciones evidencia
un edad igual o menor a 56 días.

Finalmente, concentrandonos en la variable objetivo, se identificó que
la resistencia media a la comprensión del concreto se acerca a los 35,8
MPa, también, a través del coeficiente de variación (46,6\%) se
evidencia una dispersión moderada de los datos. Finalmente, se destaca
que la resistencia máxima a la comprensión de este material es de 82,6
MPa.

    \paragraph{Análisis Gráfico:}\label{anuxe1lisis-gruxe1fico}

Se realizará un análisis basado en la graficación de histogramas para
conocer visualmente cómo se comparta la distribución de los datos para
cada una de las variables en el Dataset. También, se propondrá una
análisis a través de gráficos de caja para revisar la presencia de datos
nulos y finalmente se analizará cuál es el comportamiento de la
correlación entre las variables explicativas y la variable objetivo.

    \begin{tcolorbox}[breakable, size=fbox, boxrule=1pt, pad at break*=1mm,colback=cellbackground, colframe=cellborder]
\prompt{In}{incolor}{6}{\boxspacing}
\begin{Verbatim}[commandchars=\\\{\}]
\PY{l+s+sd}{\PYZsq{}\PYZsq{}\PYZsq{}Histogramas\PYZsq{}\PYZsq{}\PYZsq{}}

\PY{c+c1}{\PYZsh{} Configuración del diseño del gráfico}

\PY{n}{plt}\PY{o}{.}\PY{n}{figure}\PY{p}{(}\PY{n}{figsize}\PY{o}{=}\PY{p}{(}\PY{l+m+mi}{12}\PY{p}{,} \PY{l+m+mi}{8}\PY{p}{)}\PY{p}{)}
\PY{n}{plt}\PY{o}{.}\PY{n}{subplots\PYZus{}adjust}\PY{p}{(}\PY{n}{hspace}\PY{o}{=}\PY{l+m+mf}{0.5}\PY{p}{)}

\PY{c+c1}{\PYZsh{} Configuración de los histogramas para cada variable del Dataset}

\PY{k}{for} \PY{n}{i}\PY{p}{,} \PY{n}{column} \PY{o+ow}{in} \PY{n+nb}{enumerate}\PY{p}{(}\PY{n}{df\PYZus{}concrete}\PY{o}{.}\PY{n}{columns}\PY{p}{)}\PY{p}{:}
    \PY{n}{ax} \PY{o}{=} \PY{n}{plt}\PY{o}{.}\PY{n}{subplot}\PY{p}{(}\PY{l+m+mi}{3}\PY{p}{,} \PY{l+m+mi}{3}\PY{p}{,} \PY{n}{i} \PY{o}{+} \PY{l+m+mi}{1}\PY{p}{)}
    \PY{n}{sns}\PY{o}{.}\PY{n}{histplot}\PY{p}{(}\PY{n}{df\PYZus{}concrete}\PY{p}{[}\PY{n}{column}\PY{p}{]}\PY{p}{,} \PY{n}{kde}\PY{o}{=}\PY{k+kc}{True}\PY{p}{,} \PY{n}{bins}\PY{o}{=}\PY{l+m+mi}{20}\PY{p}{,} \PY{n}{legend}\PY{o}{=}\PY{k+kc}{False}\PY{p}{,} \PY{n}{ax}\PY{o}{=}\PY{n}{ax}\PY{p}{)}
    \PY{n}{ax}\PY{o}{.}\PY{n}{set\PYZus{}xlabel}\PY{p}{(}\PY{l+s+s1}{\PYZsq{}}\PY{l+s+s1}{\PYZsq{}}\PY{p}{)} 
    \PY{n}{plt}\PY{o}{.}\PY{n}{title}\PY{p}{(}\PY{n}{column}\PY{p}{)}

\PY{n}{plt}\PY{o}{.}\PY{n}{tight\PYZus{}layout}\PY{p}{(}\PY{p}{)}
\PY{n}{plt}\PY{o}{.}\PY{n}{show}\PY{p}{(}\PY{p}{)}
\end{Verbatim}
\end{tcolorbox}

    \begin{center}
    \adjustimage{max size={0.9\linewidth}{0.9\paperheight}}{punto_5_files/punto_5_18_0.png}
    \end{center}
    { \hspace*{\fill} \\}
    
    \textbf{Análisis:}

Inicialmente, en términos gráficos se observa que el agua, el agregado
grueso y el fino, son los componentes que exhiben los mejores
comportamientos en términos de las distribuciones, lo cual puede generar
un menor impacto en el sesgo de las predicciones que se pretenden
realizar para la resistencia de la comprensión del concreto.
Adicionalmente, la cercanía de estas distribuciones a un comportamiento
normal, exhibe una menor variabilidad en los datos que serán utilizados
para el modelo de regresión.

De otro lado, el sesgo positivo que se presenta en la variable ``Edad''
puede generar posteriormente un problema de subestimación de la
resistencia a la compresión del concreto, debido a que la alta
concentración de los datos cercanos a 0 puede generar una menor
precisión en la predicción de dicha resistencia para el concreto que
evidencie una edad avanzada en términos de los días.

Ahora bien, teniendo en cuenta que cada registro expone los componentes
de la mezcla del concreto, se observa que la gran mayoría de las mezclas
en este conjunto de datos exhibe una cantidad mínima de este material.
Este resultado, puede representar problemas en la generalización del
modelo frente a situaciones o casos en los que este componente tenga una
concentración o presencia más importante en otros tipos de mezclas.

Finalmente, la variable objetivo gráficamente exhibe un comportamiento
cercano a una distribución normal con un leve sesgo hacia la derecha, lo
que sugiere que el uso de un modelo lineal podría ser adecuado para la
estimación de la comprensión del concreto.

    \begin{tcolorbox}[breakable, size=fbox, boxrule=1pt, pad at break*=1mm,colback=cellbackground, colframe=cellborder]
\prompt{In}{incolor}{7}{\boxspacing}
\begin{Verbatim}[commandchars=\\\{\}]
\PY{l+s+sd}{\PYZsq{}\PYZsq{}\PYZsq{}Boxplot\PYZsq{}\PYZsq{}\PYZsq{}}

\PY{c+c1}{\PYZsh{} Configuración del diseño del gráfico}
\PY{n}{plt}\PY{o}{.}\PY{n}{figure}\PY{p}{(}\PY{n}{figsize}\PY{o}{=}\PY{p}{(}\PY{l+m+mi}{12}\PY{p}{,} \PY{l+m+mi}{8}\PY{p}{)}\PY{p}{)}
\PY{n}{plt}\PY{o}{.}\PY{n}{subplots\PYZus{}adjust}\PY{p}{(}\PY{n}{hspace}\PY{o}{=}\PY{l+m+mf}{0.5}\PY{p}{)}

\PY{c+c1}{\PYZsh{} Configuración de los boxplots para cada variable del Dataset}
\PY{k}{for} \PY{n}{i}\PY{p}{,} \PY{n}{column} \PY{o+ow}{in} \PY{n+nb}{enumerate}\PY{p}{(}\PY{n}{df\PYZus{}concrete}\PY{o}{.}\PY{n}{columns}\PY{p}{)}\PY{p}{:}
    \PY{n}{ax} \PY{o}{=} \PY{n}{plt}\PY{o}{.}\PY{n}{subplot}\PY{p}{(}\PY{l+m+mi}{3}\PY{p}{,} \PY{l+m+mi}{3}\PY{p}{,} \PY{n}{i} \PY{o}{+} \PY{l+m+mi}{1}\PY{p}{)}
    \PY{n}{sns}\PY{o}{.}\PY{n}{boxplot}\PY{p}{(}\PY{n}{x}\PY{o}{=}\PY{n}{df\PYZus{}concrete}\PY{p}{[}\PY{n}{column}\PY{p}{]}\PY{p}{,} \PY{n}{orient}\PY{o}{=}\PY{l+s+s1}{\PYZsq{}}\PY{l+s+s1}{h}\PY{l+s+s1}{\PYZsq{}}\PY{p}{,} \PY{n}{color}\PY{o}{=}\PY{l+s+s1}{\PYZsq{}}\PY{l+s+s1}{orange}\PY{l+s+s1}{\PYZsq{}}\PY{p}{,} \PY{n}{ax}\PY{o}{=}\PY{n}{ax}\PY{p}{)}
    \PY{n}{ax}\PY{o}{.}\PY{n}{set\PYZus{}xlabel}\PY{p}{(}\PY{l+s+s1}{\PYZsq{}}\PY{l+s+s1}{\PYZsq{}}\PY{p}{)} 
    \PY{n}{plt}\PY{o}{.}\PY{n}{title}\PY{p}{(}\PY{n}{column}\PY{p}{)}

\PY{n}{plt}\PY{o}{.}\PY{n}{tight\PYZus{}layout}\PY{p}{(}\PY{p}{)}
\PY{n}{plt}\PY{o}{.}\PY{n}{show}\PY{p}{(}\PY{p}{)} 
\end{Verbatim}
\end{tcolorbox}

    \begin{center}
    \adjustimage{max size={0.9\linewidth}{0.9\paperheight}}{punto_5_files/punto_5_20_0.png}
    \end{center}
    { \hspace*{\fill} \\}
    
    \textbf{Análisis:}

A través del análisis gráfico con los Box Plots, se evidencia la
presencia de algunos valores atípicos en las variables Water,
Superplasticizer, Fine y Compressive Strenght. Sin embargo, se destaca
que en la variable Age la presencia y ubicación de los outliers es la
más alejada del tercer cuartil, es decir donde están concentrado el 75\%
de los datos.

Para una estimación posterior de un modelo de regresión lineal, este
comportamiento de datos atípicos podría generar una afectación en la
precisión de las predicciones que realice el modelo acerca de la
comprensión del concreto, también, puede generar algún tipo de sesgo en
la estimación de los coeficientes de cada una de las variables
explicativas.

    \begin{tcolorbox}[breakable, size=fbox, boxrule=1pt, pad at break*=1mm,colback=cellbackground, colframe=cellborder]
\prompt{In}{incolor}{8}{\boxspacing}
\begin{Verbatim}[commandchars=\\\{\}]
\PY{l+s+sd}{\PYZsq{}\PYZsq{}\PYZsq{}Gráficos de Dispersión\PYZsq{}\PYZsq{}\PYZsq{}}


\PY{c+c1}{\PYZsh{} Configuración del diseño del gráfico}
\PY{n}{plt}\PY{o}{.}\PY{n}{figure}\PY{p}{(}\PY{n}{figsize}\PY{o}{=}\PY{p}{(}\PY{l+m+mi}{13}\PY{p}{,} \PY{l+m+mi}{13}\PY{p}{)}\PY{p}{)}
\PY{n}{plt}\PY{o}{.}\PY{n}{subplots\PYZus{}adjust}\PY{p}{(}\PY{n}{hspace}\PY{o}{=}\PY{l+m+mf}{0.9}\PY{p}{)}

\PY{c+c1}{\PYZsh{} Configuración de los gráficos de dispersión para cada variable del Dataset}
\PY{k}{for} \PY{n}{i}\PY{p}{,} \PY{n}{column} \PY{o+ow}{in} \PY{n+nb}{enumerate}\PY{p}{(}\PY{n}{df\PYZus{}concrete}\PY{o}{.}\PY{n}{columns}\PY{p}{[}\PY{p}{:}\PY{o}{\PYZhy{}}\PY{l+m+mi}{1}\PY{p}{]}\PY{p}{)}\PY{p}{:}  \PY{c+c1}{\PYZsh{} Excluir la última columna (variable objetivo)}
    \PY{n}{ax} \PY{o}{=} \PY{n}{plt}\PY{o}{.}\PY{n}{subplot}\PY{p}{(}\PY{l+m+mi}{3}\PY{p}{,} \PY{l+m+mi}{3}\PY{p}{,} \PY{n}{i} \PY{o}{+} \PY{l+m+mi}{1}\PY{p}{)}
    \PY{n}{sns}\PY{o}{.}\PY{n}{scatterplot}\PY{p}{(}\PY{n}{x}\PY{o}{=}\PY{n}{df\PYZus{}concrete}\PY{p}{[}\PY{n}{column}\PY{p}{]}\PY{p}{,} \PY{n}{y}\PY{o}{=}\PY{n}{df\PYZus{}concrete}\PY{p}{[}\PY{l+s+s1}{\PYZsq{}}\PY{l+s+s1}{Compressive strength}\PY{l+s+s1}{\PYZsq{}}\PY{p}{]}\PY{p}{,} \PY{n}{color}\PY{o}{=}\PY{l+s+s1}{\PYZsq{}}\PY{l+s+s1}{navy}\PY{l+s+s1}{\PYZsq{}}\PY{p}{,} \PY{n}{ax}\PY{o}{=}\PY{n}{ax}\PY{p}{)}
    \PY{n}{plt}\PY{o}{.}\PY{n}{xlabel}\PY{p}{(}\PY{n}{column}\PY{p}{)}
    \PY{n}{plt}\PY{o}{.}\PY{n}{ylabel}\PY{p}{(}\PY{l+s+s1}{\PYZsq{}}\PY{l+s+s1}{Compressive Strength}\PY{l+s+s1}{\PYZsq{}}\PY{p}{)}
    \PY{n}{plt}\PY{o}{.}\PY{n}{title}\PY{p}{(}\PY{l+s+sa}{f}\PY{l+s+s1}{\PYZsq{}}\PY{l+s+si}{\PYZob{}}\PY{n}{column}\PY{l+s+si}{\PYZcb{}}\PY{l+s+s1}{ vs. Compressive Strength}\PY{l+s+s1}{\PYZsq{}}\PY{p}{)}

\PY{n}{plt}\PY{o}{.}\PY{n}{tight\PYZus{}layout}\PY{p}{(}\PY{p}{)}
\PY{n}{plt}\PY{o}{.}\PY{n}{show}\PY{p}{(}\PY{p}{)}
\end{Verbatim}
\end{tcolorbox}

    \begin{center}
    \adjustimage{max size={0.9\linewidth}{0.9\paperheight}}{punto_5_files/punto_5_22_0.png}
    \end{center}
    { \hspace*{\fill} \\}
    
    \textbf{Análisis:}

El análisis gráfico de la dispersión entre las variables explicativas
frente a la resistencia a la comprensión del concreto, permite observar
que la única variable que evidencia una forma elíptica es la relación
entre la variable objetivo y la variable ``Cement''. Este resultado
indica una posible relación lineal entre las variables mencionadas, lo
que permitiría aplicar la correlación de pearson para llevar a cabo una
evualuación de la fuerza y la dirección de dicha relación.

Aquí es importante destacar que, la correlación de Pearson es una medida
enfocada en capturar relaciones lineales entre variables, por tanto, la
relación entre el cemento y la resistencia a la comprensión del concreto
puede ser abordada a través de un modelo lineal.

No obstante, en las otras variables se observan agrupaciones
irregulares, que necesitarían utilizar modelos o aproximaciones
alternativas para identificar la relación entre las variables
explicativas y la variable objetivo (Kendall o Spearman), debido a que
la correlación de Pearson solo funciona cuando las distribuciones son
simétricas y es una medida sensible ante valores atípicos.

    \begin{tcolorbox}[breakable, size=fbox, boxrule=1pt, pad at break*=1mm,colback=cellbackground, colframe=cellborder]
\prompt{In}{incolor}{9}{\boxspacing}
\begin{Verbatim}[commandchars=\\\{\}]
\PY{l+s+sd}{\PYZsq{}\PYZsq{}\PYZsq{}Matriz de Correlación\PYZsq{}\PYZsq{}\PYZsq{}}

\PY{n}{correlation\PYZus{}matrix} \PY{o}{=} \PY{n}{df\PYZus{}concrete}\PY{o}{.}\PY{n}{corr}\PY{p}{(}\PY{p}{)}
\PY{n+nb}{print}\PY{p}{(}\PY{n}{correlation\PYZus{}matrix}\PY{p}{)}
\end{Verbatim}
\end{tcolorbox}

    \begin{Verbatim}[commandchars=\\\{\}]
                        Cement      Slag   Fly\_Ash     Water  \textbackslash{}
Cement                1.000000 -0.275193 -0.397475 -0.081544
Slag                 -0.275193  1.000000 -0.323569  0.107286
Fly\_Ash              -0.397475 -0.323569  1.000000 -0.257044
Water                -0.081544  0.107286 -0.257044  1.000000
Superplasticizer      0.092771  0.043376  0.377340 -0.657464
Coarse               -0.109356 -0.283998 -0.009977 -0.182312
Fine                 -0.222720 -0.281593  0.079076 -0.450635
Age                   0.081947 -0.044246 -0.154370  0.277604
Compressive strength  0.497833  0.134824 -0.105753 -0.289613

                      Superplasticizer    Coarse      Fine       Age  \textbackslash{}
Cement                        0.092771 -0.109356 -0.222720  0.081947
Slag                          0.043376 -0.283998 -0.281593 -0.044246
Fly\_Ash                       0.377340 -0.009977  0.079076 -0.154370
Water                        -0.657464 -0.182312 -0.450635  0.277604
Superplasticizer              1.000000 -0.266303  0.222501 -0.192717
Coarse                       -0.266303  1.000000 -0.178506 -0.003016
Fine                          0.222501 -0.178506  1.000000 -0.156094
Age                          -0.192717 -0.003016 -0.156094  1.000000
Compressive strength          0.366102 -0.164928 -0.167249  0.328877

                      Compressive strength
Cement                            0.497833
Slag                              0.134824
Fly\_Ash                          -0.105753
Water                            -0.289613
Superplasticizer                  0.366102
Coarse                           -0.164928
Fine                             -0.167249
Age                               0.328877
Compressive strength              1.000000
    \end{Verbatim}

    \begin{tcolorbox}[breakable, size=fbox, boxrule=1pt, pad at break*=1mm,colback=cellbackground, colframe=cellborder]
\prompt{In}{incolor}{10}{\boxspacing}
\begin{Verbatim}[commandchars=\\\{\}]
\PY{l+s+sd}{\PYZsq{}\PYZsq{}\PYZsq{}Mapa de calor\PYZsq{}\PYZsq{}\PYZsq{}}


\PY{n}{plt}\PY{o}{.}\PY{n}{figure}\PY{p}{(}\PY{n}{figsize}\PY{o}{=}\PY{p}{(}\PY{l+m+mi}{10}\PY{p}{,} \PY{l+m+mi}{8}\PY{p}{)}\PY{p}{)}
\PY{n}{sns}\PY{o}{.}\PY{n}{heatmap}\PY{p}{(}\PY{n}{correlation\PYZus{}matrix}\PY{p}{,} \PY{n}{annot}\PY{o}{=}\PY{k+kc}{True}\PY{p}{,} \PY{n}{cmap}\PY{o}{=}\PY{l+s+s1}{\PYZsq{}}\PY{l+s+s1}{coolwarm}\PY{l+s+s1}{\PYZsq{}}\PY{p}{,} \PY{n}{fmt}\PY{o}{=}\PY{l+s+s2}{\PYZdq{}}\PY{l+s+s2}{.2f}\PY{l+s+s2}{\PYZdq{}}\PY{p}{)}
\PY{n}{plt}\PY{o}{.}\PY{n}{title}\PY{p}{(}\PY{l+s+s1}{\PYZsq{}}\PY{l+s+s1}{Matriz de Correlación}\PY{l+s+s1}{\PYZsq{}}\PY{p}{)}
\PY{n}{plt}\PY{o}{.}\PY{n}{show}\PY{p}{(}\PY{p}{)}
\end{Verbatim}
\end{tcolorbox}

    \begin{center}
    \adjustimage{max size={0.9\linewidth}{0.9\paperheight}}{punto_5_files/punto_5_25_0.png}
    \end{center}
    { \hspace*{\fill} \\}
    
    \textbf{Análisis:}

A través del método ``Corr'' se identificó que la variable ``Cement'' es
aquella que presenta la relación más fuerte frente a la resistencia a la
compresión del concreto. Esto debido a que se evidencia una correlación
positiva de 0,50, que según Sarabia (2021), sugiere una correlación
positiva moderada entre la cantidad de cemento utilizada en la mezcla y
la resistencia del concreto.

Por otro lado, se observó que para las variables ``Superplasticizer''
(0,37) y ``Age'' (0,33) existe una correlación positiva débil frente a
la variable objetivo. Estos valores sugieren que a pesar de que dichas
variables tienen cierta influencia sobre la resistencia del concreto, el
efecto no es tan determinante como la cantidad de cemento utilizada.

    5.3. Entrenar un modelo de regresión lineal múltiple utilizando el
conjunto de datos y evalue si hay significancia en el modelo.

\textless/span

    \begin{tcolorbox}[breakable, size=fbox, boxrule=1pt, pad at break*=1mm,colback=cellbackground, colframe=cellborder]
\prompt{In}{incolor}{11}{\boxspacing}
\begin{Verbatim}[commandchars=\\\{\}]
\PY{k+kn}{from} \PY{n+nn}{sklearn}\PY{n+nn}{.}\PY{n+nn}{model\PYZus{}selection} \PY{k+kn}{import} \PY{n}{train\PYZus{}test\PYZus{}split}
\PY{k+kn}{from} \PY{n+nn}{sklearn}\PY{n+nn}{.}\PY{n+nn}{linear\PYZus{}model} \PY{k+kn}{import} \PY{n}{LinearRegression}
\PY{k+kn}{from} \PY{n+nn}{sklearn}\PY{n+nn}{.}\PY{n+nn}{metrics} \PY{k+kn}{import} \PY{n}{r2\PYZus{}score}
\PY{k+kn}{import} \PY{n+nn}{statsmodels}\PY{n+nn}{.}\PY{n+nn}{api} \PY{k}{as} \PY{n+nn}{sm}

\PY{c+c1}{\PYZsh{} División de los datos en variables predictoras (X) y variable objetivo (y)}

\PY{n}{X} \PY{o}{=} \PY{n}{df\PYZus{}concrete}\PY{o}{.}\PY{n}{drop}\PY{p}{(}\PY{n}{columns}\PY{o}{=}\PY{p}{[}\PY{l+s+s1}{\PYZsq{}}\PY{l+s+s1}{Compressive strength}\PY{l+s+s1}{\PYZsq{}}\PY{p}{]}\PY{p}{,} \PY{n}{axis}\PY{o}{=}\PY{l+m+mi}{1}\PY{p}{)}
\PY{n}{y} \PY{o}{=} \PY{n}{df\PYZus{}concrete}\PY{p}{[}\PY{l+s+s1}{\PYZsq{}}\PY{l+s+s1}{Compressive strength}\PY{l+s+s1}{\PYZsq{}}\PY{p}{]}

\PY{c+c1}{\PYZsh{} Se añade una columna de 1\PYZsq{}s para el término independiente}

\PY{n}{X} \PY{o}{=} \PY{n}{sm}\PY{o}{.}\PY{n}{add\PYZus{}constant}\PY{p}{(}\PY{n}{X}\PY{p}{)}

\PY{c+c1}{\PYZsh{} Entrenamiento del modelo de regresión lineal utilizando OLS (Mínimos Cuadrados Ordinarios)}

\PY{n}{model} \PY{o}{=} \PY{n}{sm}\PY{o}{.}\PY{n}{OLS}\PY{p}{(}\PY{n}{y}\PY{p}{,} \PY{n}{X}\PY{p}{)}
\PY{n}{results} \PY{o}{=} \PY{n}{model}\PY{o}{.}\PY{n}{fit}\PY{p}{(}\PY{p}{)}

\PY{c+c1}{\PYZsh{} Resumen completo del modelo}

\PY{n+nb}{print}\PY{p}{(}\PY{n}{results}\PY{o}{.}\PY{n}{summary}\PY{p}{(}\PY{p}{)}\PY{p}{)}
\end{Verbatim}
\end{tcolorbox}

    \begin{Verbatim}[commandchars=\\\{\}]
                             OLS Regression Results
================================================================================
Dep. Variable:     Compressive strength   R-squared:                       0.615
Model:                              OLS   Adj. R-squared:                  0.612
Method:                   Least Squares   F-statistic:                     204.3
Date:                  Sat, 13 Apr 2024   Prob (F-statistic):          6.76e-206
Time:                          17:06:00   Log-Likelihood:                -3869.0
No. Observations:                  1030   AIC:                             7756.
Df Residuals:                      1021   BIC:                             7800.
Df Model:                             8
Covariance Type:              nonrobust
================================================================================
====
                       coef    std err          t      P>|t|      [0.025
0.975]
--------------------------------------------------------------------------------
----
const              -23.1638     26.588     -0.871      0.384     -75.338
29.010
Cement               0.1198      0.008     14.110      0.000       0.103
0.136
Slag                 0.1038      0.010     10.245      0.000       0.084
0.124
Fly\_Ash              0.0879      0.013      6.988      0.000       0.063
0.113
Water               -0.1503      0.040     -3.741      0.000      -0.229
-0.071
Superplasticizer     0.2907      0.093      3.110      0.002       0.107
0.474
Coarse               0.0180      0.009      1.919      0.055      -0.000
0.036
Fine                 0.0202      0.011      1.883      0.060      -0.001
0.041
Age                  0.1142      0.005     21.046      0.000       0.104
0.125
==============================================================================
Omnibus:                        5.379   Durbin-Watson:                   1.281
Prob(Omnibus):                  0.068   Jarque-Bera (JB):                5.305
Skew:                          -0.174   Prob(JB):                       0.0705
Kurtosis:                       3.045   Cond. No.                     1.06e+05
==============================================================================

Notes:
[1] Standard Errors assume that the covariance matrix of the errors is correctly
specified.
[2] The condition number is large, 1.06e+05. This might indicate that there are
strong multicollinearity or other numerical problems.
    \end{Verbatim}

    \textbf{Análisis:}

Se llevó a cabo el cálculo de un modelo de regresión lineal múltiple a
través del Método de Mínimos Cuadrados Ordinarios (OLS), de este se
puede interpretar los siguiente:

*El coeficiente de determinación, indica que el 61\% de la variabilidad
en la resistencia a la comprensión del concreto es explicada por las
variables independientes que se incluyeron en el modelo.

*Ahora bien, la prueba de significancia del modelo evalúa si al menos
una de las variables predictoras tiene un efecto significativo sobre la
variable dependiente. En este caso, la prueba de hipótesis se plantea de
la siguiente manera:

\begin{itemize}
\tightlist
\item
  Ho: todos los coeficientes del modelo son iguales a 0
\item
  H1: al menos uno de los coeficientes del modelo es diferente de 0
\end{itemize}

Para este modelo específico el, el valor F es menor al nivel de
significancia al 5\%, por tanto, se rechaza la hipótesis nula y se puede
afirmar que al menos una de las variables explicativas del modelo tiene
un efecto significativo sobre la resistencia a la comprensión del
concreto (variable dependiente). En resumen, hay evidencia de
significancia en el modelo propuesto.

    5.4. Analizar la significancia estadística de las variables predictoras
y construir un modelo de regresión lineal múltiple reducido con las
variables significativas. Revise su desempeño con respecto al modelo
completo revisando el Adj R2 y los criterios de información de Akaike
y de Bayes (AIC y BIC).

\textless/span

    \textbf{Análisis de significancia estadística de variables
independientes:}

Teniendo en cuenta la salida proporcionada en el punto anterior, en la
que se plantea un modelo de regresión múltiple que explica la
resistencia a la comprensión del concreto explicada por el cemento, la
escoria, la ceniza volante, el agua, el superplastificante, el agregado
grueso, el agregado fino y la edad del concreto, tenemos lo siguiente:

La signifancia de los coeficientes que acompañan a las variables
predictoras se evalúa a través del valor ``p'' otorgado en el modelo,
para este contexto, la prueba de hipótesis se plantea de la siguiente
manera:

\begin{itemize}
\tightlist
\item
  Ho: el coeficiente es igual a cero
\item
  H1: el coeficiente es distinto de cero
\end{itemize}

Para este modelo específico, el valor p es menor a un nivel de
significancia correspondiente al 5\% para las variables cemento,
escoria, ceniza volante, agua, superplastificante, y la edad. Por tanto,
se rechaza la hipótesis nula y es posible asegurar que los coeficientes
que acompañan a estas variables predictoras son significativas.

Por su parte, el intercepto y el agregado grueso, evidencian un valor
crítico mayor al nivel de significancia correspondiente al 5\%. Por
tanto, no hay evidencia para rechazar la hipótesis nula, lo que explica
la no significancia de los coeficientes, pues ambos son iguales a cero.

Finalmente, aunque el valor p asociado al coeficiente de la variable
agregado fino es 0.06 (una cifra menor pero muy cercana al valor
crítico) no proporciona una evidencia concluyente para rechazar la
hipótesis nula. Por lo tanto, no se considerará como un coeficiente
significativamente distinto de cero.

    \textbf{Modelo de Regresión Lineal Reducido:}

Teniendo en cuenta los resultados observados en el análisis de la
signficancia de los coeficientes se plantea un modelo reducido teniendo
encuenta las siguientes variables:

\begin{itemize}
\tightlist
\item
  Variable dependiente: Compressive strength
\item
  Variables independientes o predictoras: Cement, Slag, Fly\_Ash, Water,
  Superplasticizer, Age
\end{itemize}

    \begin{tcolorbox}[breakable, size=fbox, boxrule=1pt, pad at break*=1mm,colback=cellbackground, colframe=cellborder]
\prompt{In}{incolor}{12}{\boxspacing}
\begin{Verbatim}[commandchars=\\\{\}]
\PY{c+c1}{\PYZsh{} Dividir los datos en variables predictoras (X) y variable objetivo (y)}
\PY{n}{X} \PY{o}{=} \PY{n}{df\PYZus{}concrete}\PY{o}{.}\PY{n}{drop}\PY{p}{(}\PY{n}{columns}\PY{o}{=}\PY{p}{[}\PY{l+s+s1}{\PYZsq{}}\PY{l+s+s1}{Compressive strength}\PY{l+s+s1}{\PYZsq{}}\PY{p}{,} \PY{l+s+s1}{\PYZsq{}}\PY{l+s+s1}{Coarse}\PY{l+s+s1}{\PYZsq{}}\PY{p}{,} \PY{l+s+s1}{\PYZsq{}}\PY{l+s+s1}{Fine}\PY{l+s+s1}{\PYZsq{}}\PY{p}{]}\PY{p}{,} \PY{n}{axis}\PY{o}{=}\PY{l+m+mi}{1}\PY{p}{)}
\PY{n}{y} \PY{o}{=} \PY{n}{df\PYZus{}concrete}\PY{p}{[}\PY{l+s+s1}{\PYZsq{}}\PY{l+s+s1}{Compressive strength}\PY{l+s+s1}{\PYZsq{}}\PY{p}{]}

\PY{c+c1}{\PYZsh{} Añadir una columna de unos para el término independiente}
\PY{n}{X} \PY{o}{=} \PY{n}{sm}\PY{o}{.}\PY{n}{add\PYZus{}constant}\PY{p}{(}\PY{n}{X}\PY{p}{)}

\PY{c+c1}{\PYZsh{} Entrenar el modelo de regresión lineal utilizando OLS (Mínimos Cuadrados Ordinarios)}
\PY{n}{model\PYZus{}reduced} \PY{o}{=} \PY{n}{sm}\PY{o}{.}\PY{n}{OLS}\PY{p}{(}\PY{n}{y}\PY{p}{,} \PY{n}{X}\PY{p}{)}
\PY{n}{results\PYZus{}reduced} \PY{o}{=} \PY{n}{model\PYZus{}reduced}\PY{o}{.}\PY{n}{fit}\PY{p}{(}\PY{p}{)}

\PY{c+c1}{\PYZsh{} Mostrar un resumen completo del nuevo modelo}
\PY{n+nb}{print}\PY{p}{(}\PY{n}{results\PYZus{}reduced}\PY{o}{.}\PY{n}{summary}\PY{p}{(}\PY{p}{)}\PY{p}{)}
\end{Verbatim}
\end{tcolorbox}

    \begin{Verbatim}[commandchars=\\\{\}]
                             OLS Regression Results
================================================================================
Dep. Variable:     Compressive strength   R-squared:                       0.614
Model:                              OLS   Adj. R-squared:                  0.612
Method:                   Least Squares   F-statistic:                     271.2
Date:                  Sat, 13 Apr 2024   Prob (F-statistic):          1.78e-207
Time:                          17:06:04   Log-Likelihood:                -3871.0
No. Observations:                  1030   AIC:                             7756.
Df Residuals:                      1023   BIC:                             7791.
Df Model:                             6
Covariance Type:              nonrobust
================================================================================
====
                       coef    std err          t      P>|t|      [0.025
0.975]
--------------------------------------------------------------------------------
----
const               29.0302      4.212      6.891      0.000      20.764
37.296
Cement               0.1054      0.004     24.821      0.000       0.097
0.114
Slag                 0.0865      0.005     17.386      0.000       0.077
0.096
Fly\_Ash              0.0687      0.008      8.881      0.000       0.054
0.084
Water               -0.2183      0.021    -10.332      0.000      -0.260
-0.177
Superplasticizer     0.2390      0.085      2.826      0.005       0.073
0.405
Age                  0.1135      0.005     20.987      0.000       0.103
0.124
==============================================================================
Omnibus:                        5.233   Durbin-Watson:                   1.286
Prob(Omnibus):                  0.073   Jarque-Bera (JB):                5.193
Skew:                          -0.174   Prob(JB):                       0.0745
Kurtosis:                       3.019   Cond. No.                     4.66e+03
==============================================================================

Notes:
[1] Standard Errors assume that the covariance matrix of the errors is correctly
specified.
[2] The condition number is large, 4.66e+03. This might indicate that there are
strong multicollinearity or other numerical problems.





    \end{Verbatim}

    \textbf{Comparación del desempeño de modelo}

\begin{longtable}[]{@{}ccc@{}}
\toprule\noalign{}
Criterio & RLM Completa & RLM Reducida \\
\midrule\noalign{}
\endhead
\bottomrule\noalign{}
\endlastfoot
R2 Ajustado & 0.612 & 0.612 \\
AIC & 7756 & 7756 \\
BIC & 7800 & 7791 \\
\end{longtable}

Los criterios de AIC (Akaike) y BIC (Información de Bayes) son
indicadores que funcionan para comparar el ajuste de modelos
estadísticos, según Amaya (2018), el AIC sigue el principio de
parsimonia que está enfocado en la simplicidad del modelo siempre y
cuando no se comprometa su capacidad explicativa.

En otras palabras, favorece aquellos modelos que presentan un buen
equilibrio entre ajuste y simplicidad teniendo en cuenta la menor
cantidad de parametros posible. Es decir, a la hora de seleccionar cuál
modelo representa la mejor bondad de ajuste, es importante seleccionar
aquel que evidencia el menor AIC.

Por otro lado, el criterio de información bayesiano (BIC) es una medida
que se utiliza para seleccionar el modelo más adecuado, sin embargo,
esta aproximación penaliza de forma más severa aquellos modelos que
tienen un mayor número de parametros. Para este criterio, buscamos
selecionar el modelo que minimiza a BIC (Amaya, 2018).

En conclusión, según los resultados de la tabla anterior, aunque el R2
ajustado es el mismo en ambos modelos, el modelo reducido demuestra un
mejor ajuste debido a su menor valor de BIC (7791 \textless{} 7800).

    5.5. Valide los supuestos del modelo εi iid N(0, 2) y en caso de no
cumplir alguno, proponga una solución. Evalúe la conveniencia de usar un
enfoque robusto en este caso.

\textless/span

    \paragraph{\texorpdfstring{\textbf{Validación de los supuestos del
modelo:}}{Validación de los supuestos del modelo:}}\label{validaciuxf3n-de-los-supuestos-del-modelo}

    \textbf{Prueba de Independencia:}

A través del Test de Durbin-Watson se busca probar el supuesto de
independencia, que plantea la siguiente prueba de hipótesis:

\begin{itemize}
\tightlist
\item
  Ho: no hay autocorrelación de los errores del modelo
\item
  H1: existe autocorrelación entre los errores del modelo
\end{itemize}

La regla general para esta prueba plantea que valores del estadístico
que se encuentren cercanos a 2 (1.5 - 2.5), plantean la ausencia de
autocorrelación. Por su parte, si este es cercano a 0 sugiere
autocorrelación positiva, de otro lado, si es cercano a 4 sugiere
autocorrelación negativa.

    \begin{tcolorbox}[breakable, size=fbox, boxrule=1pt, pad at break*=1mm,colback=cellbackground, colframe=cellborder]
\prompt{In}{incolor}{13}{\boxspacing}
\begin{Verbatim}[commandchars=\\\{\}]
\PY{c+c1}{\PYZsh{} Calcular el estadístico de Durbin\PYZhy{}Watson}
\PY{n}{durbin\PYZus{}watson\PYZus{}statistic} \PY{o}{=} \PY{n}{sm}\PY{o}{.}\PY{n}{stats}\PY{o}{.}\PY{n}{stattools}\PY{o}{.}\PY{n}{durbin\PYZus{}watson}\PY{p}{(}\PY{n}{results\PYZus{}reduced}\PY{o}{.}\PY{n}{resid}\PY{p}{)}

\PY{c+c1}{\PYZsh{} Mostrar el resultado}
\PY{n+nb}{print}\PY{p}{(}\PY{l+s+s2}{\PYZdq{}}\PY{l+s+s2}{Estadístico de Durbin\PYZhy{}Watson:}\PY{l+s+s2}{\PYZdq{}}\PY{p}{,} \PY{n}{durbin\PYZus{}watson\PYZus{}statistic}\PY{p}{)}
\end{Verbatim}
\end{tcolorbox}

    \begin{Verbatim}[commandchars=\\\{\}]
Estadístico de Durbin-Watson: 1.2859452611038242
    \end{Verbatim}

    \textbf{Análisis}

Teniendo en cuenta que el estadístico obtenido es menor a 2 (1.28), no
hay suficiente evidencia para rechazar Ho, por tanto, se puede afirmar
que el supuesto de independencia no se cumple para el modelo, debido a
que se identificó a través de la prueba la presencia de autocorrelación
positiva entre los residuos.

    \textbf{Prueba de Normalidad:}

A través del Test de Shapiro Wilks se busca probar el supuesto de
normalidad, que plantea la siguiente prueba de hipótesis:

\begin{itemize}
\tightlist
\item
  Ho: los residuos del modelo siguen una distribución normal
\item
  H1: los residuos del modelo no siguen una distribución normal
\end{itemize}

Para esta prueba específica, un p-value menor al nivel de significancia
del 5\% sugiere el rechazo de la hipótesis nula.

    \begin{tcolorbox}[breakable, size=fbox, boxrule=1pt, pad at break*=1mm,colback=cellbackground, colframe=cellborder]
\prompt{In}{incolor}{14}{\boxspacing}
\begin{Verbatim}[commandchars=\\\{\}]
\PY{k+kn}{from} \PY{n+nn}{scipy} \PY{k+kn}{import} \PY{n}{stats}

\PY{n}{shapiro\PYZus{}test\PYZus{}statistic}\PY{p}{,} \PY{n}{shapiro\PYZus{}p\PYZus{}value} \PY{o}{=} \PY{n}{stats}\PY{o}{.}\PY{n}{shapiro}\PY{p}{(}\PY{n}{results\PYZus{}reduced}\PY{o}{.}\PY{n}{resid}\PY{p}{)}
\PY{n+nb}{print}\PY{p}{(}\PY{l+s+s2}{\PYZdq{}}\PY{l+s+s2}{Estadístico de prueba para Shapiro Wilks: }\PY{l+s+s2}{\PYZdq{}}\PY{p}{,} \PY{n}{shapiro\PYZus{}test\PYZus{}statistic}\PY{p}{)}
\PY{n+nb}{print}\PY{p}{(}\PY{l+s+s2}{\PYZdq{}}\PY{l+s+s2}{P\PYZhy{}Value: }\PY{l+s+s2}{\PYZdq{}}\PY{p}{,} \PY{n}{shapiro\PYZus{}p\PYZus{}value}\PY{p}{)}
\end{Verbatim}
\end{tcolorbox}

    \begin{Verbatim}[commandchars=\\\{\}]
Estadístico de prueba para Shapiro Wilks:  0.9958420991897583
P-Value:  0.007003553677350283
    \end{Verbatim}

    \begin{tcolorbox}[breakable, size=fbox, boxrule=1pt, pad at break*=1mm,colback=cellbackground, colframe=cellborder]
\prompt{In}{incolor}{15}{\boxspacing}
\begin{Verbatim}[commandchars=\\\{\}]
\PY{k+kn}{import} \PY{n+nn}{matplotlib}\PY{n+nn}{.}\PY{n+nn}{pyplot} \PY{k}{as} \PY{n+nn}{plt}
\PY{k+kn}{import} \PY{n+nn}{statsmodels}\PY{n+nn}{.}\PY{n+nn}{api} \PY{k}{as} \PY{n+nn}{sm}
\PY{k+kn}{from} \PY{n+nn}{statsmodels}\PY{n+nn}{.}\PY{n+nn}{graphics}\PY{n+nn}{.}\PY{n+nn}{gofplots} \PY{k+kn}{import} \PY{n}{qqplot}

\PY{c+c1}{\PYZsh{}Residuos del modelo de regresión reducido}

\PY{n}{residuos} \PY{o}{=} \PY{n}{results\PYZus{}reduced}\PY{o}{.}\PY{n}{resid}

\PY{c+c1}{\PYZsh{} Gráfico QQ Plot }

\PY{n}{fig}\PY{p}{,} \PY{n}{ax} \PY{o}{=} \PY{n}{plt}\PY{o}{.}\PY{n}{subplots}\PY{p}{(}\PY{n}{figsize}\PY{o}{=}\PY{p}{(}\PY{l+m+mi}{8}\PY{p}{,} \PY{l+m+mi}{6}\PY{p}{)}\PY{p}{)}
\PY{n}{qqplot}\PY{p}{(}\PY{n}{residuos}\PY{p}{,} \PY{n}{line}\PY{o}{=}\PY{l+s+s1}{\PYZsq{}}\PY{l+s+s1}{s}\PY{l+s+s1}{\PYZsq{}}\PY{p}{,} \PY{n}{ax}\PY{o}{=}\PY{n}{ax}\PY{p}{)}
\PY{n}{ax}\PY{o}{.}\PY{n}{set\PYZus{}title}\PY{p}{(}\PY{l+s+s1}{\PYZsq{}}\PY{l+s+s1}{Gráfico de QQ Plot \PYZhy{} Residuos Modelo Reducido}\PY{l+s+s1}{\PYZsq{}}\PY{p}{)}
\PY{n}{plt}\PY{o}{.}\PY{n}{show}\PY{p}{(}\PY{p}{)}
\end{Verbatim}
\end{tcolorbox}

    \begin{center}
    \adjustimage{max size={0.9\linewidth}{0.9\paperheight}}{punto_5_files/punto_5_42_0.png}
    \end{center}
    { \hspace*{\fill} \\}
    
    \textbf{Análisis}

A partir de la prueba realizada correspondiente a Shapiro Wilks, se
rechaza la hipótesis nula que planteaba un comportamiento normal en la
distribución de los errores. Sin embargo, al analizar el gráfico Q-Q
Plot, se observa que la mayoría de los cuantiles teóricos y observados
para los términos de error están alineados con la línea de referencia.
Es decir, que es posible concluir que los términos de error del modelo
propuesto demuestran o sugieren un comportamiento normal.

    \textbf{Prueba de Media Cero:}

A través del One Sample T-Test se busca probar el supuesto de media cero
en los errores, que plantea la siguiente prueba de hipótesis:

\begin{itemize}
\tightlist
\item
  Ho: la media de los errores del modelo es cero
\item
  H1: la media de los errores del modelo es distinta de cero
\end{itemize}

Para esta prueba específica, un p-value menor al nivel de significancia
del 5\% sugiere el rechazo de la hipótesis nula.

    \begin{tcolorbox}[breakable, size=fbox, boxrule=1pt, pad at break*=1mm,colback=cellbackground, colframe=cellborder]
\prompt{In}{incolor}{16}{\boxspacing}
\begin{Verbatim}[commandchars=\\\{\}]
\PY{c+c1}{\PYZsh{}One Sample T\PYZhy{}Test}

\PY{k+kn}{from} \PY{n+nn}{scipy} \PY{k+kn}{import} \PY{n}{stats}


\PY{n}{residuos} \PY{o}{=} \PY{n}{results\PYZus{}reduced}\PY{o}{.}\PY{n}{resid}

\PY{c+c1}{\PYZsh{} Prueba de One Sample T\PYZhy{}Test}

\PY{n}{t\PYZus{}statistic}\PY{p}{,} \PY{n}{p\PYZus{}valor} \PY{o}{=} \PY{n}{stats}\PY{o}{.}\PY{n}{ttest\PYZus{}1samp}\PY{p}{(}\PY{n}{residuos}\PY{p}{,} \PY{l+m+mi}{0}\PY{p}{)}


\PY{n+nb}{print}\PY{p}{(}\PY{l+s+s2}{\PYZdq{}}\PY{l+s+s2}{Estadístico t:}\PY{l+s+s2}{\PYZdq{}}\PY{p}{,} \PY{n}{t\PYZus{}statistic}\PY{p}{)}
\PY{n+nb}{print}\PY{p}{(}\PY{l+s+s2}{\PYZdq{}}\PY{l+s+s2}{P\PYZhy{}Value:}\PY{l+s+s2}{\PYZdq{}}\PY{p}{,} \PY{n}{p\PYZus{}valor}\PY{p}{)}
\end{Verbatim}
\end{tcolorbox}

    \begin{Verbatim}[commandchars=\\\{\}]
Estadístico t: 5.426463010503265e-14
P-Value: 0.9999999999999567
    \end{Verbatim}

    \textbf{Análisis}

Teniendo en cuenta el resultado obtenido en el P-Value, no hay
suficiente evidencia para rechazar la hipótesis nula planteada. Por
tanto, el supuesto de media cero en los errores del modelo se cumple.

    \textbf{Prueba de Homocedasticidad}

A través de la prueba de Breusch Pagan se busca probar el supuesto
varianza constante en los errores, que plantea la siguiente prueba de
hipótesis:

\begin{itemize}
\tightlist
\item
  Ho: la varianza de los errores es constante (homocedasticidad)
\item
  H1: la varianza de los errores no es constante (heterocedasticidad)
\end{itemize}

Para esta prueba específica, un p-value menor al nivel de significancia
del 5\% sugiere el rechazo de la hipótesis nula.

    \begin{tcolorbox}[breakable, size=fbox, boxrule=1pt, pad at break*=1mm,colback=cellbackground, colframe=cellborder]
\prompt{In}{incolor}{17}{\boxspacing}
\begin{Verbatim}[commandchars=\\\{\}]
\PY{c+c1}{\PYZsh{}A través de este código se utiliza el método .exog que accede a la matriz de variables independientes del modelo}

\PY{n}{bp\PYZus{}test} \PY{o}{=} \PY{n}{sm}\PY{o}{.}\PY{n}{stats}\PY{o}{.}\PY{n}{diagnostic}\PY{o}{.}\PY{n}{het\PYZus{}breuschpagan}\PY{p}{(}\PY{n}{results\PYZus{}reduced}\PY{o}{.}\PY{n}{resid}\PY{p}{,} \PY{n}{results\PYZus{}reduced}\PY{o}{.}\PY{n}{model}\PY{o}{.}\PY{n}{exog}\PY{p}{)}
\PY{n+nb}{print}\PY{p}{(}\PY{l+s+s2}{\PYZdq{}}\PY{l+s+s2}{Estadístico de prueba de Breusch\PYZhy{}Pagan:}\PY{l+s+s2}{\PYZdq{}}\PY{p}{,} \PY{n}{bp\PYZus{}test}\PY{p}{[}\PY{l+m+mi}{0}\PY{p}{]}\PY{p}{)}
\PY{n+nb}{print}\PY{p}{(}\PY{l+s+s2}{\PYZdq{}}\PY{l+s+s2}{P\PYZhy{}Value:}\PY{l+s+s2}{\PYZdq{}}\PY{p}{,} \PY{n}{bp\PYZus{}test}\PY{p}{[}\PY{l+m+mi}{1}\PY{p}{]}\PY{p}{)}
\end{Verbatim}
\end{tcolorbox}

    \begin{Verbatim}[commandchars=\\\{\}]
Estadístico de prueba de Breusch-Pagan: 139.18162251188608
P-Value: 1.4915855798339575e-27
    \end{Verbatim}

    \begin{tcolorbox}[breakable, size=fbox, boxrule=1pt, pad at break*=1mm,colback=cellbackground, colframe=cellborder]
\prompt{In}{incolor}{18}{\boxspacing}
\begin{Verbatim}[commandchars=\\\{\}]
\PY{n}{plt}\PY{o}{.}\PY{n}{figure}\PY{p}{(}\PY{n}{figsize}\PY{o}{=}\PY{p}{(}\PY{l+m+mi}{15}\PY{p}{,}\PY{l+m+mi}{5}\PY{p}{)}\PY{p}{)}
\PY{n}{plt}\PY{o}{.}\PY{n}{plot}\PY{p}{(}\PY{n}{results\PYZus{}reduced}\PY{o}{.}\PY{n}{resid}\PY{p}{,} \PY{l+s+s1}{\PYZsq{}}\PY{l+s+s1}{.\PYZhy{}}\PY{l+s+s1}{\PYZsq{}}\PY{p}{,} \PY{n}{color} \PY{o}{=}\PY{l+s+s2}{\PYZdq{}}\PY{l+s+s2}{darkblue}\PY{l+s+s2}{\PYZdq{}}\PY{p}{,} \PY{n}{linewidth}\PY{o}{=}\PY{l+m+mf}{0.3}\PY{p}{)}
\PY{n}{plt}\PY{o}{.}\PY{n}{show}\PY{p}{(}\PY{p}{)}
\end{Verbatim}
\end{tcolorbox}

    \begin{center}
    \adjustimage{max size={0.9\linewidth}{0.9\paperheight}}{punto_5_files/punto_5_49_0.png}
    \end{center}
    { \hspace*{\fill} \\}
    
    \textbf{Análisis}

A partir del resultado de la prueba de homocedasticidad, se obtuvo un
valor lo suficientemente pequeño para rechazar la hipótesis nula, es
decir, hay presencia de heterocedasticidad en los errores.
Adicionalmente, en términos gráficos se puede concluir que el
comportamiento de la variabilidad de los errores no es constante.

    \textbf{Conclusiones generales de las pruebas}

A través de la validación realizada de los supuestos asociados a los
residuales del modelo reducido, se encontró que no se cumplen el supuesto de independenciay homocedasticidad:

\begin{itemize}
\item
  En el caso de la falta de independencia en los errores, esto implica
  revisar otros tipos de modelos que eventualmente puedan capturar de
  forma más efectiva las estructuras de dependencia que existen en los
  datos planteados para este ejercicio.
\item
  Finalmente, la ausencia de varianza constante en los errores también
  supone una necesidad de revisar modelos más robustos que puedan
  capturar los diferentes comportamientos que tienen los errores en
  términos de variabilidad con respecto a las variables predictoras.
\end{itemize}

    \paragraph{\texorpdfstring{\textbf{Enfoque
Robusto:}}{Enfoque Robusto:}}\label{enfoque-robusto}

    \begin{tcolorbox}[breakable, size=fbox, boxrule=1pt, pad at break*=1mm,colback=cellbackground, colframe=cellborder]
\prompt{In}{incolor}{19}{\boxspacing}
\begin{Verbatim}[commandchars=\\\{\}]
\PY{c+c1}{\PYZsh{}Paso 1. Estandarización de las variables}


\PY{k+kn}{from} \PY{n+nn}{sklearn}\PY{n+nn}{.}\PY{n+nn}{preprocessing} \PY{k+kn}{import} \PY{n}{StandardScaler}


\PY{c+c1}{\PYZsh{} Se seleccionan las variables predictoras del modelo}

\PY{n}{X} \PY{o}{=} \PY{n}{df\PYZus{}concrete}\PY{p}{[}\PY{p}{[}\PY{l+s+s1}{\PYZsq{}}\PY{l+s+s1}{Cement}\PY{l+s+s1}{\PYZsq{}}\PY{p}{,} \PY{l+s+s1}{\PYZsq{}}\PY{l+s+s1}{Slag}\PY{l+s+s1}{\PYZsq{}}\PY{p}{,} \PY{l+s+s1}{\PYZsq{}}\PY{l+s+s1}{Fly\PYZus{}Ash}\PY{l+s+s1}{\PYZsq{}}\PY{p}{,} \PY{l+s+s1}{\PYZsq{}}\PY{l+s+s1}{Water}\PY{l+s+s1}{\PYZsq{}}\PY{p}{,} \PY{l+s+s1}{\PYZsq{}}\PY{l+s+s1}{Superplasticizer}\PY{l+s+s1}{\PYZsq{}}\PY{p}{,} \PY{l+s+s1}{\PYZsq{}}\PY{l+s+s1}{Age}\PY{l+s+s1}{\PYZsq{}}\PY{p}{]}\PY{p}{]}

\PY{c+c1}{\PYZsh{} Inicializa el objeto StandardScaler}

\PY{n}{scaler} \PY{o}{=} \PY{n}{StandardScaler}\PY{p}{(}\PY{p}{)}

\PY{c+c1}{\PYZsh{} Se ajusta el escalador a los datos y se transforman}

\PY{n}{X\PYZus{}scaled} \PY{o}{=} \PY{n}{scaler}\PY{o}{.}\PY{n}{fit\PYZus{}transform}\PY{p}{(}\PY{n}{X}\PY{p}{)}

\PY{c+c1}{\PYZsh{} Creación de nuevos nombres para distinguirlos de los originales}

\PY{n}{new\PYZus{}column\PYZus{}names} \PY{o}{=} \PY{p}{[}\PY{n}{name} \PY{o}{+} \PY{l+s+s1}{\PYZsq{}}\PY{l+s+s1}{\PYZus{}E}\PY{l+s+s1}{\PYZsq{}} \PY{k}{for} \PY{n}{name} \PY{o+ow}{in} \PY{n}{X}\PY{o}{.}\PY{n}{columns}\PY{p}{]}

\PY{c+c1}{\PYZsh{} Creación de un nuevo data frame con las variables estandarizadas}

\PY{n}{X\PYZus{}scaled\PYZus{}df} \PY{o}{=} \PY{n}{pd}\PY{o}{.}\PY{n}{DataFrame}\PY{p}{(}\PY{n}{X\PYZus{}scaled}\PY{p}{,} \PY{n}{columns}\PY{o}{=}\PY{n}{new\PYZus{}column\PYZus{}names}\PY{p}{)}
\end{Verbatim}
\end{tcolorbox}

    \begin{tcolorbox}[breakable, size=fbox, boxrule=1pt, pad at break*=1mm,colback=cellbackground, colframe=cellborder]
\prompt{In}{incolor}{20}{\boxspacing}
\begin{Verbatim}[commandchars=\\\{\}]
\PY{c+c1}{\PYZsh{} Revisamos que se hayan renombrado correctamente las variables}

\PY{n}{X\PYZus{}scaled\PYZus{}df}\PY{o}{.}\PY{n}{head}\PY{p}{(}\PY{l+m+mi}{5}\PY{p}{)}
\end{Verbatim}
\end{tcolorbox}

            \begin{tcolorbox}[breakable, size=fbox, boxrule=.5pt, pad at break*=1mm, opacityfill=0]
\prompt{Out}{outcolor}{20}{\boxspacing}
\begin{Verbatim}[commandchars=\\\{\}]
   Cement\_E    Slag\_E  Fly\_Ash\_E   Water\_E  Superplasticizer\_E     Age\_E
0  2.477918 -0.856886  -0.847132 -0.916663           -0.620225 -0.279733
1  2.477918 -0.856886  -0.847132 -0.916663           -0.620225 -0.279733
2  0.491443  0.795533  -0.847132  2.175367           -1.038944  3.553066
3  0.491443  0.795533  -0.847132  2.175367           -1.038944  5.057677
4 -0.790432  0.678414  -0.847132  0.488805           -1.038944  4.978487
\end{Verbatim}
\end{tcolorbox}
        
    \begin{tcolorbox}[breakable, size=fbox, boxrule=1pt, pad at break*=1mm,colback=cellbackground, colframe=cellborder]
\prompt{In}{incolor}{21}{\boxspacing}
\begin{Verbatim}[commandchars=\\\{\}]
\PY{c+c1}{\PYZsh{} Añadir la variable predictora original al data frame estandarizado}

\PY{n}{X\PYZus{}scaled\PYZus{}df}\PY{p}{[}\PY{l+s+s1}{\PYZsq{}}\PY{l+s+s1}{Compressive strength}\PY{l+s+s1}{\PYZsq{}}\PY{p}{]} \PY{o}{=} \PY{n}{df\PYZus{}concrete}\PY{p}{[}\PY{l+s+s1}{\PYZsq{}}\PY{l+s+s1}{Compressive strength}\PY{l+s+s1}{\PYZsq{}}\PY{p}{]}
\end{Verbatim}
\end{tcolorbox}

    \begin{tcolorbox}[breakable, size=fbox, boxrule=1pt, pad at break*=1mm,colback=cellbackground, colframe=cellborder]
\prompt{In}{incolor}{22}{\boxspacing}
\begin{Verbatim}[commandchars=\\\{\}]
\PY{n}{X\PYZus{}scaled\PYZus{}df}\PY{o}{.}\PY{n}{head}\PY{p}{(}\PY{l+m+mi}{5}\PY{p}{)}
\end{Verbatim}
\end{tcolorbox}

            \begin{tcolorbox}[breakable, size=fbox, boxrule=.5pt, pad at break*=1mm, opacityfill=0]
\prompt{Out}{outcolor}{22}{\boxspacing}
\begin{Verbatim}[commandchars=\\\{\}]
   Cement\_E    Slag\_E  Fly\_Ash\_E   Water\_E  Superplasticizer\_E     Age\_E  \textbackslash{}
0  2.477918 -0.856886  -0.847132 -0.916663           -0.620225 -0.279733
1  2.477918 -0.856886  -0.847132 -0.916663           -0.620225 -0.279733
2  0.491443  0.795533  -0.847132  2.175367           -1.038944  3.553066
3  0.491443  0.795533  -0.847132  2.175367           -1.038944  5.057677
4 -0.790432  0.678414  -0.847132  0.488805           -1.038944  4.978487

   Compressive strength
0             79.986111
1             61.887366
2             40.269535
3             41.052780
4             44.296075
\end{Verbatim}
\end{tcolorbox}
        
    \begin{tcolorbox}[breakable, size=fbox, boxrule=1pt, pad at break*=1mm,colback=cellbackground, colframe=cellborder]
\prompt{In}{incolor}{23}{\boxspacing}
\begin{Verbatim}[commandchars=\\\{\}]
\PY{c+c1}{\PYZsh{}Reordenamos las variables}

\PY{n}{X\PYZus{}scaled\PYZus{}df} \PY{o}{=} \PY{n}{X\PYZus{}scaled\PYZus{}df}\PY{p}{[}\PY{p}{[}\PY{l+s+s1}{\PYZsq{}}\PY{l+s+s1}{Compressive strength}\PY{l+s+s1}{\PYZsq{}}\PY{p}{,} \PY{l+s+s1}{\PYZsq{}}\PY{l+s+s1}{Cement\PYZus{}E}\PY{l+s+s1}{\PYZsq{}}\PY{p}{,} \PY{l+s+s1}{\PYZsq{}}\PY{l+s+s1}{Slag\PYZus{}E}\PY{l+s+s1}{\PYZsq{}}\PY{p}{,} \PY{l+s+s1}{\PYZsq{}}\PY{l+s+s1}{Fly\PYZus{}Ash\PYZus{}E}\PY{l+s+s1}{\PYZsq{}}\PY{p}{,} \PY{l+s+s1}{\PYZsq{}}\PY{l+s+s1}{Water\PYZus{}E}\PY{l+s+s1}{\PYZsq{}}\PY{p}{,} \PY{l+s+s1}{\PYZsq{}}\PY{l+s+s1}{Superplasticizer\PYZus{}E}\PY{l+s+s1}{\PYZsq{}}\PY{p}{,}\PY{l+s+s1}{\PYZsq{}}\PY{l+s+s1}{Age\PYZus{}E}\PY{l+s+s1}{\PYZsq{}}\PY{p}{]}\PY{p}{]}
\PY{n}{X\PYZus{}scaled\PYZus{}df}\PY{o}{.}\PY{n}{head}\PY{p}{(}\PY{l+m+mi}{5}\PY{p}{)}
\end{Verbatim}
\end{tcolorbox}

            \begin{tcolorbox}[breakable, size=fbox, boxrule=.5pt, pad at break*=1mm, opacityfill=0]
\prompt{Out}{outcolor}{23}{\boxspacing}
\begin{Verbatim}[commandchars=\\\{\}]
   Compressive strength  Cement\_E    Slag\_E  Fly\_Ash\_E   Water\_E  \textbackslash{}
0             79.986111  2.477918 -0.856886  -0.847132 -0.916663
1             61.887366  2.477918 -0.856886  -0.847132 -0.916663
2             40.269535  0.491443  0.795533  -0.847132  2.175367
3             41.052780  0.491443  0.795533  -0.847132  2.175367
4             44.296075 -0.790432  0.678414  -0.847132  0.488805

   Superplasticizer\_E     Age\_E
0           -0.620225 -0.279733
1           -0.620225 -0.279733
2           -1.038944  3.553066
3           -1.038944  5.057677
4           -1.038944  4.978487
\end{Verbatim}
\end{tcolorbox}
        
    \begin{tcolorbox}[breakable, size=fbox, boxrule=1pt, pad at break*=1mm,colback=cellbackground, colframe=cellborder]
\prompt{In}{incolor}{24}{\boxspacing}
\begin{Verbatim}[commandchars=\\\{\}]
\PY{c+c1}{\PYZsh{}Calculamos las matrices de correlación a través del coeficiente de Spearman}

\PY{n}{corr} \PY{o}{=} \PY{n}{X\PYZus{}scaled\PYZus{}df}\PY{o}{.}\PY{n}{corr}\PY{p}{(}\PY{n}{method}\PY{o}{=} \PY{l+s+s2}{\PYZdq{}}\PY{l+s+s2}{spearman}\PY{l+s+s2}{\PYZdq{}}\PY{p}{)}
\PY{n}{corr}
\end{Verbatim}
\end{tcolorbox}

            \begin{tcolorbox}[breakable, size=fbox, boxrule=.5pt, pad at break*=1mm, opacityfill=0]
\prompt{Out}{outcolor}{24}{\boxspacing}
\begin{Verbatim}[commandchars=\\\{\}]
                      Compressive strength  Cement\_E    Slag\_E  Fly\_Ash\_E  \textbackslash{}
Compressive strength              1.000000  0.477601  0.162473  -0.077957
Cement\_E                          0.477601  1.000000 -0.250407  -0.418352
Slag\_E                            0.162473 -0.250407  1.000000  -0.247313
Fly\_Ash\_E                        -0.077957 -0.418352 -0.247313   1.000000
Water\_E                          -0.308371 -0.094417  0.049426  -0.283085
Superplasticizer\_E                0.347589  0.038397  0.093568   0.454713
Age\_E                             0.596020  0.004631 -0.017424   0.002802

                       Water\_E  Superplasticizer\_E     Age\_E
Compressive strength -0.308371            0.347589  0.596020
Cement\_E             -0.094417            0.038397  0.004631
Slag\_E                0.049426            0.093568 -0.017424
Fly\_Ash\_E            -0.283085            0.454713  0.002802
Water\_E               1.000000           -0.687060  0.090939
Superplasticizer\_E   -0.687060            1.000000 -0.009753
Age\_E                 0.090939           -0.009753  1.000000
\end{Verbatim}
\end{tcolorbox}
        
    \begin{tcolorbox}[breakable, size=fbox, boxrule=1pt, pad at break*=1mm,colback=cellbackground, colframe=cellborder]
\prompt{In}{incolor}{25}{\boxspacing}
\begin{Verbatim}[commandchars=\\\{\}]
\PY{c+c1}{\PYZsh{}Renombramos la matriz de covarianzas como Sigma para facilitar el entendimiento respecto a la formula definida para hallar los coeficientes}

\PY{n}{Sigma} \PY{o}{=} \PY{n}{corr}

\PY{n}{Sigma}
\end{Verbatim}
\end{tcolorbox}

            \begin{tcolorbox}[breakable, size=fbox, boxrule=.5pt, pad at break*=1mm, opacityfill=0]
\prompt{Out}{outcolor}{25}{\boxspacing}
\begin{Verbatim}[commandchars=\\\{\}]
                      Compressive strength  Cement\_E    Slag\_E  Fly\_Ash\_E  \textbackslash{}
Compressive strength              1.000000  0.477601  0.162473  -0.077957
Cement\_E                          0.477601  1.000000 -0.250407  -0.418352
Slag\_E                            0.162473 -0.250407  1.000000  -0.247313
Fly\_Ash\_E                        -0.077957 -0.418352 -0.247313   1.000000
Water\_E                          -0.308371 -0.094417  0.049426  -0.283085
Superplasticizer\_E                0.347589  0.038397  0.093568   0.454713
Age\_E                             0.596020  0.004631 -0.017424   0.002802

                       Water\_E  Superplasticizer\_E     Age\_E
Compressive strength -0.308371            0.347589  0.596020
Cement\_E             -0.094417            0.038397  0.004631
Slag\_E                0.049426            0.093568 -0.017424
Fly\_Ash\_E            -0.283085            0.454713  0.002802
Water\_E               1.000000           -0.687060  0.090939
Superplasticizer\_E   -0.687060            1.000000 -0.009753
Age\_E                 0.090939           -0.009753  1.000000
\end{Verbatim}
\end{tcolorbox}
        
    \begin{tcolorbox}[breakable, size=fbox, boxrule=1pt, pad at break*=1mm,colback=cellbackground, colframe=cellborder]
\prompt{In}{incolor}{26}{\boxspacing}
\begin{Verbatim}[commandchars=\\\{\}]
\PY{c+c1}{\PYZsh{}Para poder realizar el paso posterior, debemos pasar la matriz anterior a un arreglo tipo Numpy }

\PY{n}{Sigma} \PY{o}{=} \PY{n}{Sigma}\PY{o}{.}\PY{n}{to\PYZus{}numpy}\PY{p}{(}\PY{p}{)}
\end{Verbatim}
\end{tcolorbox}

    \begin{tcolorbox}[breakable, size=fbox, boxrule=1pt, pad at break*=1mm,colback=cellbackground, colframe=cellborder]
\prompt{In}{incolor}{27}{\boxspacing}
\begin{Verbatim}[commandchars=\\\{\}]
\PY{n}{Sigma}
\end{Verbatim}
\end{tcolorbox}

            \begin{tcolorbox}[breakable, size=fbox, boxrule=.5pt, pad at break*=1mm, opacityfill=0]
\prompt{Out}{outcolor}{27}{\boxspacing}
\begin{Verbatim}[commandchars=\\\{\}]
array([[ 1.        ,  0.47760116,  0.16247343, -0.07795675, -0.30837067,
         0.34758892,  0.59601954],
       [ 0.47760116,  1.        , -0.25040725, -0.41835167, -0.09441678,
         0.03839716,  0.00463067],
       [ 0.16247343, -0.25040725,  1.        , -0.24731281,  0.04942609,
         0.093568  , -0.01742443],
       [-0.07795675, -0.41835167, -0.24731281,  1.        , -0.28308461,
         0.45471292,  0.00280195],
       [-0.30837067, -0.09441678,  0.04942609, -0.28308461,  1.        ,
        -0.68705981,  0.09093886],
       [ 0.34758892,  0.03839716,  0.093568  ,  0.45471292, -0.68705981,
         1.        , -0.00975295],
       [ 0.59601954,  0.00463067, -0.01742443,  0.00280195,  0.09093886,
        -0.00975295,  1.        ]])
\end{Verbatim}
\end{tcolorbox}
        
    \begin{tcolorbox}[breakable, size=fbox, boxrule=1pt, pad at break*=1mm,colback=cellbackground, colframe=cellborder]
\prompt{In}{incolor}{28}{\boxspacing}
\begin{Verbatim}[commandchars=\\\{\}]
\PY{c+c1}{\PYZsh{}Matriz de covarianza para las variables de entrada}

\PY{n}{SigmaXX} \PY{o}{=} \PY{n}{Sigma}\PY{p}{[}\PY{l+m+mi}{1}\PY{p}{:}\PY{p}{,} \PY{l+m+mi}{1}\PY{p}{:}\PY{p}{]}

\PY{c+c1}{\PYZsh{}Vector de covarianzas entre las variables de entrada y la variable respuesta}

\PY{n}{SigmaXy} \PY{o}{=} \PY{n}{Sigma}\PY{p}{[}\PY{l+m+mi}{1}\PY{p}{:}\PY{p}{,} \PY{l+m+mi}{0}\PY{p}{]}

\PY{c+c1}{\PYZsh{}Estimación betas de las variables de entrada, que sale de realizar la multiplicación de sigma XX (invertida) por sigma Xy para hallar los betas}

\PY{n}{betas}\PY{o}{=} \PY{n}{np}\PY{o}{.}\PY{n}{matmul}\PY{p}{(}\PY{n}{np}\PY{o}{.}\PY{n}{linalg}\PY{o}{.}\PY{n}{inv}\PY{p}{(}\PY{n}{SigmaXX}\PY{p}{)}\PY{p}{,} \PY{n}{SigmaXy}\PY{p}{)}

\PY{n}{betas}
\end{Verbatim}
\end{tcolorbox}

            \begin{tcolorbox}[breakable, size=fbox, boxrule=.5pt, pad at break*=1mm, opacityfill=0]
\prompt{Out}{outcolor}{28}{\boxspacing}
\begin{Verbatim}[commandchars=\\\{\}]
array([ 0.62688197,  0.38601384,  0.19128868, -0.24395635,  0.03887093,
        0.62187096])
\end{Verbatim}
\end{tcolorbox}
        
    Ahora bien, con el objetivo de interpretar los coeficientes en las
unidades originales del ejercicio, se procede a desestandarizarlos. Este
proceso se lleva a cabo múltiplicando el coeficiente estandarizado por
el ratio entre la desviación estandar de la variable respuesta original
y la desviación estandar de la variable explicativa original según
corresponda.

    \begin{tcolorbox}[breakable, size=fbox, boxrule=1pt, pad at break*=1mm,colback=cellbackground, colframe=cellborder]
\prompt{In}{incolor}{29}{\boxspacing}
\begin{Verbatim}[commandchars=\\\{\}]
\PY{c+c1}{\PYZsh{} Desviaciones estándar originales de las variables predictoras y la variable de dependiente}

\PY{n}{desviacion\PYZus{}y} \PY{o}{=} \PY{n}{np}\PY{o}{.}\PY{n}{std}\PY{p}{(}\PY{n}{df\PYZus{}concrete}\PY{p}{[}\PY{l+s+s1}{\PYZsq{}}\PY{l+s+s1}{Compressive strength}\PY{l+s+s1}{\PYZsq{}}\PY{p}{]}\PY{p}{)}  
\PY{n}{desviaciones\PYZus{}x} \PY{o}{=} \PY{n}{np}\PY{o}{.}\PY{n}{std}\PY{p}{(}\PY{n}{df\PYZus{}concrete}\PY{p}{[}\PY{p}{[}\PY{l+s+s1}{\PYZsq{}}\PY{l+s+s1}{Cement}\PY{l+s+s1}{\PYZsq{}}\PY{p}{,} \PY{l+s+s1}{\PYZsq{}}\PY{l+s+s1}{Slag}\PY{l+s+s1}{\PYZsq{}}\PY{p}{,} \PY{l+s+s1}{\PYZsq{}}\PY{l+s+s1}{Fly\PYZus{}Ash}\PY{l+s+s1}{\PYZsq{}}\PY{p}{,} \PY{l+s+s1}{\PYZsq{}}\PY{l+s+s1}{Water}\PY{l+s+s1}{\PYZsq{}}\PY{p}{,} \PY{l+s+s1}{\PYZsq{}}\PY{l+s+s1}{Superplasticizer}\PY{l+s+s1}{\PYZsq{}}\PY{p}{,} \PY{l+s+s1}{\PYZsq{}}\PY{l+s+s1}{Age}\PY{l+s+s1}{\PYZsq{}}\PY{p}{]}\PY{p}{]}\PY{p}{)}  

\PY{c+c1}{\PYZsh{} Desestandarización de los coeficientes}

\PY{n}{coeficientes\PYZus{}desestandarizados} \PY{o}{=} \PY{n}{betas} \PY{o}{*} \PY{p}{(}\PY{n}{desviacion\PYZus{}y} \PY{o}{/} \PY{n}{desviaciones\PYZus{}x}\PY{p}{)}

\PY{c+c1}{\PYZsh{} Imprimir coeficientes desestandarizados}
\PY{n+nb}{print}\PY{p}{(}\PY{l+s+s2}{\PYZdq{}}\PY{l+s+s2}{Coeficientes desestandarizados:}\PY{l+s+s2}{\PYZdq{}}\PY{p}{)}
\PY{n+nb}{print}\PY{p}{(}\PY{n}{coeficientes\PYZus{}desestandarizados}\PY{p}{)}
\end{Verbatim}
\end{tcolorbox}

    \begin{Verbatim}[commandchars=\\\{\}]
Coeficientes desestandarizados:
Cement              0.100208
Slag                0.074741
Fly\_Ash             0.049934
Water              -0.190838
Superplasticizer    0.108708
Age                 0.164458
dtype: float64
    \end{Verbatim}

    Para el cálculo del intercepto, utilizamos la formula que indica la
resta entre la mediana de la variable dependiente y el producto entre la
mediana de las variables independientes y los coeficientes estimados en
el punto anterior. En este caso, usamos la mediana debido a que la misma
es menos sensible a los valores atípicos que existan en el conjunto de
datos.

    \begin{tcolorbox}[breakable, size=fbox, boxrule=1pt, pad at break*=1mm,colback=cellbackground, colframe=cellborder]
\prompt{In}{incolor}{39}{\boxspacing}
\begin{Verbatim}[commandchars=\\\{\}]
\PY{c+c1}{\PYZsh{} Calcular las medianas de las variables predictoras estandarizadas}

\PY{n}{x\PYZus{}bar} \PY{o}{=} \PY{n}{df\PYZus{}concrete}\PY{p}{[}\PY{p}{[}\PY{l+s+s2}{\PYZdq{}}\PY{l+s+s2}{Cement}\PY{l+s+s2}{\PYZdq{}}\PY{p}{,} \PY{l+s+s2}{\PYZdq{}}\PY{l+s+s2}{Slag}\PY{l+s+s2}{\PYZdq{}}\PY{p}{,} \PY{l+s+s2}{\PYZdq{}}\PY{l+s+s2}{Fly\PYZus{}Ash}\PY{l+s+s2}{\PYZdq{}}\PY{p}{,} \PY{l+s+s2}{\PYZdq{}}\PY{l+s+s2}{Water}\PY{l+s+s2}{\PYZdq{}}\PY{p}{,} \PY{l+s+s2}{\PYZdq{}}\PY{l+s+s2}{Superplasticizer}\PY{l+s+s2}{\PYZdq{}}\PY{p}{,} \PY{l+s+s2}{\PYZdq{}}\PY{l+s+s2}{Age}\PY{l+s+s2}{\PYZdq{}}\PY{p}{]}\PY{p}{]}\PY{o}{.}\PY{n}{median}\PY{p}{(}\PY{p}{)}

\PY{c+c1}{\PYZsh{} Calcular el intercepto}

\PY{n}{intercepto} \PY{o}{=} \PY{n}{df\PYZus{}concrete}\PY{p}{[}\PY{l+s+s2}{\PYZdq{}}\PY{l+s+s2}{Compressive strength}\PY{l+s+s2}{\PYZdq{}}\PY{p}{]}\PY{o}{.}\PY{n}{median}\PY{p}{(}\PY{p}{)} \PY{o}{\PYZhy{}} \PY{n}{np}\PY{o}{.}\PY{n}{dot}\PY{p}{(}\PY{n}{x\PYZus{}bar}\PY{o}{.}\PY{n}{to\PYZus{}numpy}\PY{p}{(}\PY{p}{)}\PY{p}{,} \PY{n}{betas}\PY{p}{)}

\PY{c+c1}{\PYZsh{} Imprimir el intercepto}

\PY{n+nb}{print}\PY{p}{(}\PY{l+s+s2}{\PYZdq{}}\PY{l+s+s2}{Intercepto:}\PY{l+s+s2}{\PYZdq{}}\PY{p}{,} \PY{n}{intercepto}\PY{p}{)}
\end{Verbatim}
\end{tcolorbox}

    \begin{Verbatim}[commandchars=\\\{\}]
Intercepto: -117.65291406345
    \end{Verbatim}

    \textbf{Análisis}

    \begin{longtable}[]{@{}ccc@{}}
\toprule\noalign{}
Modelo & RLM OLS & RLM Robusta \\
\midrule\noalign{}
\endhead
\bottomrule\noalign{}
\endlastfoot
Intercepto & 0.612 & -117.652 \\
Cement & 0.1054 & 0.100208 \\
Slag & 0.0865 & 0.074741 \\
Fly\_Ash & 0.0687 & 0.049934 \\
Water & -0.2183 & -0.190838 \\
Superplasticizer & 0.2390 & 0.108708 \\
Age & 0.1135 & 0.164458 \\
\end{longtable}

La conveniencia de utilizar un enfoque robusto en este ejercicio, radica
en la necesidad de evitar los efectos que pueden generar los valores
atípicos de los datos y corregir los posibles sesgos que pueden ser
generados por el incumplimineto de los supuestos que surgieron en la
estimación por mínimos cuadrados ordinarios.

Sin embargo, es importante mencionar que la estabilización de los
coeficientes a través de métodos robustos no asegura la correción de los
problemas asociados al incumplimiento de los supuestos del modelo por
MCO.

    \textbf{Bibliografía}

Amaya Jiménez, L. (2018). Criterio de Akaike para la selección de
modelos con transformaciones (Doctoral dissertation, Universidad Santo
Tomás).

Granados, R. M. (2016). Modelos de regresión lineal múltiple. Granada,
España: Departamento de Economía Aplicada, Universidad de Granada.

Sarabia Collazo, A. A. Indice de inmunidad-inflamación sistémica como
predictor independiente de morbimortalidad en neumonía por COVID-19.


    % Add a bibliography block to the postdoc
    
    
    
\end{document}
