\documentclass[11pt]{article}

    \usepackage[breakable]{tcolorbox}
    \usepackage[utf8]{inputenc}  % Add this line
    \usepackage{parskip} % Stop auto-indenting (to mimic markdown behaviour)
    

    % Basic figure setup, for now with no caption control since it's done
    % automatically by Pandoc (which extracts ![](path) syntax from Markdown).
    \usepackage{graphicx}
    % Maintain compatibility with old templates. Remove in nbconvert 6.0
    \let\Oldincludegraphics\includegraphics
    % Ensure that by default, figures have no caption (until we provide a
    % proper Figure object with a Caption API and a way to capture that
    % in the conversion process - todo).
    \usepackage{caption}
    \DeclareCaptionFormat{nocaption}{}
    \captionsetup{format=nocaption,aboveskip=0pt,belowskip=0pt}

    \usepackage{float}
    \floatplacement{figure}{H} % forces figures to be placed at the correct location
    \usepackage{xcolor} % Allow colors to be defined
    \usepackage{enumerate} % Needed for markdown enumerations to work
    \usepackage{geometry} % Used to adjust the document margins
    \usepackage{amsmath} % Equations
    \usepackage{amssymb} % Equations
    \usepackage{textcomp} % defines textquotesingle
    % Hack from http://tex.stackexchange.com/a/47451/13684:
    \AtBeginDocument{%
        \def\PYZsq{\textquotesingle}% Upright quotes in Pygmentized code
    }
    \usepackage{upquote} % Upright quotes for verbatim code
    \usepackage{eurosym} % defines \euro

    \usepackage{iftex}
    \ifPDFTeX
        \usepackage[T1]{fontenc}
        \IfFileExists{alphabeta.sty}{
              \usepackage{alphabeta}
          }{
              \usepackage[mathletters]{ucs}
              \usepackage[utf8x]{inputenc}
          }
    \else
        \usepackage{fontspec}
        \usepackage{unicode-math}
    \fi

    \usepackage{fancyvrb} % verbatim replacement that allows latex
    \usepackage{grffile} % extends the file name processing of package graphics
                         % to support a larger range
    \makeatletter % fix for old versions of grffile with XeLaTeX
    \@ifpackagelater{grffile}{2019/11/01}
    {
      % Do nothing on new versions
    }
    {
      \def\Gread@@xetex#1{%
        \IfFileExists{"\Gin@base".bb}%
        {\Gread@eps{\Gin@base.bb}}%
        {\Gread@@xetex@aux#1}%
      }
    }
    \makeatother
    \usepackage[Export]{adjustbox} % Used to constrain images to a maximum size
    \adjustboxset{max size={0.9\linewidth}{0.9\paperheight}}

    % The hyperref package gives us a pdf with properly built
    % internal navigation ('pdf bookmarks' for the table of contents,
    % internal cross-reference links, web links for URLs, etc.)
    \usepackage{hyperref}
    % The default LaTeX title has an obnoxious amount of whitespace. By default,
    % titling removes some of it. It also provides customization options.
    \usepackage{titling}
    \usepackage{longtable} % longtable support required by pandoc >1.10
    \usepackage{booktabs}  % table support for pandoc > 1.12.2
    \usepackage{array}     % table support for pandoc >= 2.11.3
    \usepackage{calc}      % table minipage width calculation for pandoc >= 2.11.1
    \usepackage[inline]{enumitem} % IRkernel/repr support (it uses the enumerate* environment)
    \usepackage[normalem]{ulem} % ulem is needed to support strikethroughs (\sout)
                                % normalem makes italics be italics, not underlines
    \usepackage{soul}      % strikethrough (\st) support for pandoc >= 3.0.0
    \usepackage{mathrsfs}
    

    
    % Colors for the hyperref package
    \definecolor{urlcolor}{rgb}{0,.145,.698}
    \definecolor{linkcolor}{rgb}{.71,0.21,0.01}
    \definecolor{citecolor}{rgb}{.12,.54,.11}

    % ANSI colors
    \definecolor{ansi-black}{HTML}{3E424D}
    \definecolor{ansi-black-intense}{HTML}{282C36}
    \definecolor{ansi-red}{HTML}{E75C58}
    \definecolor{ansi-red-intense}{HTML}{B22B31}
    \definecolor{ansi-green}{HTML}{00A250}
    \definecolor{ansi-green-intense}{HTML}{007427}
    \definecolor{ansi-yellow}{HTML}{DDB62B}
    \definecolor{ansi-yellow-intense}{HTML}{B27D12}
    \definecolor{ansi-blue}{HTML}{208FFB}
    \definecolor{ansi-blue-intense}{HTML}{0065CA}
    \definecolor{ansi-magenta}{HTML}{D160C4}
    \definecolor{ansi-magenta-intense}{HTML}{A03196}
    \definecolor{ansi-cyan}{HTML}{60C6C8}
    \definecolor{ansi-cyan-intense}{HTML}{258F8F}
    \definecolor{ansi-white}{HTML}{C5C1B4}
    \definecolor{ansi-white-intense}{HTML}{A1A6B2}
    \definecolor{ansi-default-inverse-fg}{HTML}{FFFFFF}
    \definecolor{ansi-default-inverse-bg}{HTML}{000000}

    % common color for the border for error outputs.
    \definecolor{outerrorbackground}{HTML}{FFDFDF}

    % commands and environments needed by pandoc snippets
    % extracted from the output of `pandoc -s`
    \providecommand{\tightlist}{%
      \setlength{\itemsep}{0pt}\setlength{\parskip}{0pt}}
    \DefineVerbatimEnvironment{Highlighting}{Verbatim}{commandchars=\\\{\}}
    % Add ',fontsize=\small' for more characters per line
    \newenvironment{Shaded}{}{}
    \newcommand{\KeywordTok}[1]{\textcolor[rgb]{0.00,0.44,0.13}{\textbf{{#1}}}}
    \newcommand{\DataTypeTok}[1]{\textcolor[rgb]{0.56,0.13,0.00}{{#1}}}
    \newcommand{\DecValTok}[1]{\textcolor[rgb]{0.25,0.63,0.44}{{#1}}}
    \newcommand{\BaseNTok}[1]{\textcolor[rgb]{0.25,0.63,0.44}{{#1}}}
    \newcommand{\FloatTok}[1]{\textcolor[rgb]{0.25,0.63,0.44}{{#1}}}
    \newcommand{\CharTok}[1]{\textcolor[rgb]{0.25,0.44,0.63}{{#1}}}
    \newcommand{\StringTok}[1]{\textcolor[rgb]{0.25,0.44,0.63}{{#1}}}
    \newcommand{\CommentTok}[1]{\textcolor[rgb]{0.38,0.63,0.69}{\textit{{#1}}}}
    \newcommand{\OtherTok}[1]{\textcolor[rgb]{0.00,0.44,0.13}{{#1}}}
    \newcommand{\AlertTok}[1]{\textcolor[rgb]{1.00,0.00,0.00}{\textbf{{#1}}}}
    \newcommand{\FunctionTok}[1]{\textcolor[rgb]{0.02,0.16,0.49}{{#1}}}
    \newcommand{\RegionMarkerTok}[1]{{#1}}
    \newcommand{\ErrorTok}[1]{\textcolor[rgb]{1.00,0.00,0.00}{\textbf{{#1}}}}
    \newcommand{\NormalTok}[1]{{#1}}

    % Additional commands for more recent versions of Pandoc
    \newcommand{\ConstantTok}[1]{\textcolor[rgb]{0.53,0.00,0.00}{{#1}}}
    \newcommand{\SpecialCharTok}[1]{\textcolor[rgb]{0.25,0.44,0.63}{{#1}}}
    \newcommand{\VerbatimStringTok}[1]{\textcolor[rgb]{0.25,0.44,0.63}{{#1}}}
    \newcommand{\SpecialStringTok}[1]{\textcolor[rgb]{0.73,0.40,0.53}{{#1}}}
    \newcommand{\ImportTok}[1]{{#1}}
    \newcommand{\DocumentationTok}[1]{\textcolor[rgb]{0.73,0.13,0.13}{\textit{{#1}}}}
    \newcommand{\AnnotationTok}[1]{\textcolor[rgb]{0.38,0.63,0.69}{\textbf{\textit{{#1}}}}}
    \newcommand{\CommentVarTok}[1]{\textcolor[rgb]{0.38,0.63,0.69}{\textbf{\textit{{#1}}}}}
    \newcommand{\VariableTok}[1]{\textcolor[rgb]{0.10,0.09,0.49}{{#1}}}
    \newcommand{\ControlFlowTok}[1]{\textcolor[rgb]{0.00,0.44,0.13}{\textbf{{#1}}}}
    \newcommand{\OperatorTok}[1]{\textcolor[rgb]{0.40,0.40,0.40}{{#1}}}
    \newcommand{\BuiltInTok}[1]{{#1}}
    \newcommand{\ExtensionTok}[1]{{#1}}
    \newcommand{\PreprocessorTok}[1]{\textcolor[rgb]{0.74,0.48,0.00}{{#1}}}
    \newcommand{\AttributeTok}[1]{\textcolor[rgb]{0.49,0.56,0.16}{{#1}}}
    \newcommand{\InformationTok}[1]{\textcolor[rgb]{0.38,0.63,0.69}{\textbf{\textit{{#1}}}}}
    \newcommand{\WarningTok}[1]{\textcolor[rgb]{0.38,0.63,0.69}{\textbf{\textit{{#1}}}}}


    % Define a nice break command that doesn't care if a line doesn't already
    % exist.
    \def\br{\hspace*{\fill} \\* }
    % Math Jax compatibility definitions
    \def\gt{>}
    \def\lt{<}
    \let\Oldtex\TeX
    \let\Oldlatex\LaTeX
    \renewcommand{\TeX}{\textrm{\Oldtex}}
    \renewcommand{\LaTeX}{\textrm{\Oldlatex}}
    % Document parameters
    % Document title
    \title{}
    \author{}
    \date{}
    
    
    
    
    
    
    
% Pygments definitions
\makeatletter
\def\PY@reset{\let\PY@it=\relax \let\PY@bf=\relax%
    \let\PY@ul=\relax \let\PY@tc=\relax%
    \let\PY@bc=\relax \let\PY@ff=\relax}
\def\PY@tok#1{\csname PY@tok@#1\endcsname}
\def\PY@toks#1+{\ifx\relax#1\empty\else%
    \PY@tok{#1}\expandafter\PY@toks\fi}
\def\PY@do#1{\PY@bc{\PY@tc{\PY@ul{%
    \PY@it{\PY@bf{\PY@ff{#1}}}}}}}
\def\PY#1#2{\PY@reset\PY@toks#1+\relax+\PY@do{#2}}

\@namedef{PY@tok@w}{\def\PY@tc##1{\textcolor[rgb]{0.73,0.73,0.73}{##1}}}
\@namedef{PY@tok@c}{\let\PY@it=\textit\def\PY@tc##1{\textcolor[rgb]{0.24,0.48,0.48}{##1}}}
\@namedef{PY@tok@cp}{\def\PY@tc##1{\textcolor[rgb]{0.61,0.40,0.00}{##1}}}
\@namedef{PY@tok@k}{\let\PY@bf=\textbf\def\PY@tc##1{\textcolor[rgb]{0.00,0.50,0.00}{##1}}}
\@namedef{PY@tok@kp}{\def\PY@tc##1{\textcolor[rgb]{0.00,0.50,0.00}{##1}}}
\@namedef{PY@tok@kt}{\def\PY@tc##1{\textcolor[rgb]{0.69,0.00,0.25}{##1}}}
\@namedef{PY@tok@o}{\def\PY@tc##1{\textcolor[rgb]{0.40,0.40,0.40}{##1}}}
\@namedef{PY@tok@ow}{\let\PY@bf=\textbf\def\PY@tc##1{\textcolor[rgb]{0.67,0.13,1.00}{##1}}}
\@namedef{PY@tok@nb}{\def\PY@tc##1{\textcolor[rgb]{0.00,0.50,0.00}{##1}}}
\@namedef{PY@tok@nf}{\def\PY@tc##1{\textcolor[rgb]{0.00,0.00,1.00}{##1}}}
\@namedef{PY@tok@nc}{\let\PY@bf=\textbf\def\PY@tc##1{\textcolor[rgb]{0.00,0.00,1.00}{##1}}}
\@namedef{PY@tok@nn}{\let\PY@bf=\textbf\def\PY@tc##1{\textcolor[rgb]{0.00,0.00,1.00}{##1}}}
\@namedef{PY@tok@ne}{\let\PY@bf=\textbf\def\PY@tc##1{\textcolor[rgb]{0.80,0.25,0.22}{##1}}}
\@namedef{PY@tok@nv}{\def\PY@tc##1{\textcolor[rgb]{0.10,0.09,0.49}{##1}}}
\@namedef{PY@tok@no}{\def\PY@tc##1{\textcolor[rgb]{0.53,0.00,0.00}{##1}}}
\@namedef{PY@tok@nl}{\def\PY@tc##1{\textcolor[rgb]{0.46,0.46,0.00}{##1}}}
\@namedef{PY@tok@ni}{\let\PY@bf=\textbf\def\PY@tc##1{\textcolor[rgb]{0.44,0.44,0.44}{##1}}}
\@namedef{PY@tok@na}{\def\PY@tc##1{\textcolor[rgb]{0.41,0.47,0.13}{##1}}}
\@namedef{PY@tok@nt}{\let\PY@bf=\textbf\def\PY@tc##1{\textcolor[rgb]{0.00,0.50,0.00}{##1}}}
\@namedef{PY@tok@nd}{\def\PY@tc##1{\textcolor[rgb]{0.67,0.13,1.00}{##1}}}
\@namedef{PY@tok@s}{\def\PY@tc##1{\textcolor[rgb]{0.73,0.13,0.13}{##1}}}
\@namedef{PY@tok@sd}{\let\PY@it=\textit\def\PY@tc##1{\textcolor[rgb]{0.73,0.13,0.13}{##1}}}
\@namedef{PY@tok@si}{\let\PY@bf=\textbf\def\PY@tc##1{\textcolor[rgb]{0.64,0.35,0.47}{##1}}}
\@namedef{PY@tok@se}{\let\PY@bf=\textbf\def\PY@tc##1{\textcolor[rgb]{0.67,0.36,0.12}{##1}}}
\@namedef{PY@tok@sr}{\def\PY@tc##1{\textcolor[rgb]{0.64,0.35,0.47}{##1}}}
\@namedef{PY@tok@ss}{\def\PY@tc##1{\textcolor[rgb]{0.10,0.09,0.49}{##1}}}
\@namedef{PY@tok@sx}{\def\PY@tc##1{\textcolor[rgb]{0.00,0.50,0.00}{##1}}}
\@namedef{PY@tok@m}{\def\PY@tc##1{\textcolor[rgb]{0.40,0.40,0.40}{##1}}}
\@namedef{PY@tok@gh}{\let\PY@bf=\textbf\def\PY@tc##1{\textcolor[rgb]{0.00,0.00,0.50}{##1}}}
\@namedef{PY@tok@gu}{\let\PY@bf=\textbf\def\PY@tc##1{\textcolor[rgb]{0.50,0.00,0.50}{##1}}}
\@namedef{PY@tok@gd}{\def\PY@tc##1{\textcolor[rgb]{0.63,0.00,0.00}{##1}}}
\@namedef{PY@tok@gi}{\def\PY@tc##1{\textcolor[rgb]{0.00,0.52,0.00}{##1}}}
\@namedef{PY@tok@gr}{\def\PY@tc##1{\textcolor[rgb]{0.89,0.00,0.00}{##1}}}
\@namedef{PY@tok@ge}{\let\PY@it=\textit}
\@namedef{PY@tok@gs}{\let\PY@bf=\textbf}
\@namedef{PY@tok@gp}{\let\PY@bf=\textbf\def\PY@tc##1{\textcolor[rgb]{0.00,0.00,0.50}{##1}}}
\@namedef{PY@tok@go}{\def\PY@tc##1{\textcolor[rgb]{0.44,0.44,0.44}{##1}}}
\@namedef{PY@tok@gt}{\def\PY@tc##1{\textcolor[rgb]{0.00,0.27,0.87}{##1}}}
\@namedef{PY@tok@err}{\def\PY@bc##1{{\setlength{\fboxsep}{\string -\fboxrule}\fcolorbox[rgb]{1.00,0.00,0.00}{1,1,1}{\strut ##1}}}}
\@namedef{PY@tok@kc}{\let\PY@bf=\textbf\def\PY@tc##1{\textcolor[rgb]{0.00,0.50,0.00}{##1}}}
\@namedef{PY@tok@kd}{\let\PY@bf=\textbf\def\PY@tc##1{\textcolor[rgb]{0.00,0.50,0.00}{##1}}}
\@namedef{PY@tok@kn}{\let\PY@bf=\textbf\def\PY@tc##1{\textcolor[rgb]{0.00,0.50,0.00}{##1}}}
\@namedef{PY@tok@kr}{\let\PY@bf=\textbf\def\PY@tc##1{\textcolor[rgb]{0.00,0.50,0.00}{##1}}}
\@namedef{PY@tok@bp}{\def\PY@tc##1{\textcolor[rgb]{0.00,0.50,0.00}{##1}}}
\@namedef{PY@tok@fm}{\def\PY@tc##1{\textcolor[rgb]{0.00,0.00,1.00}{##1}}}
\@namedef{PY@tok@vc}{\def\PY@tc##1{\textcolor[rgb]{0.10,0.09,0.49}{##1}}}
\@namedef{PY@tok@vg}{\def\PY@tc##1{\textcolor[rgb]{0.10,0.09,0.49}{##1}}}
\@namedef{PY@tok@vi}{\def\PY@tc##1{\textcolor[rgb]{0.10,0.09,0.49}{##1}}}
\@namedef{PY@tok@vm}{\def\PY@tc##1{\textcolor[rgb]{0.10,0.09,0.49}{##1}}}
\@namedef{PY@tok@sa}{\def\PY@tc##1{\textcolor[rgb]{0.73,0.13,0.13}{##1}}}
\@namedef{PY@tok@sb}{\def\PY@tc##1{\textcolor[rgb]{0.73,0.13,0.13}{##1}}}
\@namedef{PY@tok@sc}{\def\PY@tc##1{\textcolor[rgb]{0.73,0.13,0.13}{##1}}}
\@namedef{PY@tok@dl}{\def\PY@tc##1{\textcolor[rgb]{0.73,0.13,0.13}{##1}}}
\@namedef{PY@tok@s2}{\def\PY@tc##1{\textcolor[rgb]{0.73,0.13,0.13}{##1}}}
\@namedef{PY@tok@sh}{\def\PY@tc##1{\textcolor[rgb]{0.73,0.13,0.13}{##1}}}
\@namedef{PY@tok@s1}{\def\PY@tc##1{\textcolor[rgb]{0.73,0.13,0.13}{##1}}}
\@namedef{PY@tok@mb}{\def\PY@tc##1{\textcolor[rgb]{0.40,0.40,0.40}{##1}}}
\@namedef{PY@tok@mf}{\def\PY@tc##1{\textcolor[rgb]{0.40,0.40,0.40}{##1}}}
\@namedef{PY@tok@mh}{\def\PY@tc##1{\textcolor[rgb]{0.40,0.40,0.40}{##1}}}
\@namedef{PY@tok@mi}{\def\PY@tc##1{\textcolor[rgb]{0.40,0.40,0.40}{##1}}}
\@namedef{PY@tok@il}{\def\PY@tc##1{\textcolor[rgb]{0.40,0.40,0.40}{##1}}}
\@namedef{PY@tok@mo}{\def\PY@tc##1{\textcolor[rgb]{0.40,0.40,0.40}{##1}}}
\@namedef{PY@tok@ch}{\let\PY@it=\textit\def\PY@tc##1{\textcolor[rgb]{0.24,0.48,0.48}{##1}}}
\@namedef{PY@tok@cm}{\let\PY@it=\textit\def\PY@tc##1{\textcolor[rgb]{0.24,0.48,0.48}{##1}}}
\@namedef{PY@tok@cpf}{\let\PY@it=\textit\def\PY@tc##1{\textcolor[rgb]{0.24,0.48,0.48}{##1}}}
\@namedef{PY@tok@c1}{\let\PY@it=\textit\def\PY@tc##1{\textcolor[rgb]{0.24,0.48,0.48}{##1}}}
\@namedef{PY@tok@cs}{\let\PY@it=\textit\def\PY@tc##1{\textcolor[rgb]{0.24,0.48,0.48}{##1}}}

\def\PYZbs{\char`\\}
\def\PYZus{\char`\_}
\def\PYZob{\char`\{}
\def\PYZcb{\char`\}}
\def\PYZca{\char`\^}
\def\PYZam{\char`\&}
\def\PYZlt{\char`\<}
\def\PYZgt{\char`\>}
\def\PYZsh{\char`\#}
\def\PYZpc{\char`\%}
\def\PYZdl{\char`\$}
\def\PYZhy{\char`\-}
\def\PYZsq{\char`\'}
\def\PYZdq{\char`\"}
\def\PYZti{\char`\~}
% for compatibility with earlier versions
\def\PYZat{@}
\def\PYZlb{[}
\def\PYZrb{]}
\makeatother


    % For linebreaks inside Verbatim environment from package fancyvrb.
    \makeatletter
        \newbox\Wrappedcontinuationbox
        \newbox\Wrappedvisiblespacebox
        \newcommand*\Wrappedvisiblespace {\textcolor{red}{\textvisiblespace}}
        \newcommand*\Wrappedcontinuationsymbol {\textcolor{red}{\llap{\tiny$\m@th\hookrightarrow$}}}
        \newcommand*\Wrappedcontinuationindent {3ex }
        \newcommand*\Wrappedafterbreak {\kern\Wrappedcontinuationindent\copy\Wrappedcontinuationbox}
        % Take advantage of the already applied Pygments mark-up to insert
        % potential linebreaks for TeX processing.
        %        {, <, #, %, $, ' and ": go to next line.
        %        _, }, ^, &, >, - and ~: stay at end of broken line.
        % Use of \textquotesingle for straight quote.
        \newcommand*\Wrappedbreaksatspecials {%
            \def\PYGZus{\discretionary{\char`\_}{\Wrappedafterbreak}{\char`\_}}%
            \def\PYGZob{\discretionary{}{\Wrappedafterbreak\char`\{}{\char`\{}}%
            \def\PYGZcb{\discretionary{\char`\}}{\Wrappedafterbreak}{\char`\}}}%
            \def\PYGZca{\discretionary{\char`\^}{\Wrappedafterbreak}{\char`\^}}%
            \def\PYGZam{\discretionary{\char`\&}{\Wrappedafterbreak}{\char`\&}}%
            \def\PYGZlt{\discretionary{}{\Wrappedafterbreak\char`\<}{\char`\<}}%
            \def\PYGZgt{\discretionary{\char`\>}{\Wrappedafterbreak}{\char`\>}}%
            \def\PYGZsh{\discretionary{}{\Wrappedafterbreak\char`\#}{\char`\#}}%
            \def\PYGZpc{\discretionary{}{\Wrappedafterbreak\char`\%}{\char`\%}}%
            \def\PYGZdl{\discretionary{}{\Wrappedafterbreak\char`\$}{\char`\$}}%
            \def\PYGZhy{\discretionary{\char`\-}{\Wrappedafterbreak}{\char`\-}}%
            \def\PYGZsq{\discretionary{}{\Wrappedafterbreak\textquotesingle}{\textquotesingle}}%
            \def\PYGZdq{\discretionary{}{\Wrappedafterbreak\char`\"}{\char`\"}}%
            \def\PYGZti{\discretionary{\char`\~}{\Wrappedafterbreak}{\char`\~}}%
        }
        % Some characters . , ; ? ! / are not pygmentized.
        % This macro makes them "active" and they will insert potential linebreaks
        \newcommand*\Wrappedbreaksatpunct {%
            \lccode`\~`\.\lowercase{\def~}{\discretionary{\hbox{\char`\.}}{\Wrappedafterbreak}{\hbox{\char`\.}}}%
            \lccode`\~`\,\lowercase{\def~}{\discretionary{\hbox{\char`\,}}{\Wrappedafterbreak}{\hbox{\char`\,}}}%
            \lccode`\~`\;\lowercase{\def~}{\discretionary{\hbox{\char`\;}}{\Wrappedafterbreak}{\hbox{\char`\;}}}%
            \lccode`\~`\:\lowercase{\def~}{\discretionary{\hbox{\char`\:}}{\Wrappedafterbreak}{\hbox{\char`\:}}}%
            \lccode`\~`\?\lowercase{\def~}{\discretionary{\hbox{\char`\?}}{\Wrappedafterbreak}{\hbox{\char`\?}}}%
            \lccode`\~`\!\lowercase{\def~}{\discretionary{\hbox{\char`\!}}{\Wrappedafterbreak}{\hbox{\char`\!}}}%
            \lccode`\~`\/\lowercase{\def~}{\discretionary{\hbox{\char`\/}}{\Wrappedafterbreak}{\hbox{\char`\/}}}%
            \catcode`\.\active
            \catcode`\,\active
            \catcode`\;\active
            \catcode`\:\active
            \catcode`\?\active
            \catcode`\!\active
            \catcode`\/\active
            \lccode`\~`\~
        }
    \makeatother

    \let\OriginalVerbatim=\Verbatim
    \makeatletter
    \renewcommand{\Verbatim}[1][1]{%
        %\parskip\z@skip
        \sbox\Wrappedcontinuationbox {\Wrappedcontinuationsymbol}%
        \sbox\Wrappedvisiblespacebox {\FV@SetupFont\Wrappedvisiblespace}%
        \def\FancyVerbFormatLine ##1{\hsize\linewidth
            \vtop{\raggedright\hyphenpenalty\z@\exhyphenpenalty\z@
                \doublehyphendemerits\z@\finalhyphendemerits\z@
                \strut ##1\strut}%
        }%
        % If the linebreak is at a space, the latter will be displayed as visible
        % space at end of first line, and a continuation symbol starts next line.
        % Stretch/shrink are however usually zero for typewriter font.
        \def\FV@Space {%
            \nobreak\hskip\z@ plus\fontdimen3\font minus\fontdimen4\font
            \discretionary{\copy\Wrappedvisiblespacebox}{\Wrappedafterbreak}
            {\kern\fontdimen2\font}%
        }%

        % Allow breaks at special characters using \PYG... macros.
        \Wrappedbreaksatspecials
        % Breaks at punctuation characters . , ; ? ! and / need catcode=\active
        \OriginalVerbatim[#1,codes*=\Wrappedbreaksatpunct]%
    }
    \makeatother

    % Exact colors from NB
    \definecolor{incolor}{HTML}{303F9F}
    \definecolor{outcolor}{HTML}{D84315}
    \definecolor{cellborder}{HTML}{CFCFCF}
    \definecolor{cellbackground}{HTML}{F7F7F7}

    % prompt
    \makeatletter
    \newcommand{\boxspacing}{\kern\kvtcb@left@rule\kern\kvtcb@boxsep}
    \makeatother
    \newcommand{\prompt}[4]{
        {\ttfamily\llap{{\color{#2}[#3]:\hspace{3pt}#4}}\vspace{-\baselineskip}}
    }
    

    
    % Prevent overflowing lines due to hard-to-break entities
    \sloppy
    % Setup hyperref package
    \hypersetup{
      breaklinks=true,  % so long urls are correctly broken across lines
      colorlinks=true,
      urlcolor=urlcolor,
      linkcolor=linkcolor,
      citecolor=citecolor,
      }
    % Slightly bigger margins than the latex defaults
    
    \geometry{verbose,tmargin=1in,bmargin=1in,lmargin=1in,rmargin=1in}
    
    

\begin{document}
    
    \maketitle
    
    

    
    \section{PUNTO 3 }\label{punto-3---alejandra-ruiz}

    \subsection{LECTURA DEL ARCHIVO WINE
QUALITY}\label{lectura-del-archivo-wine-quality}

    \begin{tcolorbox}[breakable, size=fbox, boxrule=1pt, pad at break*=1mm,colback=cellbackground, colframe=cellborder]
\prompt{In}{incolor}{2}{\boxspacing}
\begin{Verbatim}[commandchars=\\\{\}]
\PY{k+kn}{import} \PY{n+nn}{pandas} \PY{k}{as} \PY{n+nn}{pd}

\PY{n}{wine} \PY{o}{=} \PY{n}{pd}\PY{o}{.}\PY{n}{read\PYZus{}excel}\PY{p}{(}\PY{l+s+sa}{r}\PY{l+s+s2}{\PYZdq{}}\PY{l+s+s2}{D:}\PY{l+s+s2}{\PYZbs{}}\PY{l+s+s2}{MAESTRIA CIENCIA DE DATOS}\PY{l+s+s2}{\PYZbs{}}\PY{l+s+s2}{Analisis Cuantitativo}\PY{l+s+s2}{\PYZbs{}}\PY{l+s+s2}{datasets}\PY{l+s+s2}{\PYZbs{}}\PY{l+s+s2}{datos.xls}\PY{l+s+s2}{\PYZdq{}}\PY{p}{,} 
                     \PY{n}{sheet\PYZus{}name}\PY{o}{=}\PY{l+s+s2}{\PYZdq{}}\PY{l+s+s2}{Wine Quality}\PY{l+s+s2}{\PYZdq{}}\PY{p}{,}
                     \PY{n}{header}\PY{o}{=}\PY{l+m+mi}{2}\PY{p}{)}
\PY{n}{wine}
\end{Verbatim}
\end{tcolorbox}

            \begin{tcolorbox}[breakable, size=fbox, boxrule=.5pt, pad at break*=1mm, opacityfill=0]
\prompt{Out}{outcolor}{2}{\boxspacing}
\begin{Verbatim}[commandchars=\\\{\}]
      Calidad del Vino  Acidez Fija  Acidez Volátil  Ácido Cítrico  \textbackslash{}
0                    6          7.0            0.27           0.36
1                    6          6.3            0.30           0.34
2                    6          8.1            0.28           0.40
3                    6          7.2            0.23           0.32
4                    6          7.2            0.23           0.32
{\ldots}                {\ldots}          {\ldots}             {\ldots}            {\ldots}
4893                 6          6.2            0.21           0.29
4894                 5          6.6            0.32           0.36
4895                 6          6.5            0.24           0.19
4896                 7          5.5            0.29           0.30
4897                 6          6.0            0.21           0.38

      Azúcar Residual  Cloruros  Dióxido de Azúfre Libre  \textbackslash{}
0                20.7     0.045                     45.0
1                 1.6     0.049                     14.0
2                 6.9     0.050                     30.0
3                 8.5     0.058                     47.0
4                 8.5     0.058                     47.0
{\ldots}               {\ldots}       {\ldots}                      {\ldots}
4893              1.6     0.039                     24.0
4894              8.0     0.047                     57.0
4895              1.2     0.041                     30.0
4896              1.1     0.022                     20.0
4897              0.8     0.020                     22.0

      Dióxido de Azúfre Total  Densidad    pH  Sulfatos  Alcohol
0                       170.0   1.00100  3.00      0.45      8.8
1                       132.0   0.99400  3.30      0.49      9.5
2                        97.0   0.99510  3.26      0.44     10.1
3                       186.0   0.99560  3.19      0.40      9.9
4                       186.0   0.99560  3.19      0.40      9.9
{\ldots}                       {\ldots}       {\ldots}   {\ldots}       {\ldots}      {\ldots}
4893                     92.0   0.99114  3.27      0.50     11.2
4894                    168.0   0.99490  3.15      0.46      9.6
4895                    111.0   0.99254  2.99      0.46      9.4
4896                    110.0   0.98869  3.34      0.38     12.8
4897                     98.0   0.98941  3.26      0.32     11.8

[4898 rows x 12 columns]
\end{Verbatim}
\end{tcolorbox}
        
    \subsubsection{ELIMINAR COLUMNAS}\label{eliminar-columnas}

    \begin{tcolorbox}[breakable, size=fbox, boxrule=1pt, pad at break*=1mm,colback=cellbackground, colframe=cellborder]
\prompt{In}{incolor}{3}{\boxspacing}
\begin{Verbatim}[commandchars=\\\{\}]
\PY{n}{wine}\PY{o}{.}\PY{n}{drop}\PY{p}{(}\PY{n}{columns}\PY{o}{=}\PY{p}{[}\PY{l+s+s1}{\PYZsq{}}\PY{l+s+s1}{pH}\PY{l+s+s1}{\PYZsq{}}\PY{p}{,}\PY{l+s+s1}{\PYZsq{}}\PY{l+s+s1}{Sulfatos}\PY{l+s+s1}{\PYZsq{}}\PY{p}{,}\PY{l+s+s1}{\PYZsq{}}\PY{l+s+s1}{Acidez Volátil}\PY{l+s+s1}{\PYZsq{}}\PY{p}{,}\PY{l+s+s1}{\PYZsq{}}\PY{l+s+s1}{Acidez Fija}\PY{l+s+s1}{\PYZsq{}}\PY{p}{,}\PY{l+s+s1}{\PYZsq{}}\PY{l+s+s1}{Calidad del Vino}\PY{l+s+s1}{\PYZsq{}}\PY{p}{]}\PY{p}{,}
          \PY{n}{inplace}\PY{o}{=}\PY{k+kc}{True}\PY{p}{)}
\end{Verbatim}
\end{tcolorbox}

    \subsubsection{VERIFICO QUE TODAS LAS VARIABLES TENGAN EL TIPO DE DATO
CORRECTO Y QUE NO HAYAN
NULOS}\label{verifico-que-todas-las-variables-tengan-el-tipo-de-dato-correcto-y-que-no-hayan-nulos}

    \begin{tcolorbox}[breakable, size=fbox, boxrule=1pt, pad at break*=1mm,colback=cellbackground, colframe=cellborder]
\prompt{In}{incolor}{4}{\boxspacing}
\begin{Verbatim}[commandchars=\\\{\}]
\PY{n}{wine}\PY{o}{.}\PY{n}{info}\PY{p}{(}\PY{p}{)}
\end{Verbatim}
\end{tcolorbox}

    \begin{Verbatim}[commandchars=\\\{\}]
<class 'pandas.core.frame.DataFrame'>
RangeIndex: 4898 entries, 0 to 4897
Data columns (total 7 columns):
 \#   Column                   Non-Null Count  Dtype
---  ------                   --------------  -----
 0   Ácido Cítrico            4898 non-null   float64
 1   Azúcar Residual          4898 non-null   float64
 2   Cloruros                 4898 non-null   float64
 3   Dióxido de Azúfre Libre  4898 non-null   float64
 4   Dióxido de Azúfre Total  4898 non-null   float64
 5   Densidad                 4898 non-null   float64
 6   Alcohol                  4898 non-null   float64
dtypes: float64(7)
memory usage: 268.0 KB
    \end{Verbatim}

    \subsection{ESTANDARIZAR}\label{estandarizar}

    \subsubsection{HACEMOS USO DE LA LIBRERÍA
SKLEARN}\label{hacemos-uso-de-la-libreruxeda-sklearn}

    \begin{tcolorbox}[breakable, size=fbox, boxrule=1pt, pad at break*=1mm,colback=cellbackground, colframe=cellborder]
\prompt{In}{incolor}{5}{\boxspacing}
\begin{Verbatim}[commandchars=\\\{\}]
\PY{k+kn}{from} \PY{n+nn}{sklearn}\PY{n+nn}{.}\PY{n+nn}{preprocessing} \PY{k+kn}{import} \PY{n}{StandardScaler}
\end{Verbatim}
\end{tcolorbox}

    \begin{tcolorbox}[breakable, size=fbox, boxrule=1pt, pad at break*=1mm,colback=cellbackground, colframe=cellborder]
\prompt{In}{incolor}{6}{\boxspacing}
\begin{Verbatim}[commandchars=\\\{\}]
\PY{c+c1}{\PYZsh{} Inicializando el escalador}
\PY{n}{scaler} \PY{o}{=} \PY{n}{StandardScaler}\PY{p}{(}\PY{p}{)}

\PY{c+c1}{\PYZsh{} Ajustando y transformando los datos}
\PY{n}{standard\PYZus{}wine} \PY{o}{=} \PY{n}{pd}\PY{o}{.}\PY{n}{DataFrame}\PY{p}{(}\PY{n}{scaler}\PY{o}{.}\PY{n}{fit\PYZus{}transform}\PY{p}{(}\PY{n}{wine}\PY{p}{)}\PY{p}{,} \PY{n}{columns}\PY{o}{=}\PY{n}{wine}\PY{o}{.}\PY{n}{columns}\PY{p}{)}
\end{Verbatim}
\end{tcolorbox}

    \begin{tcolorbox}[breakable, size=fbox, boxrule=1pt, pad at break*=1mm,colback=cellbackground, colframe=cellborder]
\prompt{In}{incolor}{7}{\boxspacing}
\begin{Verbatim}[commandchars=\\\{\}]
\PY{n}{standard\PYZus{}wine}
\end{Verbatim}
\end{tcolorbox}

            \begin{tcolorbox}[breakable, size=fbox, boxrule=.5pt, pad at break*=1mm, opacityfill=0]
\prompt{Out}{outcolor}{7}{\boxspacing}
\begin{Verbatim}[commandchars=\\\{\}]
      Ácido Cítrico  Azúcar Residual  Cloruros  Dióxido de Azúfre Libre  \textbackslash{}
0          0.213280         2.821349 -0.035355                 0.569932
1          0.048001        -0.944765  0.147747                -1.253019
2          0.543838         0.100282  0.193523                -0.312141
3         -0.117278         0.415768  0.559727                 0.687541
4         -0.117278         0.415768  0.559727                 0.687541
{\ldots}             {\ldots}              {\ldots}       {\ldots}                      {\ldots}
4893      -0.365197        -0.944765 -0.310008                -0.664970
4894       0.213280         0.317179  0.056196                 1.275590
4895      -1.191592        -1.023637 -0.218457                -0.312141
4896      -0.282557        -1.043355 -1.088192                -0.900190
4897       0.378559        -1.102508 -1.179743                -0.782580

      Dióxido de Azúfre Total  Densidad   Alcohol
0                    0.744565  2.331512 -1.393152
1                   -0.149685 -0.009154 -0.824276
2                   -0.973336  0.358665 -0.336667
3                    1.121091  0.525855 -0.499203
4                    1.121091  0.525855 -0.499203
{\ldots}                       {\ldots}       {\ldots}       {\ldots}
4893                -1.091000 -0.965483  0.557282
4894                 0.697499  0.291789 -0.743008
4895                -0.643875 -0.497350 -0.905544
4896                -0.667408 -1.784717  1.857572
4897                -0.949803 -1.543962  1.044891

[4898 rows x 7 columns]
\end{Verbatim}
\end{tcolorbox}
        
    \subsection{CÁLCULO DE LAS
CORRELACIONES}\label{cuxe1lculo-de-las-correlaciones}

    \begin{tcolorbox}[breakable, size=fbox, boxrule=1pt, pad at break*=1mm,colback=cellbackground, colframe=cellborder]
\prompt{In}{incolor}{8}{\boxspacing}
\begin{Verbatim}[commandchars=\\\{\}]
\PY{k+kn}{import} \PY{n+nn}{seaborn} \PY{k}{as} \PY{n+nn}{sns}
\PY{k+kn}{import} \PY{n+nn}{matplotlib}\PY{n+nn}{.}\PY{n+nn}{pyplot} \PY{k}{as} \PY{n+nn}{plt}
\end{Verbatim}
\end{tcolorbox}

    \begin{tcolorbox}[breakable, size=fbox, boxrule=1pt, pad at break*=1mm,colback=cellbackground, colframe=cellborder]
\prompt{In}{incolor}{9}{\boxspacing}
\begin{Verbatim}[commandchars=\\\{\}]
\PY{n}{sns}\PY{o}{.}\PY{n}{pairplot}\PY{p}{(}\PY{n}{standard\PYZus{}wine}\PY{p}{)}
\PY{n}{plt}\PY{o}{.}\PY{n}{show}\PY{p}{(}\PY{p}{)}
\end{Verbatim}
\end{tcolorbox}

    \begin{center}
    \adjustimage{max size={0.9\linewidth}{0.9\paperheight}}{punto_3_files/punto_3_14_0.png}
    \end{center}
    { \hspace*{\fill} \\}
    
    \subsubsection{PEARSON}\label{pearson}

    \begin{tcolorbox}[breakable, size=fbox, boxrule=1pt, pad at break*=1mm,colback=cellbackground, colframe=cellborder]
\prompt{In}{incolor}{10}{\boxspacing}
\begin{Verbatim}[commandchars=\\\{\}]
\PY{n}{corr\PYZus{}pearson} \PY{o}{=} \PY{n}{standard\PYZus{}wine}\PY{o}{.}\PY{n}{corr}\PY{p}{(}\PY{n}{method}\PY{o}{=}\PY{l+s+s1}{\PYZsq{}}\PY{l+s+s1}{pearson}\PY{l+s+s1}{\PYZsq{}}\PY{p}{)}\PY{o}{.}\PY{n}{reset\PYZus{}index}\PY{p}{(}\PY{n}{drop}\PY{o}{=}\PY{k+kc}{True}\PY{p}{)}
\end{Verbatim}
\end{tcolorbox}

    \begin{tcolorbox}[breakable, size=fbox, boxrule=1pt, pad at break*=1mm,colback=cellbackground, colframe=cellborder]
\prompt{In}{incolor}{11}{\boxspacing}
\begin{Verbatim}[commandchars=\\\{\}]
\PY{n}{sns}\PY{o}{.}\PY{n}{set}\PY{p}{(}\PY{n}{font\PYZus{}scale}\PY{o}{=}\PY{l+m+mf}{1.0}\PY{p}{)}
\end{Verbatim}
\end{tcolorbox}

    \begin{tcolorbox}[breakable, size=fbox, boxrule=1pt, pad at break*=1mm,colback=cellbackground, colframe=cellborder]
\prompt{In}{incolor}{12}{\boxspacing}
\begin{Verbatim}[commandchars=\\\{\}]
\PY{n}{plt}\PY{o}{.}\PY{n}{figure}\PY{p}{(}\PY{n}{figsize}\PY{o}{=}\PY{p}{(}\PY{l+m+mi}{10}\PY{p}{,} \PY{l+m+mi}{8}\PY{p}{)}\PY{p}{)}
\PY{n}{sns}\PY{o}{.}\PY{n}{heatmap}\PY{p}{(}\PY{n}{corr\PYZus{}pearson}\PY{p}{,} \PY{n}{annot}\PY{o}{=}\PY{k+kc}{True}\PY{p}{,} \PY{n}{fmt}\PY{o}{=}\PY{l+s+s2}{\PYZdq{}}\PY{l+s+s2}{.2f}\PY{l+s+s2}{\PYZdq{}}\PY{p}{,} \PY{n}{cmap}\PY{o}{=}\PY{l+s+s2}{\PYZdq{}}\PY{l+s+s2}{coolwarm}\PY{l+s+s2}{\PYZdq{}}\PY{p}{,} \PY{n}{xticklabels}\PY{o}{=}\PY{n}{standard\PYZus{}wine}\PY{o}{.}\PY{n}{columns}\PY{p}{,} \PY{n}{yticklabels}\PY{o}{=}\PY{n}{standard\PYZus{}wine}\PY{o}{.}\PY{n}{columns}\PY{p}{)}
\PY{n}{plt}\PY{o}{.}\PY{n}{title}\PY{p}{(}\PY{l+s+s2}{\PYZdq{}}\PY{l+s+s2}{Correlación de Pearson}\PY{l+s+s2}{\PYZdq{}}\PY{p}{)}
\PY{n}{plt}\PY{o}{.}\PY{n}{show}\PY{p}{(}\PY{p}{)}
\end{Verbatim}
\end{tcolorbox}

    \begin{center}
    \adjustimage{max size={0.9\linewidth}{0.9\paperheight}}{punto_3_files/punto_3_18_0.png}
    \end{center}
    { \hspace*{\fill} \\}
    
    \subparagraph{La mayor correlación contra la variable Y (Densidad) Es
con Alcohol de manera negativa (-0.78), Ázucar Residual de manera
positiva (0.84), y una correlación media (0.53) con Dióxido de Azúfre
Total}\label{la-mayor-correlaciuxf3n-contra-la-variable-y-densidad-es-con-alcohol-de-manera-negativa--0.78-uxe1zucar-residual-de-manera-positiva-0.84-y-una-correlaciuxf3n-media-0.53-con-diuxf3xido-de-azuxfafre-total}

    \begin{tcolorbox}[breakable, size=fbox, boxrule=1pt, pad at break*=1mm,colback=cellbackground, colframe=cellborder]
\prompt{In}{incolor}{13}{\boxspacing}
\begin{Verbatim}[commandchars=\\\{\}]
\PY{n}{corr\PYZus{}pearson}
\end{Verbatim}
\end{tcolorbox}

            \begin{tcolorbox}[breakable, size=fbox, boxrule=.5pt, pad at break*=1mm, opacityfill=0]
\prompt{Out}{outcolor}{13}{\boxspacing}
\begin{Verbatim}[commandchars=\\\{\}]
   Ácido Cítrico  Azúcar Residual  Cloruros  Dióxido de Azúfre Libre  \textbackslash{}
0       1.000000         0.094212  0.114364                 0.094077
1       0.094212         1.000000  0.088685                 0.299098
2       0.114364         0.088685  1.000000                 0.101392
3       0.094077         0.299098  0.101392                 1.000000
4       0.121131         0.401439  0.198910                 0.615501
5       0.149503         0.838966  0.257211                 0.294210
6      -0.075729        -0.450631 -0.360189                -0.250104

   Dióxido de Azúfre Total  Densidad   Alcohol
0                 0.121131  0.149503 -0.075729
1                 0.401439  0.838966 -0.450631
2                 0.198910  0.257211 -0.360189
3                 0.615501  0.294210 -0.250104
4                 1.000000  0.529881 -0.448892
5                 0.529881  1.000000 -0.780138
6                -0.448892 -0.780138  1.000000
\end{Verbatim}
\end{tcolorbox}
        
    \subsubsection{KENDALL}\label{kendall}

    \begin{tcolorbox}[breakable, size=fbox, boxrule=1pt, pad at break*=1mm,colback=cellbackground, colframe=cellborder]
\prompt{In}{incolor}{14}{\boxspacing}
\begin{Verbatim}[commandchars=\\\{\}]
\PY{n}{corr\PYZus{}kendall} \PY{o}{=} \PY{n}{standard\PYZus{}wine}\PY{o}{.}\PY{n}{corr}\PY{p}{(}\PY{n}{method}\PY{o}{=}\PY{l+s+s1}{\PYZsq{}}\PY{l+s+s1}{kendall}\PY{l+s+s1}{\PYZsq{}}\PY{p}{)}\PY{o}{.}\PY{n}{reset\PYZus{}index}\PY{p}{(}\PY{n}{drop}\PY{o}{=}\PY{k+kc}{True}\PY{p}{)}
\end{Verbatim}
\end{tcolorbox}

    \begin{tcolorbox}[breakable, size=fbox, boxrule=1pt, pad at break*=1mm,colback=cellbackground, colframe=cellborder]
\prompt{In}{incolor}{15}{\boxspacing}
\begin{Verbatim}[commandchars=\\\{\}]
\PY{n}{plt}\PY{o}{.}\PY{n}{figure}\PY{p}{(}\PY{n}{figsize}\PY{o}{=}\PY{p}{(}\PY{l+m+mi}{10}\PY{p}{,} \PY{l+m+mi}{8}\PY{p}{)}\PY{p}{)}
\PY{n}{sns}\PY{o}{.}\PY{n}{heatmap}\PY{p}{(}\PY{n}{corr\PYZus{}kendall}\PY{p}{,} \PY{n}{annot}\PY{o}{=}\PY{k+kc}{True}\PY{p}{,} \PY{n}{fmt}\PY{o}{=}\PY{l+s+s2}{\PYZdq{}}\PY{l+s+s2}{.2f}\PY{l+s+s2}{\PYZdq{}}\PY{p}{,} \PY{n}{cmap}\PY{o}{=}\PY{l+s+s2}{\PYZdq{}}\PY{l+s+s2}{coolwarm}\PY{l+s+s2}{\PYZdq{}}\PY{p}{,} \PY{n}{xticklabels}\PY{o}{=}\PY{n}{standard\PYZus{}wine}\PY{o}{.}\PY{n}{columns}\PY{p}{,} \PY{n}{yticklabels}\PY{o}{=}\PY{n}{standard\PYZus{}wine}\PY{o}{.}\PY{n}{columns}\PY{p}{)}
\PY{n}{plt}\PY{o}{.}\PY{n}{title}\PY{p}{(}\PY{l+s+s2}{\PYZdq{}}\PY{l+s+s2}{Correlación de Kendall}\PY{l+s+s2}{\PYZdq{}}\PY{p}{)}
\PY{n}{plt}\PY{o}{.}\PY{n}{show}\PY{p}{(}\PY{p}{)}
\end{Verbatim}
\end{tcolorbox}

    \begin{center}
    \adjustimage{max size={0.9\linewidth}{0.9\paperheight}}{punto_3_files/punto_3_23_0.png}
    \end{center}
    { \hspace*{\fill} \\}
    
    \subparagraph{Comparando frente a Pearson, la correlación fuerte que
tenía Densidad y Ázucar Residual disminuyó (0.59) significativamente, y
con Alcohol también bajó pero no tanto (-0.64). Incrementó de forma
negativa un poco la correlación de Ácido Cítrico frente a todas las
variables.}\label{comparando-frente-a-pearson-la-correlaciuxf3n-fuerte-que-tenuxeda-densidad-y-uxe1zucar-residual-disminuyuxf3-0.59-significativamente-y-con-alcohol-tambiuxe9n-bajuxf3-pero-no-tanto--0.64.-incrementuxf3-de-forma-negativa-un-poco-la-correlaciuxf3n-de-uxe1cido-cuxedtrico-frente-a-todas-las-variables.}

    \subparagraph{Recordemos que Kendall Penaliza, por esta razón es menor,
adicionalmente Pearson es muy buneo cuando sí hay comportamientos
lineales, y al observar las gráficas de disperción, claramente se ve un
comportamiento elíptico entre Densidad y Alcohol, por eso Kendall no
bajó tanto esta correlación, pero la de Ázucar Residual tenía unos
puntos atípicos alejados de la concentración, pero que conservaban
linealidad, y por eso es que Kendall sí bajó significativamente esta
correlación, dado que Pearson sí se afecta bastante con los atípicos,
Kendall
NO.}\label{recordemos-que-kendall-penaliza-por-esta-razuxf3n-es-menor-adicionalmente-pearson-es-muy-buneo-cuando-suxed-hay-comportamientos-lineales-y-al-observar-las-gruxe1ficas-de-disperciuxf3n-claramente-se-ve-un-comportamiento-eluxedptico-entre-densidad-y-alcohol-por-eso-kendall-no-bajuxf3-tanto-esta-correlaciuxf3n-pero-la-de-uxe1zucar-residual-tenuxeda-unos-puntos-atuxedpicos-alejados-de-la-concentraciuxf3n-pero-que-conservaban-linealidad-y-por-eso-es-que-kendall-suxed-bajuxf3-significativamente-esta-correlaciuxf3n-dado-que-pearson-suxed-se-afecta-bastante-con-los-atuxedpicos-kendall-no.}

    \begin{tcolorbox}[breakable, size=fbox, boxrule=1pt, pad at break*=1mm,colback=cellbackground, colframe=cellborder]
\prompt{In}{incolor}{16}{\boxspacing}
\begin{Verbatim}[commandchars=\\\{\}]
\PY{n}{corr\PYZus{}kendall}
\end{Verbatim}
\end{tcolorbox}

            \begin{tcolorbox}[breakable, size=fbox, boxrule=.5pt, pad at break*=1mm, opacityfill=0]
\prompt{Out}{outcolor}{16}{\boxspacing}
\begin{Verbatim}[commandchars=\\\{\}]
   Ácido Cítrico  Azúcar Residual  Cloruros  Dióxido de Azúfre Libre  \textbackslash{}
0       1.000000         0.015329  0.022292                 0.060809
1       0.015329         1.000000  0.155274                 0.236748
2       0.022292         0.155274  1.000000                 0.113851
3       0.060809         0.236748  0.113851                 1.000000
4       0.062188         0.293319  0.257075                 0.444696
5       0.061542         0.588989  0.349119                 0.217295
6      -0.019981        -0.305601 -0.404039                -0.182539

   Dióxido de Azúfre Total  Densidad   Alcohol
0                 0.062188  0.061542 -0.019981
1                 0.293319  0.588989 -0.305601
2                 0.257075  0.349119 -0.404039
3                 0.444696  0.217295 -0.182539
4                 1.000000  0.388378 -0.325826
5                 0.388378  1.000000 -0.635104
6                -0.325826 -0.635104  1.000000
\end{Verbatim}
\end{tcolorbox}
        
    \subsubsection{SPEARMAN}\label{spearman}

    \begin{tcolorbox}[breakable, size=fbox, boxrule=1pt, pad at break*=1mm,colback=cellbackground, colframe=cellborder]
\prompt{In}{incolor}{17}{\boxspacing}
\begin{Verbatim}[commandchars=\\\{\}]
\PY{n}{corr\PYZus{}spearman} \PY{o}{=} \PY{n}{standard\PYZus{}wine}\PY{o}{.}\PY{n}{corr}\PY{p}{(}\PY{n}{method}\PY{o}{=}\PY{l+s+s1}{\PYZsq{}}\PY{l+s+s1}{spearman}\PY{l+s+s1}{\PYZsq{}}\PY{p}{)}\PY{o}{.}\PY{n}{reset\PYZus{}index}\PY{p}{(}\PY{n}{drop}\PY{o}{=}\PY{k+kc}{True}\PY{p}{)}
\end{Verbatim}
\end{tcolorbox}

    \begin{tcolorbox}[breakable, size=fbox, boxrule=1pt, pad at break*=1mm,colback=cellbackground, colframe=cellborder]
\prompt{In}{incolor}{18}{\boxspacing}
\begin{Verbatim}[commandchars=\\\{\}]
\PY{n}{plt}\PY{o}{.}\PY{n}{figure}\PY{p}{(}\PY{n}{figsize}\PY{o}{=}\PY{p}{(}\PY{l+m+mi}{10}\PY{p}{,} \PY{l+m+mi}{8}\PY{p}{)}\PY{p}{)}
\PY{n}{sns}\PY{o}{.}\PY{n}{heatmap}\PY{p}{(}\PY{n}{corr\PYZus{}spearman}\PY{p}{,} \PY{n}{annot}\PY{o}{=}\PY{k+kc}{True}\PY{p}{,} \PY{n}{fmt}\PY{o}{=}\PY{l+s+s2}{\PYZdq{}}\PY{l+s+s2}{.2f}\PY{l+s+s2}{\PYZdq{}}\PY{p}{,} \PY{n}{cmap}\PY{o}{=}\PY{l+s+s2}{\PYZdq{}}\PY{l+s+s2}{coolwarm}\PY{l+s+s2}{\PYZdq{}}\PY{p}{,} \PY{n}{xticklabels}\PY{o}{=}\PY{n}{standard\PYZus{}wine}\PY{o}{.}\PY{n}{columns}\PY{p}{,} \PY{n}{yticklabels}\PY{o}{=}\PY{n}{standard\PYZus{}wine}\PY{o}{.}\PY{n}{columns}\PY{p}{)}
\PY{n}{plt}\PY{o}{.}\PY{n}{title}\PY{p}{(}\PY{l+s+s2}{\PYZdq{}}\PY{l+s+s2}{Correlación de Spearman}\PY{l+s+s2}{\PYZdq{}}\PY{p}{)}
\PY{n}{plt}\PY{o}{.}\PY{n}{show}\PY{p}{(}\PY{p}{)}
\end{Verbatim}
\end{tcolorbox}

    \begin{center}
    \adjustimage{max size={0.9\linewidth}{0.9\paperheight}}{punto_3_files/punto_3_29_0.png}
    \end{center}
    { \hspace*{\fill} \\}
    
    \begin{tcolorbox}[breakable, size=fbox, boxrule=1pt, pad at break*=1mm,colback=cellbackground, colframe=cellborder]
\prompt{In}{incolor}{19}{\boxspacing}
\begin{Verbatim}[commandchars=\\\{\}]
\PY{n}{corr\PYZus{}spearman}
\end{Verbatim}
\end{tcolorbox}

            \begin{tcolorbox}[breakable, size=fbox, boxrule=.5pt, pad at break*=1mm, opacityfill=0]
\prompt{Out}{outcolor}{19}{\boxspacing}
\begin{Verbatim}[commandchars=\\\{\}]
   Ácido Cítrico  Azúcar Residual  Cloruros  Dióxido de Azúfre Libre  \textbackslash{}
0       1.000000         0.024621  0.032659                 0.088314
1       0.024621         1.000000  0.227844                 0.346107
2       0.032659         0.227844  1.000000                 0.167046
3       0.088314         0.346107  0.167046                 1.000000
4       0.093219         0.431252  0.375244                 0.618616
5       0.091425         0.780365  0.508302                 0.327822
6      -0.029170        -0.445257 -0.570806                -0.272569

   Dióxido de Azúfre Total  Densidad   Alcohol
0                 0.093219  0.091425 -0.029170
1                 0.431252  0.780365 -0.445257
2                 0.375244  0.508302 -0.570806
3                 0.618616  0.327822 -0.272569
4                 1.000000  0.563824 -0.476619
5                 0.563824  1.000000 -0.821855
6                -0.476619 -0.821855  1.000000
\end{Verbatim}
\end{tcolorbox}
        
    \subparagraph{Con Spearman frente a Pearson, bajó pero muy poco la
correlación entre Densidad y Ázucar Residual, pero incrementó la
correlación frente a las demás variables incluída Alcohol, exceptuando
Ácido Cítrico. Recordemos que Pearson es más sencible a los Outliers, y
como podemos observar en el diagrama de dispersión, el de Ázucar vs
Densidad, hay varios outliars, que aunque están alineados con los demás,
esto puede estar generando un incremento en la correlación vs el de
Spearman.}\label{con-spearman-frente-a-pearson-bajuxf3-pero-muy-poco-la-correlaciuxf3n-entre-densidad-y-uxe1zucar-residual-pero-incrementuxf3-la-correlaciuxf3n-frente-a-las-demuxe1s-variables-incluuxedda-alcohol-exceptuando-uxe1cido-cuxedtrico.-recordemos-que-pearson-es-muxe1s-sencible-a-los-outliers-y-como-podemos-observar-en-el-diagrama-de-dispersiuxf3n-el-de-uxe1zucar-vs-densidad-hay-varios-outliars-que-aunque-estuxe1n-alineados-con-los-demuxe1s-esto-puede-estar-generando-un-incremento-en-la-correlaciuxf3n-vs-el-de-spearman.}

    \subsection{MODELADO}\label{modelado}

    \subsubsection{PARTICIÓN DEL DATASET}\label{particiuxf3n-del-dataset}

    \begin{tcolorbox}[breakable, size=fbox, boxrule=1pt, pad at break*=1mm,colback=cellbackground, colframe=cellborder]
\prompt{In}{incolor}{20}{\boxspacing}
\begin{Verbatim}[commandchars=\\\{\}]
\PY{k+kn}{from} \PY{n+nn}{sklearn}\PY{n+nn}{.}\PY{n+nn}{model\PYZus{}selection} \PY{k+kn}{import} \PY{n}{train\PYZus{}test\PYZus{}split}
\end{Verbatim}
\end{tcolorbox}

    \begin{tcolorbox}[breakable, size=fbox, boxrule=1pt, pad at break*=1mm,colback=cellbackground, colframe=cellborder]
\prompt{In}{incolor}{21}{\boxspacing}
\begin{Verbatim}[commandchars=\\\{\}]
\PY{n}{y} \PY{o}{=} \PY{n}{standard\PYZus{}wine}\PY{p}{[}\PY{l+s+s1}{\PYZsq{}}\PY{l+s+s1}{Densidad}\PY{l+s+s1}{\PYZsq{}}\PY{p}{]}
\PY{n}{X} \PY{o}{=} \PY{n}{standard\PYZus{}wine}\PY{o}{.}\PY{n}{drop}\PY{p}{(}\PY{n}{columns}\PY{o}{=}\PY{l+s+s1}{\PYZsq{}}\PY{l+s+s1}{Densidad}\PY{l+s+s1}{\PYZsq{}}\PY{p}{)}
\end{Verbatim}
\end{tcolorbox}

    \begin{tcolorbox}[breakable, size=fbox, boxrule=1pt, pad at break*=1mm,colback=cellbackground, colframe=cellborder]
\prompt{In}{incolor}{22}{\boxspacing}
\begin{Verbatim}[commandchars=\\\{\}]
\PY{n}{X\PYZus{}train}\PY{p}{,} \PY{n}{X\PYZus{}test}\PY{p}{,} \PY{n}{y\PYZus{}train}\PY{p}{,} \PY{n}{y\PYZus{}test} \PY{o}{=} \PY{n}{train\PYZus{}test\PYZus{}split}\PY{p}{(}\PY{n}{X}\PY{p}{,} \PY{n}{y}\PY{p}{,} \PY{n}{test\PYZus{}size}\PY{o}{=}\PY{l+m+mf}{0.2}\PY{p}{,} \PY{n}{random\PYZus{}state}\PY{o}{=}\PY{l+m+mi}{0}\PY{p}{)}
\end{Verbatim}
\end{tcolorbox}

    \subsubsection{Recordemos que la Matriz más invertible es la Matriz
Identidad, la cuál sería que no hubiera ninguna correlación entre
variables, (tener correlaciones cercanas a 0), La Calidad de una Matriz
se evalúa a partir de su Capacidad de Invertibilidad, y como analizamos,
la mayoría de correlaciones entre las X está más cercanas a 0, por lo
que validamos este supuesto para obtener buenos valores de
ˆβ}\label{recordemos-que-la-matriz-muxe1s-invertible-es-la-matriz-identidad-la-cuuxe1l-seruxeda-que-no-hubiera-ninguna-correlaciuxf3n-entre-variables-tener-correlaciones-cercanas-a-0-la-calidad-de-una-matriz-se-evaluxfaa-a-partir-de-su-capacidad-de-invertibilidad-y-como-analizamos-la-mayoruxeda-de-correlaciones-entre-las-x-estuxe1-muxe1s-cercanas-a-0-por-lo-que-validamos-este-supuesto-para-obtener-buenos-valores-de-ux2c6ux3b2}

% \subsubsection{Calcular ˆβ(⋅) y ^β0(·) ESTIMADORES MÍNIMOS CUADRADOS con
% Pearson, Generar y Evaluar el
% Modelo}\label{calcular-ux2c6ux3b2-y-ux2c6ux3b20-estimadores-muxednimos-cuadrados-con-pearson-generar-y-evaluar-el-modelo}

    \paragraph{Separar la Matriz de Correlación, dejar sólo entre las
Variables X y dejar la de la Variable Y con X (Como Densidad es la
Penúltima Columna, significa que la penúltima fila también representa la
correlación de la X con Y
-Densidad-)}\label{separar-la-matriz-de-correlaciuxf3n-dejar-suxf3lo-entre-las-variables-x-y-dejar-la-de-la-variable-y-con-x-como-densidad-es-la-penuxfaltima-columna-significa-que-la-penuxfaltima-fila-tambiuxe9n-representa-la-correlaciuxf3n-de-la-x-con-y--densidad-}

    \begin{tcolorbox}[breakable, size=fbox, boxrule=1pt, pad at break*=1mm,colback=cellbackground, colframe=cellborder]
\prompt{In}{incolor}{23}{\boxspacing}
\begin{Verbatim}[commandchars=\\\{\}]
\PY{n}{p\PYZus{}xy} \PY{o}{=} \PY{n}{corr\PYZus{}pearson}\PY{p}{[}\PY{l+s+s1}{\PYZsq{}}\PY{l+s+s1}{Densidad}\PY{l+s+s1}{\PYZsq{}}\PY{p}{]}
\PY{n}{p\PYZus{}xx} \PY{o}{=} \PY{n}{corr\PYZus{}pearson}\PY{o}{.}\PY{n}{drop}\PY{p}{(}\PY{n}{columns}\PY{o}{=}\PY{l+s+s1}{\PYZsq{}}\PY{l+s+s1}{Densidad}\PY{l+s+s1}{\PYZsq{}}\PY{p}{)}
\PY{n}{indice\PYZus{}penultima\PYZus{}fila} \PY{o}{=} \PY{n}{p\PYZus{}xx}\PY{o}{.}\PY{n}{index}\PY{p}{[}\PY{o}{\PYZhy{}}\PY{l+m+mi}{2}\PY{p}{]}  \PY{c+c1}{\PYZsh{} Obtener el índice de la penúltima fila}
\PY{n}{p\PYZus{}xx} \PY{o}{=} \PY{n}{p\PYZus{}xx}\PY{o}{.}\PY{n}{drop}\PY{p}{(}\PY{n}{index}\PY{o}{=}\PY{n}{indice\PYZus{}penultima\PYZus{}fila}\PY{p}{)}
\PY{n}{p\PYZus{}xy} \PY{o}{=} \PY{n}{p\PYZus{}xy}\PY{o}{.}\PY{n}{drop}\PY{p}{(}\PY{n}{index}\PY{o}{=}\PY{n}{p\PYZus{}xy}\PY{o}{.}\PY{n}{index}\PY{p}{[}\PY{o}{\PYZhy{}}\PY{l+m+mi}{2}\PY{p}{]}\PY{p}{)}
\end{Verbatim}
\end{tcolorbox}

    \begin{tcolorbox}[breakable, size=fbox, boxrule=1pt, pad at break*=1mm,colback=cellbackground, colframe=cellborder]
\prompt{In}{incolor}{24}{\boxspacing}
\begin{Verbatim}[commandchars=\\\{\}]
\PY{n}{p\PYZus{}xy}
\end{Verbatim}
\end{tcolorbox}

            \begin{tcolorbox}[breakable, size=fbox, boxrule=.5pt, pad at break*=1mm, opacityfill=0]
\prompt{Out}{outcolor}{24}{\boxspacing}
\begin{Verbatim}[commandchars=\\\{\}]
0    0.149503
1    0.838966
2    0.257211
3    0.294210
4    0.529881
6   -0.780138
Name: Densidad, dtype: float64
\end{Verbatim}
\end{tcolorbox}
        
    \paragraph{Invertir la Matriz de Correlación entre las Variables
X}\label{invertir-la-matriz-de-correlaciuxf3n-entre-las-variables-x}

    \begin{tcolorbox}[breakable, size=fbox, boxrule=1pt, pad at break*=1mm,colback=cellbackground, colframe=cellborder]
\prompt{In}{incolor}{25}{\boxspacing}
\begin{Verbatim}[commandchars=\\\{\}]
\PY{k+kn}{import} \PY{n+nn}{numpy} \PY{k}{as} \PY{n+nn}{np}
\end{Verbatim}
\end{tcolorbox}

    \begin{tcolorbox}[breakable, size=fbox, boxrule=1pt, pad at break*=1mm,colback=cellbackground, colframe=cellborder]
\prompt{In}{incolor}{26}{\boxspacing}
\begin{Verbatim}[commandchars=\\\{\}]
\PY{n}{inv\PYZus{}pxx} \PY{o}{=} \PY{n}{np}\PY{o}{.}\PY{n}{linalg}\PY{o}{.}\PY{n}{inv}\PY{p}{(}\PY{n}{p\PYZus{}xx}\PY{p}{)}
\end{Verbatim}
\end{tcolorbox}

    \paragraph{MULTIPLIACIÓN DE MATRICES PARA CALCULAR
BETA}\label{multipliaciuxf3n-de-matrices-para-calcular-beta}

    \begin{tcolorbox}[breakable, size=fbox, boxrule=1pt, pad at break*=1mm,colback=cellbackground, colframe=cellborder]
\prompt{In}{incolor}{27}{\boxspacing}
\begin{Verbatim}[commandchars=\\\{\}]
\PY{n}{beta} \PY{o}{=} \PY{n}{np}\PY{o}{.}\PY{n}{dot}\PY{p}{(}\PY{n}{inv\PYZus{}pxx}\PY{p}{,} \PY{n}{p\PYZus{}xy}\PY{p}{)}
\end{Verbatim}
\end{tcolorbox}

    \begin{tcolorbox}[breakable, size=fbox, boxrule=1pt, pad at break*=1mm,colback=cellbackground, colframe=cellborder]
\prompt{In}{incolor}{28}{\boxspacing}
\begin{Verbatim}[commandchars=\\\{\}]
\PY{n}{beta}
\end{Verbatim}
\end{tcolorbox}

            \begin{tcolorbox}[breakable, size=fbox, boxrule=.5pt, pad at break*=1mm, opacityfill=0]
\prompt{Out}{outcolor}{28}{\boxspacing}
\begin{Verbatim}[commandchars=\\\{\}]
array([ 0.04912064,  0.5980971 ,  0.01421093, -0.08283074,  0.12203417,
       -0.46771405])
\end{Verbatim}
\end{tcolorbox}
        
    \paragraph{OBTENER LA MEDIA DE X y Y (ˆμY
ˆμX)}\label{obtener-la-media-de-x-y-y-ux2c6ux3bcy-ux2c6ux3bcx}

    \begin{tcolorbox}[breakable, size=fbox, boxrule=1pt, pad at break*=1mm,colback=cellbackground, colframe=cellborder]
\prompt{In}{incolor}{29}{\boxspacing}
\begin{Verbatim}[commandchars=\\\{\}]
\PY{n}{mu\PYZus{}Y} \PY{o}{=} \PY{n}{y\PYZus{}train}\PY{o}{.}\PY{n}{mean}\PY{p}{(}\PY{p}{)}
\PY{n}{mu\PYZus{}X} \PY{o}{=} \PY{n}{X\PYZus{}train}\PY{o}{.}\PY{n}{mean}\PY{p}{(}\PY{p}{)}
\end{Verbatim}
\end{tcolorbox}

    \paragraph{CALCULAR BETA0}\label{calcular-beta0}

    \begin{tcolorbox}[breakable, size=fbox, boxrule=1pt, pad at break*=1mm,colback=cellbackground, colframe=cellborder]
\prompt{In}{incolor}{30}{\boxspacing}
\begin{Verbatim}[commandchars=\\\{\}]
\PY{n}{beta\PYZus{}0} \PY{o}{=} \PY{n}{mu\PYZus{}Y} \PY{o}{\PYZhy{}} \PY{n}{np}\PY{o}{.}\PY{n}{dot}\PY{p}{(}\PY{n}{mu\PYZus{}X}\PY{p}{,} \PY{n}{beta}\PY{p}{)}
\end{Verbatim}
\end{tcolorbox}

    \paragraph{Recordemos que Beta nos dice cuánto se espera que cambie Y
con un cambio unitario en las variables X, pero como hicimos una
estandarización, debe tenerse en cuenta que este cambio representa en el
valor estandarizado y no en el valor
real.}\label{recordemos-que-beta-nos-dice-cuuxe1nto-se-espera-que-cambie-y-con-un-cambio-unitario-en-las-variables-x-pero-como-hicimos-una-estandarizaciuxf3n-debe-tenerse-en-cuenta-que-este-cambio-representa-en-el-valor-estandarizado-y-no-en-el-valor-real.}

    \paragraph{MODELO}\label{modelo}

    \begin{tcolorbox}[breakable, size=fbox, boxrule=1pt, pad at break*=1mm,colback=cellbackground, colframe=cellborder]
\prompt{In}{incolor}{31}{\boxspacing}
\begin{Verbatim}[commandchars=\\\{\}]
\PY{n}{y\PYZus{}pred\PYZus{}test} \PY{o}{=} \PY{n}{np}\PY{o}{.}\PY{n}{dot}\PY{p}{(}\PY{n}{X\PYZus{}test}\PY{p}{,} \PY{n}{beta}\PY{p}{)} \PY{o}{+} \PY{n}{beta\PYZus{}0}
\end{Verbatim}
\end{tcolorbox}

    \begin{tcolorbox}[breakable, size=fbox, boxrule=1pt, pad at break*=1mm,colback=cellbackground, colframe=cellborder]
\prompt{In}{incolor}{32}{\boxspacing}
\begin{Verbatim}[commandchars=\\\{\}]
\PY{n}{y\PYZus{}pred\PYZus{}test}
\end{Verbatim}
\end{tcolorbox}

            \begin{tcolorbox}[breakable, size=fbox, boxrule=.5pt, pad at break*=1mm, opacityfill=0]
\prompt{Out}{outcolor}{32}{\boxspacing}
\begin{Verbatim}[commandchars=\\\{\}]
array([-4.48681074e-01,  4.70796910e-01, -5.76152127e-01, -9.57462920e-01,
        8.27468996e-01,  5.74656484e-01, -7.21273130e-01, -8.64718189e-01,
       -8.58474351e-01,  1.12970887e+00,  5.76218024e-01,  1.49882313e+00,
       ...
        4.36813376e-01,  4.99813251e-02,  1.28680238e+00,  9.64560648e-01])
\end{Verbatim}
\end{tcolorbox}
        
    \paragraph{CALCULAR ERROR CUADRÁTICO MEDIO (MSE), LA RAIZ DEL ERROR
(RMSE) Y EL COEFICIENTE DE DETERMINACIÓN
(R²)}\label{calcular-error-cuadruxe1tico-medio-mse-la-raiz-del-error-rmse-y-el-coeficiente-de-determinaciuxf3n-ruxb2}

    \begin{tcolorbox}[breakable, size=fbox, boxrule=1pt, pad at break*=1mm,colback=cellbackground, colframe=cellborder]
\prompt{In}{incolor}{33}{\boxspacing}
\begin{Verbatim}[commandchars=\\\{\}]
\PY{k+kn}{from} \PY{n+nn}{sklearn}\PY{n+nn}{.}\PY{n+nn}{metrics} \PY{k+kn}{import} \PY{n}{mean\PYZus{}squared\PYZus{}error}\PY{p}{,} \PY{n}{r2\PYZus{}score}
\end{Verbatim}
\end{tcolorbox}

    \begin{tcolorbox}[breakable, size=fbox, boxrule=1pt, pad at break*=1mm,colback=cellbackground, colframe=cellborder]
\prompt{In}{incolor}{34}{\boxspacing}
\begin{Verbatim}[commandchars=\\\{\}]
\PY{n}{mse} \PY{o}{=} \PY{n}{mean\PYZus{}squared\PYZus{}error}\PY{p}{(}\PY{n}{y\PYZus{}test}\PY{p}{,} \PY{n}{y\PYZus{}pred\PYZus{}test}\PY{p}{)}
\PY{n}{rmse} \PY{o}{=} \PY{n}{np}\PY{o}{.}\PY{n}{sqrt}\PY{p}{(}\PY{n}{mse}\PY{p}{)}
\PY{n}{r2} \PY{o}{=} \PY{n}{r2\PYZus{}score}\PY{p}{(}\PY{n}{y\PYZus{}test}\PY{p}{,} \PY{n}{y\PYZus{}pred\PYZus{}test}\PY{p}{)}
\end{Verbatim}
\end{tcolorbox}

    \begin{tcolorbox}[breakable, size=fbox, boxrule=1pt, pad at break*=1mm,colback=cellbackground, colframe=cellborder]
\prompt{In}{incolor}{35}{\boxspacing}
\begin{Verbatim}[commandchars=\\\{\}]
\PY{n}{mse}
\end{Verbatim}
\end{tcolorbox}

            \begin{tcolorbox}[breakable, size=fbox, boxrule=.5pt, pad at break*=1mm, opacityfill=0]
\prompt{Out}{outcolor}{35}{\boxspacing}
\begin{Verbatim}[commandchars=\\\{\}]
0.13755303612153544
\end{Verbatim}
\end{tcolorbox}
        
    \begin{tcolorbox}[breakable, size=fbox, boxrule=1pt, pad at break*=1mm,colback=cellbackground, colframe=cellborder]
\prompt{In}{incolor}{36}{\boxspacing}
\begin{Verbatim}[commandchars=\\\{\}]
\PY{n}{rmse}
\end{Verbatim}
\end{tcolorbox}

            \begin{tcolorbox}[breakable, size=fbox, boxrule=.5pt, pad at break*=1mm, opacityfill=0]
\prompt{Out}{outcolor}{36}{\boxspacing}
\begin{Verbatim}[commandchars=\\\{\}]
0.370881431351767
\end{Verbatim}
\end{tcolorbox}
        
    \begin{tcolorbox}[breakable, size=fbox, boxrule=1pt, pad at break*=1mm,colback=cellbackground, colframe=cellborder]
\prompt{In}{incolor}{37}{\boxspacing}
\begin{Verbatim}[commandchars=\\\{\}]
\PY{n}{r2}
\end{Verbatim}
\end{tcolorbox}

            \begin{tcolorbox}[breakable, size=fbox, boxrule=.5pt, pad at break*=1mm, opacityfill=0]
\prompt{Out}{outcolor}{37}{\boxspacing}
\begin{Verbatim}[commandchars=\\\{\}]
0.8810170491518581
\end{Verbatim}
\end{tcolorbox}
        
    \subsubsection{REPETIR EL PROCESO PARA
KENDALL}\label{repetir-el-proceso-para-kendall}

    \begin{tcolorbox}[breakable, size=fbox, boxrule=1pt, pad at break*=1mm,colback=cellbackground, colframe=cellborder]
\prompt{In}{incolor}{38}{\boxspacing}
\begin{Verbatim}[commandchars=\\\{\}]
\PY{n}{p\PYZus{}xy} \PY{o}{=} \PY{n}{corr\PYZus{}kendall}\PY{p}{[}\PY{l+s+s1}{\PYZsq{}}\PY{l+s+s1}{Densidad}\PY{l+s+s1}{\PYZsq{}}\PY{p}{]}
\PY{n}{p\PYZus{}xx} \PY{o}{=} \PY{n}{corr\PYZus{}kendall}\PY{o}{.}\PY{n}{drop}\PY{p}{(}\PY{n}{columns}\PY{o}{=}\PY{l+s+s1}{\PYZsq{}}\PY{l+s+s1}{Densidad}\PY{l+s+s1}{\PYZsq{}}\PY{p}{)}
\PY{n}{indice\PYZus{}penultima\PYZus{}fila} \PY{o}{=} \PY{n}{p\PYZus{}xx}\PY{o}{.}\PY{n}{index}\PY{p}{[}\PY{o}{\PYZhy{}}\PY{l+m+mi}{2}\PY{p}{]}  \PY{c+c1}{\PYZsh{} Obtener el índice de la penúltima fila}
\PY{n}{p\PYZus{}xx} \PY{o}{=} \PY{n}{p\PYZus{}xx}\PY{o}{.}\PY{n}{drop}\PY{p}{(}\PY{n}{index}\PY{o}{=}\PY{n}{indice\PYZus{}penultima\PYZus{}fila}\PY{p}{)}
\PY{n}{p\PYZus{}xy} \PY{o}{=} \PY{n}{p\PYZus{}xy}\PY{o}{.}\PY{n}{drop}\PY{p}{(}\PY{n}{index}\PY{o}{=}\PY{n}{p\PYZus{}xy}\PY{o}{.}\PY{n}{index}\PY{p}{[}\PY{o}{\PYZhy{}}\PY{l+m+mi}{2}\PY{p}{]}\PY{p}{)}
\end{Verbatim}
\end{tcolorbox}

    \begin{tcolorbox}[breakable, size=fbox, boxrule=1pt, pad at break*=1mm,colback=cellbackground, colframe=cellborder]
\prompt{In}{incolor}{39}{\boxspacing}
\begin{Verbatim}[commandchars=\\\{\}]
\PY{n}{inv\PYZus{}pxx} \PY{o}{=} \PY{n}{np}\PY{o}{.}\PY{n}{linalg}\PY{o}{.}\PY{n}{inv}\PY{p}{(}\PY{n}{p\PYZus{}xx}\PY{p}{)}
\end{Verbatim}
\end{tcolorbox}

    \begin{tcolorbox}[breakable, size=fbox, boxrule=1pt, pad at break*=1mm,colback=cellbackground, colframe=cellborder]
\prompt{In}{incolor}{40}{\boxspacing}
\begin{Verbatim}[commandchars=\\\{\}]
\PY{n}{beta\PYZus{}kendall} \PY{o}{=} \PY{n}{np}\PY{o}{.}\PY{n}{dot}\PY{p}{(}\PY{n}{inv\PYZus{}pxx}\PY{p}{,} \PY{n}{p\PYZus{}xy}\PY{p}{)}
\PY{n}{beta\PYZus{}kendall}
\end{Verbatim}
\end{tcolorbox}

            \begin{tcolorbox}[breakable, size=fbox, boxrule=.5pt, pad at break*=1mm, opacityfill=0]
\prompt{Out}{outcolor}{40}{\boxspacing}
\begin{Verbatim}[commandchars=\\\{\}]
array([ 0.03908945,  0.41339588,  0.07885295, -0.02163421,  0.10921176,
       -0.4444938 ])
\end{Verbatim}
\end{tcolorbox}
        
    \begin{tcolorbox}[breakable, size=fbox, boxrule=1pt, pad at break*=1mm,colback=cellbackground, colframe=cellborder]
\prompt{In}{incolor}{41}{\boxspacing}
\begin{Verbatim}[commandchars=\\\{\}]
\PY{n}{beta\PYZus{}0} \PY{o}{=} \PY{n}{mu\PYZus{}Y} \PY{o}{\PYZhy{}} \PY{n}{np}\PY{o}{.}\PY{n}{dot}\PY{p}{(}\PY{n}{mu\PYZus{}X}\PY{p}{,} \PY{n}{beta\PYZus{}kendall}\PY{p}{)}
\end{Verbatim}
\end{tcolorbox}

    \begin{tcolorbox}[breakable, size=fbox, boxrule=1pt, pad at break*=1mm,colback=cellbackground, colframe=cellborder]
\prompt{In}{incolor}{42}{\boxspacing}
\begin{Verbatim}[commandchars=\\\{\}]
\PY{n}{y\PYZus{}pred\PYZus{}test\PYZus{}kendall} \PY{o}{=} \PY{n}{np}\PY{o}{.}\PY{n}{dot}\PY{p}{(}\PY{n}{X\PYZus{}test}\PY{p}{,} \PY{n}{beta\PYZus{}kendall}\PY{p}{)} \PY{o}{+} \PY{n}{beta\PYZus{}0}
\end{Verbatim}
\end{tcolorbox}

    \begin{tcolorbox}[breakable, size=fbox, boxrule=1pt, pad at break*=1mm,colback=cellbackground, colframe=cellborder]
\prompt{In}{incolor}{43}{\boxspacing}
\begin{Verbatim}[commandchars=\\\{\}]
\PY{n}{mse} \PY{o}{=} \PY{n}{mean\PYZus{}squared\PYZus{}error}\PY{p}{(}\PY{n}{y\PYZus{}test}\PY{p}{,} \PY{n}{y\PYZus{}pred\PYZus{}test\PYZus{}kendall}\PY{p}{)}
\PY{n+nb}{print}\PY{p}{(}\PY{l+s+sa}{f}\PY{l+s+s2}{\PYZdq{}}\PY{l+s+s2}{mse: }\PY{l+s+si}{\PYZob{}}\PY{n}{mse}\PY{l+s+si}{\PYZcb{}}\PY{l+s+s2}{\PYZdq{}}\PY{p}{)}
\PY{n}{rmse} \PY{o}{=} \PY{n}{np}\PY{o}{.}\PY{n}{sqrt}\PY{p}{(}\PY{n}{mse}\PY{p}{)}
\PY{n+nb}{print}\PY{p}{(}\PY{l+s+sa}{f}\PY{l+s+s2}{\PYZdq{}}\PY{l+s+s2}{rmse: }\PY{l+s+si}{\PYZob{}}\PY{n}{rmse}\PY{l+s+si}{\PYZcb{}}\PY{l+s+s2}{\PYZdq{}}\PY{p}{)}
\PY{n}{r2} \PY{o}{=} \PY{n}{r2\PYZus{}score}\PY{p}{(}\PY{n}{y\PYZus{}test}\PY{p}{,} \PY{n}{y\PYZus{}pred\PYZus{}test\PYZus{}kendall}\PY{p}{)}
\PY{n+nb}{print}\PY{p}{(}\PY{l+s+sa}{f}\PY{l+s+s2}{\PYZdq{}}\PY{l+s+s2}{r2: }\PY{l+s+si}{\PYZob{}}\PY{n}{r2}\PY{l+s+si}{\PYZcb{}}\PY{l+s+s2}{\PYZdq{}}\PY{p}{)}
\end{Verbatim}
\end{tcolorbox}

    \begin{Verbatim}[commandchars=\\\{\}]
mse: 0.20428729823767142
rmse: 0.45198152422158966
r2: 0.8232921188040059
    \end{Verbatim}

    \subsubsection{REPETIR EL PROCESO PARA
SPEARMAN}\label{repetir-el-proceso-para-spearman}

    \begin{tcolorbox}[breakable, size=fbox, boxrule=1pt, pad at break*=1mm,colback=cellbackground, colframe=cellborder]
\prompt{In}{incolor}{44}{\boxspacing}
\begin{Verbatim}[commandchars=\\\{\}]
\PY{n}{p\PYZus{}xy} \PY{o}{=} \PY{n}{corr\PYZus{}spearman}\PY{p}{[}\PY{l+s+s1}{\PYZsq{}}\PY{l+s+s1}{Densidad}\PY{l+s+s1}{\PYZsq{}}\PY{p}{]}
\PY{n}{p\PYZus{}xx} \PY{o}{=} \PY{n}{corr\PYZus{}spearman}\PY{o}{.}\PY{n}{drop}\PY{p}{(}\PY{n}{columns}\PY{o}{=}\PY{l+s+s1}{\PYZsq{}}\PY{l+s+s1}{Densidad}\PY{l+s+s1}{\PYZsq{}}\PY{p}{)}
\PY{n}{indice\PYZus{}penultima\PYZus{}fila} \PY{o}{=} \PY{n}{p\PYZus{}xx}\PY{o}{.}\PY{n}{index}\PY{p}{[}\PY{o}{\PYZhy{}}\PY{l+m+mi}{2}\PY{p}{]}  \PY{c+c1}{\PYZsh{} Obtener el índice de la penúltima fila}
\PY{n}{p\PYZus{}xx} \PY{o}{=} \PY{n}{p\PYZus{}xx}\PY{o}{.}\PY{n}{drop}\PY{p}{(}\PY{n}{index}\PY{o}{=}\PY{n}{indice\PYZus{}penultima\PYZus{}fila}\PY{p}{)}
\PY{n}{p\PYZus{}xy} \PY{o}{=} \PY{n}{p\PYZus{}xy}\PY{o}{.}\PY{n}{drop}\PY{p}{(}\PY{n}{index}\PY{o}{=}\PY{n}{p\PYZus{}xy}\PY{o}{.}\PY{n}{index}\PY{p}{[}\PY{o}{\PYZhy{}}\PY{l+m+mi}{2}\PY{p}{]}\PY{p}{)}
\end{Verbatim}
\end{tcolorbox}

    \begin{tcolorbox}[breakable, size=fbox, boxrule=1pt, pad at break*=1mm,colback=cellbackground, colframe=cellborder]
\prompt{In}{incolor}{45}{\boxspacing}
\begin{Verbatim}[commandchars=\\\{\}]
\PY{n}{inv\PYZus{}pxx} \PY{o}{=} \PY{n}{np}\PY{o}{.}\PY{n}{linalg}\PY{o}{.}\PY{n}{inv}\PY{p}{(}\PY{n}{p\PYZus{}xx}\PY{p}{)}
\end{Verbatim}
\end{tcolorbox}

    \begin{tcolorbox}[breakable, size=fbox, boxrule=1pt, pad at break*=1mm,colback=cellbackground, colframe=cellborder]
\prompt{In}{incolor}{46}{\boxspacing}
\begin{Verbatim}[commandchars=\\\{\}]
\PY{n}{beta\PYZus{}spearman} \PY{o}{=} \PY{n}{np}\PY{o}{.}\PY{n}{dot}\PY{p}{(}\PY{n}{inv\PYZus{}pxx}\PY{p}{,} \PY{n}{p\PYZus{}xy}\PY{p}{)}
\PY{n}{beta\PYZus{}spearman}
\end{Verbatim}
\end{tcolorbox}

            \begin{tcolorbox}[breakable, size=fbox, boxrule=.5pt, pad at break*=1mm, opacityfill=0]
\prompt{Out}{outcolor}{46}{\boxspacing}
\begin{Verbatim}[commandchars=\\\{\}]
array([ 0.057818  ,  0.50869835,  0.06059505, -0.07683468,  0.11309123,
       -0.52612008])
\end{Verbatim}
\end{tcolorbox}
        
    \begin{tcolorbox}[breakable, size=fbox, boxrule=1pt, pad at break*=1mm,colback=cellbackground, colframe=cellborder]
\prompt{In}{incolor}{47}{\boxspacing}
\begin{Verbatim}[commandchars=\\\{\}]
\PY{n}{beta\PYZus{}0} \PY{o}{=} \PY{n}{mu\PYZus{}Y} \PY{o}{\PYZhy{}} \PY{n}{np}\PY{o}{.}\PY{n}{dot}\PY{p}{(}\PY{n}{mu\PYZus{}X}\PY{p}{,} \PY{n}{beta\PYZus{}spearman}\PY{p}{)}
\end{Verbatim}
\end{tcolorbox}

    \begin{tcolorbox}[breakable, size=fbox, boxrule=1pt, pad at break*=1mm,colback=cellbackground, colframe=cellborder]
\prompt{In}{incolor}{48}{\boxspacing}
\begin{Verbatim}[commandchars=\\\{\}]
\PY{n}{y\PYZus{}pred\PYZus{}test\PYZus{}spearman} \PY{o}{=} \PY{n}{np}\PY{o}{.}\PY{n}{dot}\PY{p}{(}\PY{n}{X\PYZus{}test}\PY{p}{,} \PY{n}{beta\PYZus{}spearman}\PY{p}{)} \PY{o}{+} \PY{n}{beta\PYZus{}0}
\end{Verbatim}
\end{tcolorbox}

    \begin{tcolorbox}[breakable, size=fbox, boxrule=1pt, pad at break*=1mm,colback=cellbackground, colframe=cellborder]
\prompt{In}{incolor}{49}{\boxspacing}
\begin{Verbatim}[commandchars=\\\{\}]
\PY{n}{mse} \PY{o}{=} \PY{n}{mean\PYZus{}squared\PYZus{}error}\PY{p}{(}\PY{n}{y\PYZus{}test}\PY{p}{,} \PY{n}{y\PYZus{}pred\PYZus{}test\PYZus{}spearman}\PY{p}{)}
\PY{n+nb}{print}\PY{p}{(}\PY{l+s+sa}{f}\PY{l+s+s2}{\PYZdq{}}\PY{l+s+s2}{mse: }\PY{l+s+si}{\PYZob{}}\PY{n}{mse}\PY{l+s+si}{\PYZcb{}}\PY{l+s+s2}{\PYZdq{}}\PY{p}{)}
\PY{n}{rmse} \PY{o}{=} \PY{n}{np}\PY{o}{.}\PY{n}{sqrt}\PY{p}{(}\PY{n}{mse}\PY{p}{)}
\PY{n+nb}{print}\PY{p}{(}\PY{l+s+sa}{f}\PY{l+s+s2}{\PYZdq{}}\PY{l+s+s2}{rmse: }\PY{l+s+si}{\PYZob{}}\PY{n}{rmse}\PY{l+s+si}{\PYZcb{}}\PY{l+s+s2}{\PYZdq{}}\PY{p}{)}
\PY{n}{r2} \PY{o}{=} \PY{n}{r2\PYZus{}score}\PY{p}{(}\PY{n}{y\PYZus{}test}\PY{p}{,} \PY{n}{y\PYZus{}pred\PYZus{}test\PYZus{}spearman}\PY{p}{)}
\PY{n+nb}{print}\PY{p}{(}\PY{l+s+sa}{f}\PY{l+s+s2}{\PYZdq{}}\PY{l+s+s2}{r2: }\PY{l+s+si}{\PYZob{}}\PY{n}{r2}\PY{l+s+si}{\PYZcb{}}\PY{l+s+s2}{\PYZdq{}}\PY{p}{)}
\end{Verbatim}
\end{tcolorbox}

    \begin{Verbatim}[commandchars=\\\{\}]
mse: 0.1598208640101793
rmse: 0.39977601730241313
r2: 0.8617554468937412
    \end{Verbatim}

    \subparagraph{COMPARANDO LOS BETAS (COEFICIENTES DE REGRESIÓN), SE
MANTIENE LA DIRECCIÓN Y EN CONCLUSIÓN LAS VARIABLES QUE TIENEN UNA
FUERTE CORRELACIÓN (ÁZUCAR RESIDUAL Y ALCOHOL) SE MANTIENEN. RECORDEMOS
QUE EL COMPORTAMIENTO NO SE PUEDE ANALIZAR DIRECTAMENTE EN LAS
VARIABLES, PORQUE FUERON ESTANDARIZADAS
INICIALMENTE.}\label{comparando-los-betas-coeficientes-de-regresiuxf3n-se-mantiene-la-direcciuxf3n-y-en-conclusiuxf3n-las-variables-que-tienen-una-fuerte-correlaciuxf3n-uxe1zucar-residual-y-alcohol-se-mantienen.-recordemos-que-el-comportamiento-no-se-puede-analizar-directamente-en-las-variables-porque-fueron-estandarizadas-inicialmente.}

    \subsubsection{AHORA ANALICEMOS LAS MEDIDAS DEL RESULTADO DE LOS
MODELOS}\label{ahora-analicemos-las-medidas-del-resultado-de-los-modelos}

\subsubsection{El MSE: Mide el promedio de los cuadrados de los errores,
es decir, la diferencia entre los valores predichos y los valores
reales. Por tanto buscamos que el MSE sea bajo, indicando que el modelo
tiene menor error en sus
predicciones.}\label{el-mse-mide-el-promedio-de-los-cuadrados-de-los-errores-es-decir-la-diferencia-entre-los-valores-predichos-y-los-valores-reales.-por-tanto-buscamos-que-el-mse-sea-bajo-indicando-que-el-modelo-tiene-menor-error-en-sus-predicciones.}

\subsubsection{el RMSE es la Raíz Cuadrada del MSE, busca estar en la
escala original de los datos, también se desea que sea bajo, puede
ajustar mejor el valor, porque penaliza si hay errores muy
grandes.}\label{el-rmse-es-la-rauxedz-cuadrada-del-mse-busca-estar-en-la-escala-original-de-los-datos-tambiuxe9n-se-desea-que-sea-bajo-puede-ajustar-mejor-el-valor-porque-penaliza-si-hay-errores-muy-grandes.}

\paragraph{El menor MSE como el RMSE fue el de Pearson, seguido del de
Spearman por una diferencia muy
pequeña.}\label{el-menor-mse-como-el-rmse-fue-el-de-pearson-seguido-del-de-spearman-por-una-diferencia-muy-pequeuxf1a.}

\subsubsection{R2 (Coeficiente de determinación): Es una medida de la
proporción de la varianza en la variable dependiente (Y). Un R2 más alto
indica que el modelo explica una mayor proporción de la variabilidad en
los datos de respuesta, finalemnte se busca tener un R2
grande.}\label{r2-coeficiente-de-determinaciuxf3n-es-una-medida-de-la-proporciuxf3n-de-la-varianza-en-la-variable-dependiente-y.-un-r2-muxe1s-alto-indica-que-el-modelo-explica-una-mayor-proporciuxf3n-de-la-variabilidad-en-los-datos-de-respuesta-finalemnte-se-busca-tener-un-r2-grande.}

\paragraph{De igual manera, el de Pearson dio un R2
mayor.}\label{de-igual-manera-el-de-pearson-dio-un-r2-mayor.}

\subsubsection{Con esto se puede concluir que sí hay un buen
comportamiento lineal entre los datos X vs Y, ya que Pearson es muy
bueno para anlizar comportamientos
lineales.}\label{con-esto-se-puede-concluir-que-suxed-hay-un-buen-comportamiento-lineal-entre-los-datos-x-vs-y-ya-que-pearson-es-muy-bueno-para-anlizar-comportamientos-lineales.}

    \subsection{PRUEBA DE HIPÓTESIS}\label{prueba-de-hipuxf3tesis}

    \subsubsection{ANALIZAR LOS RESIDUOS DEL MODELO CON
PEARSON}\label{analizar-los-residuos-del-modelo-con-pearson}

    \begin{tcolorbox}[breakable, size=fbox, boxrule=1pt, pad at break*=1mm,colback=cellbackground, colframe=cellborder]
\prompt{In}{incolor}{50}{\boxspacing}
\begin{Verbatim}[commandchars=\\\{\}]
\PY{n}{residuos} \PY{o}{=} \PY{n}{y\PYZus{}test} \PY{o}{\PYZhy{}} \PY{n}{y\PYZus{}pred\PYZus{}test}
\end{Verbatim}
\end{tcolorbox}

    \begin{tcolorbox}[breakable, size=fbox, boxrule=1pt, pad at break*=1mm,colback=cellbackground, colframe=cellborder]
\prompt{In}{incolor}{51}{\boxspacing}
\begin{Verbatim}[commandchars=\\\{\}]
\PY{n}{residuos}
\end{Verbatim}
\end{tcolorbox}

            \begin{tcolorbox}[breakable, size=fbox, boxrule=.5pt, pad at break*=1mm, opacityfill=0]
\prompt{Out}{outcolor}{51}{\boxspacing}
\begin{Verbatim}[commandchars=\\\{\}]
2762   -0.082107
42     -0.011818
1419   -0.235516
3664   -0.469466
2125   -0.268176
          {\ldots}
2111    0.193262
1828    0.323108
1256    0.141493
3335   -0.011934
230     0.096304
Name: Densidad, Length: 980, dtype: float64
\end{Verbatim}
\end{tcolorbox}
        
    \paragraph{INDEPENDENCIA (PRUEBA DE DURBIN
WATSON)}\label{independencia-prueba-de-durbin-watson}

    \begin{tcolorbox}[breakable, size=fbox, boxrule=1pt, pad at break*=1mm,colback=cellbackground, colframe=cellborder]
\prompt{In}{incolor}{52}{\boxspacing}
\begin{Verbatim}[commandchars=\\\{\}]
\PY{n}{pip} \PY{n}{install} \PY{n}{statsmodels} \PY{o}{\PYZhy{}}\PY{o}{\PYZhy{}}\PY{n}{upgrade}
\end{Verbatim}
\end{tcolorbox}

    \begin{tcolorbox}[breakable, size=fbox, boxrule=1pt, pad at break*=1mm,colback=cellbackground, colframe=cellborder]
\prompt{In}{incolor}{53}{\boxspacing}
\begin{Verbatim}[commandchars=\\\{\}]
\PY{k+kn}{from} \PY{n+nn}{scipy} \PY{k+kn}{import} \PY{n}{stats}
\PY{k+kn}{from} \PY{n+nn}{statsmodels}\PY{n+nn}{.}\PY{n+nn}{stats}\PY{n+nn}{.}\PY{n+nn}{stattools} \PY{k+kn}{import} \PY{n}{durbin\PYZus{}watson}
\end{Verbatim}
\end{tcolorbox}

    \begin{tcolorbox}[breakable, size=fbox, boxrule=1pt, pad at break*=1mm,colback=cellbackground, colframe=cellborder]
\prompt{In}{incolor}{54}{\boxspacing}
\begin{Verbatim}[commandchars=\\\{\}]
\PY{n}{dw} \PY{o}{=} \PY{n}{durbin\PYZus{}watson}\PY{p}{(}\PY{n}{residuos}\PY{p}{)}
\end{Verbatim}
\end{tcolorbox}

    \begin{tcolorbox}[breakable, size=fbox, boxrule=1pt, pad at break*=1mm,colback=cellbackground, colframe=cellborder]
\prompt{In}{incolor}{55}{\boxspacing}
\begin{Verbatim}[commandchars=\\\{\}]
\PY{n}{dw}
\end{Verbatim}
\end{tcolorbox}

            \begin{tcolorbox}[breakable, size=fbox, boxrule=.5pt, pad at break*=1mm, opacityfill=0]
\prompt{Out}{outcolor}{55}{\boxspacing}
\begin{Verbatim}[commandchars=\\\{\}]
1.9122533543275961
\end{Verbatim}
\end{tcolorbox}
        
    \subparagraph{Valores del estadístico Durbin-Watson cercanos a 2
sugieren que no hay autocorrelación, valores significativamente menores
que 2 indican autocorrelación positiva, y valores significativamente
mayores que 2 indican autocorrelación negativa. Buscamos que NO haya
autocorrelación}\label{valores-del-estaduxedstico-durbin-watson-cercanos-a-2-sugieren-que-no-hay-autocorrelaciuxf3n-valores-significativamente-menores-que-2-indican-autocorrelaciuxf3n-positiva-y-valores-significativamente-mayores-que-2-indican-autocorrelaciuxf3n-negativa.-buscamos-que-no-haya-autocorrelaciuxf3n}

    \begin{tcolorbox}[breakable, size=fbox, boxrule=1pt, pad at break*=1mm,colback=cellbackground, colframe=cellborder]
\prompt{In}{incolor}{56}{\boxspacing}
\begin{Verbatim}[commandchars=\\\{\}]
\PY{c+c1}{\PYZsh{} Graficar los residuos}
\PY{n}{plt}\PY{o}{.}\PY{n}{scatter}\PY{p}{(}\PY{n}{y\PYZus{}pred\PYZus{}test}\PY{p}{,} \PY{n}{residuos}\PY{p}{)}
\PY{n}{plt}\PY{o}{.}\PY{n}{xlabel}\PY{p}{(}\PY{l+s+s1}{\PYZsq{}}\PY{l+s+s1}{Valores Predichos}\PY{l+s+s1}{\PYZsq{}}\PY{p}{)}
\PY{n}{plt}\PY{o}{.}\PY{n}{ylabel}\PY{p}{(}\PY{l+s+s1}{\PYZsq{}}\PY{l+s+s1}{Residuos}\PY{l+s+s1}{\PYZsq{}}\PY{p}{)}
\PY{n}{plt}\PY{o}{.}\PY{n}{axhline}\PY{p}{(}\PY{n}{y}\PY{o}{=}\PY{l+m+mi}{0}\PY{p}{,} \PY{n}{color}\PY{o}{=}\PY{l+s+s1}{\PYZsq{}}\PY{l+s+s1}{red}\PY{l+s+s1}{\PYZsq{}}\PY{p}{,} \PY{n}{linestyle}\PY{o}{=}\PY{l+s+s1}{\PYZsq{}}\PY{l+s+s1}{\PYZhy{}\PYZhy{}}\PY{l+s+s1}{\PYZsq{}}\PY{p}{)}
\PY{n}{plt}\PY{o}{.}\PY{n}{title}\PY{p}{(}\PY{l+s+s1}{\PYZsq{}}\PY{l+s+s1}{Gráfico de Residuos}\PY{l+s+s1}{\PYZsq{}}\PY{p}{)}
\PY{n}{plt}\PY{o}{.}\PY{n}{show}\PY{p}{(}\PY{p}{)}
\end{Verbatim}
\end{tcolorbox}

    \begin{center}
    \adjustimage{max size={0.9\linewidth}{0.9\paperheight}}{punto_3_files/punto_3_88_0.png}
    \end{center}
    { \hspace*{\fill} \\}
    
    \paragraph{NORMALIDAD (SHAPIRO WILKS)}\label{normalidad-shapiro-wilks}

    \begin{tcolorbox}[breakable, size=fbox, boxrule=1pt, pad at break*=1mm,colback=cellbackground, colframe=cellborder]
\prompt{In}{incolor}{57}{\boxspacing}
\begin{Verbatim}[commandchars=\\\{\}]
\PY{n}{shapiro\PYZus{}test} \PY{o}{=} \PY{n}{stats}\PY{o}{.}\PY{n}{shapiro}\PY{p}{(}\PY{n}{residuos}\PY{p}{)}
\PY{n+nb}{print}\PY{p}{(}\PY{l+s+sa}{f}\PY{l+s+s2}{\PYZdq{}}\PY{l+s+s2}{Shapiro\PYZhy{}Wilk test statistic: }\PY{l+s+si}{\PYZob{}}\PY{n}{shapiro\PYZus{}test}\PY{p}{[}\PY{l+m+mi}{0}\PY{p}{]}\PY{l+s+si}{\PYZcb{}}\PY{l+s+s2}{, p\PYZhy{}value: }\PY{l+s+si}{\PYZob{}}\PY{n}{shapiro\PYZus{}test}\PY{p}{[}\PY{l+m+mi}{1}\PY{p}{]}\PY{l+s+si}{\PYZcb{}}\PY{l+s+s2}{\PYZdq{}}\PY{p}{)}
\end{Verbatim}
\end{tcolorbox}

    \begin{Verbatim}[commandchars=\\\{\}]
Shapiro-Wilk test statistic: 0.5886951684951782, p-value: 4.540207024412407e-43
    \end{Verbatim}

    \subparagraph{AL SER UN P-VALUE TAN PEQUEÑO (Menor a 0.05) RECHAZAMOS LA
HIÓTESIS, POR TANTO DECIMOS QUE NO HAY
NORMALIDAD}\label{al-ser-un-p-value-tan-pequeuxf1o-menor-a-0.05-rechazamos-la-hiuxf3tesis-por-tanto-decimos-que-no-hay-normalidad}

    \begin{tcolorbox}[breakable, size=fbox, boxrule=1pt, pad at break*=1mm,colback=cellbackground, colframe=cellborder]
\prompt{In}{incolor}{58}{\boxspacing}
\begin{Verbatim}[commandchars=\\\{\}]
\PY{k+kn}{import} \PY{n+nn}{statsmodels}\PY{n+nn}{.}\PY{n+nn}{api} \PY{k}{as} \PY{n+nn}{sm}
\PY{n}{sm}\PY{o}{.}\PY{n}{qqplot}\PY{p}{(}\PY{n}{residuos}\PY{p}{,} \PY{n}{line} \PY{o}{=}\PY{l+s+s1}{\PYZsq{}}\PY{l+s+s1}{45}\PY{l+s+s1}{\PYZsq{}}\PY{p}{,} \PY{n}{fit}\PY{o}{=}\PY{k+kc}{True}\PY{p}{)}
\PY{n}{plt}\PY{o}{.}\PY{n}{title}\PY{p}{(}\PY{l+s+s1}{\PYZsq{}}\PY{l+s+s1}{QQ\PYZhy{}plot de los Residuos}\PY{l+s+s1}{\PYZsq{}}\PY{p}{)}
\PY{n}{plt}\PY{o}{.}\PY{n}{show}\PY{p}{(}\PY{p}{)}
\end{Verbatim}
\end{tcolorbox}

    \begin{center}
    \adjustimage{max size={0.9\linewidth}{0.9\paperheight}}{punto_3_files/punto_3_92_0.png}
    \end{center}
    { \hspace*{\fill} \\}
    
    \subparagraph{DESPUÉS DE VER EL GRÁFICO, EFECTIVAMENTE LOS PUNTOS NO
ESTÁN TAN ENCIMA DE LA RECTA
TEÓRICA}\label{despuuxe9s-de-ver-el-gruxe1fico-efectivamente-los-puntos-no-estuxe1n-tan-encima-de-la-recta-teuxf3rica}

    \paragraph{MEDIA CERO (Valor Esperado de Ei = 0 -One Sample
t-test)}\label{media-cero-valor-esperado-de-ei-0--one-sample-t-test}

    \begin{tcolorbox}[breakable, size=fbox, boxrule=1pt, pad at break*=1mm,colback=cellbackground, colframe=cellborder]
\prompt{In}{incolor}{59}{\boxspacing}
\begin{Verbatim}[commandchars=\\\{\}]
\PY{k+kn}{from} \PY{n+nn}{scipy}\PY{n+nn}{.}\PY{n+nn}{stats} \PY{k+kn}{import} \PY{n}{ttest\PYZus{}1samp}

\PY{n}{t\PYZus{}statistic}\PY{p}{,} \PY{n}{p\PYZus{}value} \PY{o}{=} \PY{n}{ttest\PYZus{}1samp}\PY{p}{(}\PY{n}{residuos}\PY{p}{,} \PY{l+m+mi}{0}\PY{p}{)}

\PY{n+nb}{print}\PY{p}{(}\PY{l+s+sa}{f}\PY{l+s+s2}{\PYZdq{}}\PY{l+s+s2}{T\PYZhy{}statistic: }\PY{l+s+si}{\PYZob{}}\PY{n}{t\PYZus{}statistic}\PY{l+s+si}{\PYZcb{}}\PY{l+s+s2}{\PYZdq{}}\PY{p}{)}
\PY{n+nb}{print}\PY{p}{(}\PY{l+s+sa}{f}\PY{l+s+s2}{\PYZdq{}}\PY{l+s+s2}{P\PYZhy{}value: }\PY{l+s+si}{\PYZob{}}\PY{n}{p\PYZus{}value}\PY{l+s+si}{\PYZcb{}}\PY{l+s+s2}{\PYZdq{}}\PY{p}{)}
\end{Verbatim}
\end{tcolorbox}

    \begin{Verbatim}[commandchars=\\\{\}]
T-statistic: 1.0024556826995998
P-value: 0.31637126431723644
    \end{Verbatim}

    \subparagraph{ESTA VEZ EL P-VALUE ES MAYOR A 0.05 POR TANTO NO SE
RECHAZA LA
HIPÓTESIS}\label{esta-vez-el-p-value-es-mayor-a-0.05-por-tanto-no-se-rechaza-la-hipuxf3tesis}

    \begin{tcolorbox}[breakable, size=fbox, boxrule=1pt, pad at break*=1mm,colback=cellbackground, colframe=cellborder]
\prompt{In}{incolor}{60}{\boxspacing}
\begin{Verbatim}[commandchars=\\\{\}]
\PY{n}{sns}\PY{o}{.}\PY{n}{boxplot}\PY{p}{(}\PY{n}{x}\PY{o}{=}\PY{n}{residuos}\PY{p}{)}
\PY{n}{plt}\PY{o}{.}\PY{n}{title}\PY{p}{(}\PY{l+s+s1}{\PYZsq{}}\PY{l+s+s1}{Boxplot de los Residuos}\PY{l+s+s1}{\PYZsq{}}\PY{p}{)}
\PY{n}{plt}\PY{o}{.}\PY{n}{xlabel}\PY{p}{(}\PY{l+s+s1}{\PYZsq{}}\PY{l+s+s1}{Residuos}\PY{l+s+s1}{\PYZsq{}}\PY{p}{)}
\PY{n}{plt}\PY{o}{.}\PY{n}{axvline}\PY{p}{(}\PY{n}{x}\PY{o}{=}\PY{l+m+mi}{0}\PY{p}{,} \PY{n}{color}\PY{o}{=}\PY{l+s+s1}{\PYZsq{}}\PY{l+s+s1}{red}\PY{l+s+s1}{\PYZsq{}}\PY{p}{,} \PY{n}{linestyle}\PY{o}{=}\PY{l+s+s1}{\PYZsq{}}\PY{l+s+s1}{\PYZhy{}\PYZhy{}}\PY{l+s+s1}{\PYZsq{}}\PY{p}{)}  \PY{c+c1}{\PYZsh{} Línea en x=0 para referencia}
\PY{n}{plt}\PY{o}{.}\PY{n}{show}\PY{p}{(}\PY{p}{)}
\end{Verbatim}
\end{tcolorbox}

    \begin{center}
    \adjustimage{max size={0.9\linewidth}{0.9\paperheight}}{punto_3_files/punto_3_97_0.png}
    \end{center}
    { \hspace*{\fill} \\}
    
    \begin{tcolorbox}[breakable, size=fbox, boxrule=1pt, pad at break*=1mm,colback=cellbackground, colframe=cellborder]
\prompt{In}{incolor}{61}{\boxspacing}
\begin{Verbatim}[commandchars=\\\{\}]
\PY{n}{plt}\PY{o}{.}\PY{n}{hist}\PY{p}{(}\PY{n}{residuos}\PY{p}{,} \PY{n}{bins}\PY{o}{=}\PY{l+m+mi}{30}\PY{p}{,} \PY{n}{edgecolor}\PY{o}{=}\PY{l+s+s1}{\PYZsq{}}\PY{l+s+s1}{black}\PY{l+s+s1}{\PYZsq{}}\PY{p}{,} \PY{n}{alpha}\PY{o}{=}\PY{l+m+mf}{0.7}\PY{p}{)}
\PY{n}{plt}\PY{o}{.}\PY{n}{title}\PY{p}{(}\PY{l+s+s1}{\PYZsq{}}\PY{l+s+s1}{Histograma de los Residuos}\PY{l+s+s1}{\PYZsq{}}\PY{p}{)}
\PY{n}{plt}\PY{o}{.}\PY{n}{xlabel}\PY{p}{(}\PY{l+s+s1}{\PYZsq{}}\PY{l+s+s1}{Residuos}\PY{l+s+s1}{\PYZsq{}}\PY{p}{)}
\PY{n}{plt}\PY{o}{.}\PY{n}{ylabel}\PY{p}{(}\PY{l+s+s1}{\PYZsq{}}\PY{l+s+s1}{Frecuencia}\PY{l+s+s1}{\PYZsq{}}\PY{p}{)}
\PY{n}{plt}\PY{o}{.}\PY{n}{axvline}\PY{p}{(}\PY{n}{x}\PY{o}{=}\PY{l+m+mi}{0}\PY{p}{,} \PY{n}{color}\PY{o}{=}\PY{l+s+s1}{\PYZsq{}}\PY{l+s+s1}{red}\PY{l+s+s1}{\PYZsq{}}\PY{p}{,} \PY{n}{linestyle}\PY{o}{=}\PY{l+s+s1}{\PYZsq{}}\PY{l+s+s1}{\PYZhy{}\PYZhy{}}\PY{l+s+s1}{\PYZsq{}}\PY{p}{)}  \PY{c+c1}{\PYZsh{} Línea vertical para la media en x=0}
\PY{n}{plt}\PY{o}{.}\PY{n}{show}\PY{p}{(}\PY{p}{)}
\end{Verbatim}
\end{tcolorbox}

    \begin{center}
    \adjustimage{max size={0.9\linewidth}{0.9\paperheight}}{punto_3_files/punto_3_98_0.png}
    \end{center}
    { \hspace*{\fill} \\}
    
    \paragraph{HOMOCEDASTICIDAD (Varianza
Constante)}\label{homocedasticidad-varianza-constante}

    \begin{tcolorbox}[breakable, size=fbox, boxrule=1pt, pad at break*=1mm,colback=cellbackground, colframe=cellborder]
\prompt{In}{incolor}{62}{\boxspacing}
\begin{Verbatim}[commandchars=\\\{\}]
\PY{c+c1}{\PYZsh{} Calcular la desviación estándar de los residuos}
\PY{n}{std\PYZus{}residuos} \PY{o}{=} \PY{n}{np}\PY{o}{.}\PY{n}{std}\PY{p}{(}\PY{n}{residuos}\PY{p}{)}

\PY{c+c1}{\PYZsh{} Crear el gráfico de dispersión de residuos}
\PY{n}{plt}\PY{o}{.}\PY{n}{scatter}\PY{p}{(}\PY{n}{y\PYZus{}pred\PYZus{}test}\PY{p}{,} \PY{n}{residuos}\PY{p}{,} \PY{n}{alpha}\PY{o}{=}\PY{l+m+mf}{0.5}\PY{p}{)}
\PY{n}{plt}\PY{o}{.}\PY{n}{axhline}\PY{p}{(}\PY{l+m+mi}{0}\PY{p}{,} \PY{n}{color}\PY{o}{=}\PY{l+s+s1}{\PYZsq{}}\PY{l+s+s1}{red}\PY{l+s+s1}{\PYZsq{}}\PY{p}{,} \PY{n}{linestyle}\PY{o}{=}\PY{l+s+s1}{\PYZsq{}}\PY{l+s+s1}{\PYZhy{}\PYZhy{}}\PY{l+s+s1}{\PYZsq{}}\PY{p}{)}  \PY{c+c1}{\PYZsh{} Línea en y=0}
\PY{n}{plt}\PY{o}{.}\PY{n}{axhline}\PY{p}{(}\PY{n}{std\PYZus{}residuos}\PY{p}{,} \PY{n}{color}\PY{o}{=}\PY{l+s+s1}{\PYZsq{}}\PY{l+s+s1}{green}\PY{l+s+s1}{\PYZsq{}}\PY{p}{,} \PY{n}{linestyle}\PY{o}{=}\PY{l+s+s1}{\PYZsq{}}\PY{l+s+s1}{\PYZhy{}\PYZhy{}}\PY{l+s+s1}{\PYZsq{}}\PY{p}{)}  \PY{c+c1}{\PYZsh{} Línea superior}
\PY{n}{plt}\PY{o}{.}\PY{n}{axhline}\PY{p}{(}\PY{o}{\PYZhy{}}\PY{n}{std\PYZus{}residuos}\PY{p}{,} \PY{n}{color}\PY{o}{=}\PY{l+s+s1}{\PYZsq{}}\PY{l+s+s1}{green}\PY{l+s+s1}{\PYZsq{}}\PY{p}{,} \PY{n}{linestyle}\PY{o}{=}\PY{l+s+s1}{\PYZsq{}}\PY{l+s+s1}{\PYZhy{}\PYZhy{}}\PY{l+s+s1}{\PYZsq{}}\PY{p}{)}  \PY{c+c1}{\PYZsh{} Línea inferior}
\PY{n}{plt}\PY{o}{.}\PY{n}{title}\PY{p}{(}\PY{l+s+s1}{\PYZsq{}}\PY{l+s+s1}{Gráfico de Residuos vs. Valores Ajustados}\PY{l+s+s1}{\PYZsq{}}\PY{p}{)}
\PY{n}{plt}\PY{o}{.}\PY{n}{xlabel}\PY{p}{(}\PY{l+s+s1}{\PYZsq{}}\PY{l+s+s1}{Valores Ajustados}\PY{l+s+s1}{\PYZsq{}}\PY{p}{)}
\PY{n}{plt}\PY{o}{.}\PY{n}{ylabel}\PY{p}{(}\PY{l+s+s1}{\PYZsq{}}\PY{l+s+s1}{Residuos}\PY{l+s+s1}{\PYZsq{}}\PY{p}{)}
\PY{n}{plt}\PY{o}{.}\PY{n}{show}\PY{p}{(}\PY{p}{)}
\end{Verbatim}
\end{tcolorbox}

    \begin{center}
    \adjustimage{max size={0.9\linewidth}{0.9\paperheight}}{punto_3_files/punto_3_100_0.png}
    \end{center}
    { \hspace*{\fill} \\}
    
    \subparagraph{EL GRÁFICO NOS MUESTRA QUE EFECTIVAMENTE NO HAY GRAN
DISPERCIÓN DE LOS DATOS, POR TANTO TIENEN UNA VARIANZA
CONSTANTE}\label{el-gruxe1fico-nos-muestra-que-efectivamente-no-hay-gran-disperciuxf3n-de-los-datos-por-tanto-tienen-una-varianza-constante}

    \subsection{HACERLE TRANSFORMACIÓN A LAS VARIABLES PARA AJUSTAR LA
GRÁFICA
LINEAL}\label{hacerle-transformaciuxf3n-a-las-variables-para-ajustar-la-gruxe1fica-lineal}

    \subsubsection{Debido a la forma que presentaba la variable Cloruros
frente a Densidad, podemos inferir que tiene un comportamiento
Logarítmico}\label{debido-a-la-forma-que-presentaba-la-variable-cloruros-frente-a-densidad-podemos-inferir-que-tiene-un-comportamiento-logaruxedtmico}

    \begin{tcolorbox}[breakable, size=fbox, boxrule=1pt, pad at break*=1mm,colback=cellbackground, colframe=cellborder]
\prompt{In}{incolor}{63}{\boxspacing}
\begin{Verbatim}[commandchars=\\\{\}]
\PY{n}{x\PYZus{}log} \PY{o}{=} \PY{n}{np}\PY{o}{.}\PY{n}{linspace}\PY{p}{(}\PY{n+nb}{min}\PY{p}{(}\PY{n}{standard\PYZus{}wine}\PY{p}{[}\PY{l+s+s1}{\PYZsq{}}\PY{l+s+s1}{Cloruros}\PY{l+s+s1}{\PYZsq{}}\PY{p}{]}\PY{p}{)}\PY{p}{,} \PY{n+nb}{max}\PY{p}{(}\PY{n}{standard\PYZus{}wine}\PY{p}{[}\PY{l+s+s1}{\PYZsq{}}\PY{l+s+s1}{Cloruros}\PY{l+s+s1}{\PYZsq{}}\PY{p}{]}\PY{p}{)}\PY{p}{,} \PY{l+m+mi}{400}\PY{p}{)}
\PY{n}{y\PYZus{}log} \PY{o}{=} \PY{n}{np}\PY{o}{.}\PY{n}{log10}\PY{p}{(}\PY{n}{x\PYZus{}log}\PY{p}{)}

\PY{c+c1}{\PYZsh{} Añadir la línea logarítmica al gráfico}
\PY{n}{plt}\PY{o}{.}\PY{n}{plot}\PY{p}{(}\PY{n}{x\PYZus{}log}\PY{p}{,} \PY{n}{y\PYZus{}log}\PY{p}{,} \PY{n}{color}\PY{o}{=}\PY{l+s+s1}{\PYZsq{}}\PY{l+s+s1}{red}\PY{l+s+s1}{\PYZsq{}}\PY{p}{,} \PY{n}{label}\PY{o}{=}\PY{l+s+s1}{\PYZsq{}}\PY{l+s+s1}{y = log10(x)}\PY{l+s+s1}{\PYZsq{}}\PY{p}{)}
\PY{n}{plt}\PY{o}{.}\PY{n}{scatter}\PY{p}{(}\PY{n}{standard\PYZus{}wine}\PY{p}{[}\PY{l+s+s1}{\PYZsq{}}\PY{l+s+s1}{Cloruros}\PY{l+s+s1}{\PYZsq{}}\PY{p}{]}\PY{p}{,} \PY{n}{standard\PYZus{}wine}\PY{p}{[}\PY{l+s+s1}{\PYZsq{}}\PY{l+s+s1}{Densidad}\PY{l+s+s1}{\PYZsq{}}\PY{p}{]}\PY{p}{,} \PY{n}{color}\PY{o}{=}\PY{l+s+s1}{\PYZsq{}}\PY{l+s+s1}{blue}\PY{l+s+s1}{\PYZsq{}}\PY{p}{)}
\PY{c+c1}{\PYZsh{} Añadir etiquetas y leyenda}
\PY{n}{plt}\PY{o}{.}\PY{n}{title}\PY{p}{(}\PY{l+s+s1}{\PYZsq{}}\PY{l+s+s1}{Gráfico de dispersión con línea logarítmica}\PY{l+s+s1}{\PYZsq{}}\PY{p}{)}
\PY{n}{plt}\PY{o}{.}\PY{n}{xlabel}\PY{p}{(}\PY{l+s+s1}{\PYZsq{}}\PY{l+s+s1}{x}\PY{l+s+s1}{\PYZsq{}}\PY{p}{)}
\PY{n}{plt}\PY{o}{.}\PY{n}{ylabel}\PY{p}{(}\PY{l+s+s1}{\PYZsq{}}\PY{l+s+s1}{y}\PY{l+s+s1}{\PYZsq{}}\PY{p}{)}
\PY{n}{plt}\PY{o}{.}\PY{n}{legend}\PY{p}{(}\PY{p}{)}
\PY{n}{plt}\PY{o}{.}\PY{n}{show}\PY{p}{(}\PY{p}{)}
\end{Verbatim}
\end{tcolorbox}

    \begin{Verbatim}[commandchars=\\\{\}]
C:\textbackslash{}Users\textbackslash{}aleja\textbackslash{}AppData\textbackslash{}Local\textbackslash{}Temp\textbackslash{}ipykernel\_29848\textbackslash{}2143002310.py:2:
RuntimeWarning: invalid value encountered in log10
  y\_log = np.log10(x\_log)
    \end{Verbatim}

    \begin{center}
    \adjustimage{max size={0.9\linewidth}{0.9\paperheight}}{punto_3_files/punto_3_104_1.png}
    \end{center}
    { \hspace*{\fill} \\}
    
    \paragraph{Hacerle la transformación a los valores
originales}\label{hacerle-la-transformaciuxf3n-a-los-valores-originales}

    \begin{tcolorbox}[breakable, size=fbox, boxrule=1pt, pad at break*=1mm,colback=cellbackground, colframe=cellborder]
\prompt{In}{incolor}{118}{\boxspacing}
\begin{Verbatim}[commandchars=\\\{\}]
\PY{n}{wine}\PY{p}{[}\PY{l+s+s1}{\PYZsq{}}\PY{l+s+s1}{log\PYZus{}Cloruros}\PY{l+s+s1}{\PYZsq{}}\PY{p}{]} \PY{o}{=} \PY{n}{np}\PY{o}{.}\PY{n}{log}\PY{p}{(}\PY{n}{wine}\PY{p}{[}\PY{l+s+s1}{\PYZsq{}}\PY{l+s+s1}{Cloruros}\PY{l+s+s1}{\PYZsq{}}\PY{p}{]}\PY{p}{)}
\end{Verbatim}
\end{tcolorbox}

    \begin{tcolorbox}[breakable, size=fbox, boxrule=1pt, pad at break*=1mm,colback=cellbackground, colframe=cellborder]
\prompt{In}{incolor}{167}{\boxspacing}
\begin{Verbatim}[commandchars=\\\{\}]
\PY{n}{plt}\PY{o}{.}\PY{n}{scatter}\PY{p}{(}\PY{n}{wine}\PY{p}{[}\PY{l+s+s1}{\PYZsq{}}\PY{l+s+s1}{log\PYZus{}Cloruros}\PY{l+s+s1}{\PYZsq{}}\PY{p}{]}\PY{p}{,}\PY{n}{wine}\PY{p}{[}\PY{l+s+s1}{\PYZsq{}}\PY{l+s+s1}{Densidad}\PY{l+s+s1}{\PYZsq{}}\PY{p}{]}\PY{p}{)}
\PY{n}{plt}\PY{o}{.}\PY{n}{ylabel}\PY{p}{(}\PY{l+s+s1}{\PYZsq{}}\PY{l+s+s1}{Densidad}\PY{l+s+s1}{\PYZsq{}}\PY{p}{)}
\PY{n}{plt}\PY{o}{.}\PY{n}{xlabel}\PY{p}{(}\PY{l+s+s1}{\PYZsq{}}\PY{l+s+s1}{log\PYZus{}Cloruros}\PY{l+s+s1}{\PYZsq{}}\PY{p}{)}
\PY{n}{plt}\PY{o}{.}\PY{n}{show}\PY{p}{(}\PY{p}{)}
\end{Verbatim}
\end{tcolorbox}

    \begin{center}
    \adjustimage{max size={0.9\linewidth}{0.9\paperheight}}{punto_3_files/punto_3_107_0.png}
    \end{center}
    { \hspace*{\fill} \\}
    
    \paragraph{Ahora se aplica raíz cuadrada al Dióxido de Azúfre Libre,
para lograr una forma más elíptica, ya que estaba más cercano a una
círculo. La transformación de raíz cuadrada comprime la variabilidad en
los datos, especialmente en los valores más
altos.}\label{ahora-se-aplica-rauxedz-cuadrada-al-diuxf3xido-de-azuxfafre-libre-para-lograr-una-forma-muxe1s-eluxedptica-ya-que-estaba-muxe1s-cercano-a-una-cuxedrculo.-la-transformaciuxf3n-de-rauxedz-cuadrada-comprime-la-variabilidad-en-los-datos-especialmente-en-los-valores-muxe1s-altos.}

    \begin{tcolorbox}[breakable, size=fbox, boxrule=1pt, pad at break*=1mm,colback=cellbackground, colframe=cellborder]
\prompt{In}{incolor}{158}{\boxspacing}
\begin{Verbatim}[commandchars=\\\{\}]
\PY{n}{wine}\PY{p}{[}\PY{l+s+s1}{\PYZsq{}}\PY{l+s+s1}{quad\PYZus{}azufre}\PY{l+s+s1}{\PYZsq{}}\PY{p}{]} \PY{o}{=} \PY{n}{np}\PY{o}{.}\PY{n}{sqrt}\PY{p}{(}\PY{n}{wine}\PY{p}{[}\PY{l+s+s1}{\PYZsq{}}\PY{l+s+s1}{Dióxido de Azúfre Libre}\PY{l+s+s1}{\PYZsq{}}\PY{p}{]}\PY{p}{)}
\end{Verbatim}
\end{tcolorbox}

    \begin{tcolorbox}[breakable, size=fbox, boxrule=1pt, pad at break*=1mm,colback=cellbackground, colframe=cellborder]
\prompt{In}{incolor}{159}{\boxspacing}
\begin{Verbatim}[commandchars=\\\{\}]
\PY{n}{plt}\PY{o}{.}\PY{n}{scatter}\PY{p}{(}\PY{n}{wine}\PY{p}{[}\PY{l+s+s1}{\PYZsq{}}\PY{l+s+s1}{quad\PYZus{}azufre}\PY{l+s+s1}{\PYZsq{}}\PY{p}{]}\PY{p}{,}\PY{n}{wine}\PY{p}{[}\PY{l+s+s1}{\PYZsq{}}\PY{l+s+s1}{Densidad}\PY{l+s+s1}{\PYZsq{}}\PY{p}{]}\PY{p}{)}
\PY{n}{plt}\PY{o}{.}\PY{n}{ylabel}\PY{p}{(}\PY{l+s+s1}{\PYZsq{}}\PY{l+s+s1}{Densidad}\PY{l+s+s1}{\PYZsq{}}\PY{p}{)}
\PY{n}{plt}\PY{o}{.}\PY{n}{xlabel}\PY{p}{(}\PY{l+s+s1}{\PYZsq{}}\PY{l+s+s1}{quad\PYZus{}azufre}\PY{l+s+s1}{\PYZsq{}}\PY{p}{)}
\PY{n}{plt}\PY{o}{.}\PY{n}{show}\PY{p}{(}\PY{p}{)}
\end{Verbatim}
\end{tcolorbox}

    \begin{center}
    \adjustimage{max size={0.9\linewidth}{0.9\paperheight}}{punto_3_files/punto_3_110_0.png}
    \end{center}
    { \hspace*{\fill} \\}
    
    \paragraph{Normalizar Datos, Se hará igualmente con la librería de
Sklearn, pero con otro método de Min y
Max}\label{normalizar-datos-se-haruxe1-igualmente-con-la-libreruxeda-de-sklearn-pero-con-otro-muxe9todo-de-min-y-max}

    \begin{tcolorbox}[breakable, size=fbox, boxrule=1pt, pad at break*=1mm,colback=cellbackground, colframe=cellborder]
\prompt{In}{incolor}{ }{\boxspacing}
\begin{Verbatim}[commandchars=\\\{\}]
\PY{k+kn}{from} \PY{n+nn}{sklearn}\PY{n+nn}{.}\PY{n+nn}{preprocessing} \PY{k+kn}{import} \PY{n}{MinMaxScaler}

\PY{c+c1}{\PYZsh{} Supongamos que wine es tu DataFrame}
\PY{n}{scaler} \PY{o}{=} \PY{n}{MinMaxScaler}\PY{p}{(}\PY{p}{)}

\PY{c+c1}{\PYZsh{} Ajustar y transformar los datos}
\PY{n}{wine\PYZus{}normalized} \PY{o}{=} \PY{n}{pd}\PY{o}{.}\PY{n}{DataFrame}\PY{p}{(}\PY{n}{scaler}\PY{o}{.}\PY{n}{fit\PYZus{}transform}\PY{p}{(}\PY{n}{wine}\PY{p}{)}\PY{p}{,} \PY{n}{columns}\PY{o}{=}\PY{n}{wine}\PY{o}{.}\PY{n}{columns}\PY{p}{,} \PY{n}{index}\PY{o}{=}\PY{n}{wine}\PY{o}{.}\PY{n}{index}\PY{p}{)}
\end{Verbatim}
\end{tcolorbox}

    \paragraph{Eliminemos la columna original de Cloruros y de Dióxido de
Azufre}\label{eliminemos-la-columna-original-de-cloruros-y-de-diuxf3xido-de-azufre}

    \begin{tcolorbox}[breakable, size=fbox, boxrule=1pt, pad at break*=1mm,colback=cellbackground, colframe=cellborder]
\prompt{In}{incolor}{172}{\boxspacing}
\begin{Verbatim}[commandchars=\\\{\}]
\PY{n}{wine\PYZus{}normalized}\PY{o}{.}\PY{n}{drop}\PY{p}{(}\PY{n}{columns}\PY{o}{=}\PY{p}{[}\PY{l+s+s1}{\PYZsq{}}\PY{l+s+s1}{Cloruros}\PY{l+s+s1}{\PYZsq{}}\PY{p}{,}
                              \PY{l+s+s1}{\PYZsq{}}\PY{l+s+s1}{Dióxido de Azúfre Libre}\PY{l+s+s1}{\PYZsq{}}\PY{p}{]}\PY{p}{,} \PY{n}{inplace}\PY{o}{=}\PY{k+kc}{True}\PY{p}{)}
\end{Verbatim}
\end{tcolorbox}

    \begin{tcolorbox}[breakable, size=fbox, boxrule=1pt, pad at break*=1mm,colback=cellbackground, colframe=cellborder]
\prompt{In}{incolor}{173}{\boxspacing}
\begin{Verbatim}[commandchars=\\\{\}]
\PY{n}{sns}\PY{o}{.}\PY{n}{pairplot}\PY{p}{(}\PY{n}{wine\PYZus{}normalized}\PY{p}{)}
\PY{n}{plt}\PY{o}{.}\PY{n}{show}\PY{p}{(}\PY{p}{)}
\end{Verbatim}
\end{tcolorbox}

    \begin{center}
    \adjustimage{max size={0.9\linewidth}{0.9\paperheight}}{punto_3_files/punto_3_115_0.png}
    \end{center}
    { \hspace*{\fill} \\}
    
    \paragraph{REVISAR LAS CORRELACIONES}\label{revisar-las-correlaciones}

    \begin{tcolorbox}[breakable, size=fbox, boxrule=1pt, pad at break*=1mm,colback=cellbackground, colframe=cellborder]
\prompt{In}{incolor}{180}{\boxspacing}
\begin{Verbatim}[commandchars=\\\{\}]
\PY{n}{corr\PYZus{}pearson\PYZus{}new} \PY{o}{=} \PY{n}{wine\PYZus{}normalized}\PY{o}{.}\PY{n}{corr}\PY{p}{(}\PY{n}{method}\PY{o}{=}\PY{l+s+s1}{\PYZsq{}}\PY{l+s+s1}{pearson}\PY{l+s+s1}{\PYZsq{}}\PY{p}{)}\PY{o}{.}\PY{n}{reset\PYZus{}index}\PY{p}{(}\PY{n}{drop}\PY{o}{=}\PY{k+kc}{True}\PY{p}{)}
\PY{n}{corr\PYZus{}pearson\PYZus{}new}
\end{Verbatim}
\end{tcolorbox}

            \begin{tcolorbox}[breakable, size=fbox, boxrule=.5pt, pad at break*=1mm, opacityfill=0]
\prompt{Out}{outcolor}{180}{\boxspacing}
\begin{Verbatim}[commandchars=\\\{\}]
   Ácido Cítrico  Azúcar Residual  Dióxido de Azúfre Total  Densidad  \textbackslash{}
0       1.000000         0.094212                 0.121131  0.149503
1       0.094212         1.000000                 0.401439  0.838966
2       0.121131         0.401439                 1.000000  0.529881
3       0.149503         0.838966                 0.529881  1.000000
4      -0.075729        -0.450631                -0.448892 -0.780138
5       0.101804         0.173478                 0.299686  0.394830
6       0.093811         0.311859                 0.619406  0.298543
7       0.151325         0.867179                 0.559643  0.958305

    Alcohol  log\_Cloruros  quad\_azufre    y\_pred
0 -0.075729      0.101804     0.093811  0.151325
1 -0.450631      0.173478     0.311859  0.867179
2 -0.448892      0.299686     0.619406  0.559643
3 -0.780138      0.394830     0.298543  0.958305
4  1.000000     -0.500900    -0.245194 -0.824481
5 -0.500900      1.000000     0.143889  0.410114
6 -0.245194      0.143889     1.000000  0.320923
7 -0.824481      0.410114     0.320923  1.000000
\end{Verbatim}
\end{tcolorbox}
        
    \subsubsection{Crear un Modelo RLM con los nuevos Datos
Transformados}\label{crear-un-modelo-rlm-con-los-nuevos-datos-transformados}

    \begin{tcolorbox}[breakable, size=fbox, boxrule=1pt, pad at break*=1mm,colback=cellbackground, colframe=cellborder]
\prompt{In}{incolor}{175}{\boxspacing}
\begin{Verbatim}[commandchars=\\\{\}]
\PY{k+kn}{import} \PY{n+nn}{statsmodels}\PY{n+nn}{.}\PY{n+nn}{api} \PY{k}{as} \PY{n+nn}{sm}

\PY{n}{X} \PY{o}{=} \PY{n}{sm}\PY{o}{.}\PY{n}{add\PYZus{}constant}\PY{p}{(}\PY{n}{wine\PYZus{}normalized}\PY{o}{.}\PY{n}{drop}\PY{p}{(}\PY{n}{columns}\PY{o}{=}\PY{l+s+s1}{\PYZsq{}}\PY{l+s+s1}{Densidad}\PY{l+s+s1}{\PYZsq{}}\PY{p}{)}\PY{p}{)}  \PY{c+c1}{\PYZsh{} Esto añade una columna de unos para actuar como intercepto}
\PY{n}{y} \PY{o}{=} \PY{n}{wine\PYZus{}normalized}\PY{p}{[}\PY{l+s+s1}{\PYZsq{}}\PY{l+s+s1}{Densidad}\PY{l+s+s1}{\PYZsq{}}\PY{p}{]}

\PY{c+c1}{\PYZsh{} Crear el modelo RLM}
\PY{n}{model} \PY{o}{=} \PY{n}{sm}\PY{o}{.}\PY{n}{RLM}\PY{p}{(}\PY{n}{y}\PY{p}{,} \PY{n}{X}\PY{p}{,} \PY{n}{M}\PY{o}{=}\PY{n}{sm}\PY{o}{.}\PY{n}{robust}\PY{o}{.}\PY{n}{norms}\PY{o}{.}\PY{n}{HuberT}\PY{p}{(}\PY{p}{)}\PY{p}{)}

\PY{c+c1}{\PYZsh{} Ajustar el modelo}
\PY{n}{results} \PY{o}{=} \PY{n}{model}\PY{o}{.}\PY{n}{fit}\PY{p}{(}\PY{p}{)}

\PY{c+c1}{\PYZsh{} Mostrar los resultados del ajuste}
\PY{n+nb}{print}\PY{p}{(}\PY{n}{results}\PY{o}{.}\PY{n}{summary}\PY{p}{(}\PY{p}{)}\PY{p}{)}
\end{Verbatim}
\end{tcolorbox}

    \begin{Verbatim}[commandchars=\\\{\}]
                    Robust linear Model Regression Results
==============================================================================
Dep. Variable:               Densidad   No. Observations:                 4898
Model:                            RLM   Df Residuals:                     4891
Method:                          IRLS   Df Model:                            6
Norm:                          HuberT
Scale Est.:                       mad
Cov Type:                          H1
Date:                Thu, 18 Apr 2024
Time:                        00:21:00
No. Iterations:                    30
================================================================================
===========
                              coef    std err          z      P>|z|      [0.025
0.975]
--------------------------------------------------------------------------------
-----------
const                       0.1283      0.002     74.529      0.000       0.125
0.132
Ácido Cítrico               0.0341      0.003     12.193      0.000       0.029
0.040
Azúcar Residual             0.4245      0.003    139.893      0.000       0.419
0.430
Dióxido de Azúfre Total     0.0705      0.003     24.457      0.000       0.065
0.076
Alcohol                    -0.1360      0.001   -101.415      0.000      -0.139
-0.133
log\_Cloruros                0.0144      0.003      5.538      0.000       0.009
0.019
quad\_azufre                -0.0457      0.003    -16.229      0.000      -0.051
-0.040
================================================================================
===========

If the model instance has been used for another fit with different fit
parameters, then the fit options might not be the correct ones anymore .
    \end{Verbatim}

    \subsubsection{sm.add\_constant(): Esto añade una columna de constantes
al conjunto de datos de entrada, que es necesario para que el modelo
incluya un término de
intercepto.}\label{sm.add_constant-esto-auxf1ade-una-columna-de-constantes-al-conjunto-de-datos-de-entrada-que-es-necesario-para-que-el-modelo-incluya-un-tuxe9rmino-de-intercepto.}

\subsubsection{sm.RLM(): Este es el constructor para un modelo de
regresión lineal
robusta.}\label{sm.rlm-este-es-el-constructor-para-un-modelo-de-regresiuxf3n-lineal-robusta.}

\subsubsection{El argumento M=sm.robust.norms.HuberT() especifica el
tipo de estimador de robustez que se usa, en este caso usamos Huber T,
que es bueno para manejar
outliers.}\label{el-argumento-msm.robust.norms.hubert-especifica-el-tipo-de-estimador-de-robustez-que-se-usa-en-este-caso-usamos-huber-t-que-es-bueno-para-manejar-outliers.}

    \subsubsection{EXPLICACIÓN DE LAS
ESTADÍSTICAS}\label{explicaciuxf3n-de-las-estaduxedsticas}

    \paragraph{coef: Coeficientes de
regresión:}\label{coef-coeficientes-de-regresiuxf3n}

\paragraph{Representan el cambio esperado en la variable dependiente
(Densidad) por cada unidad de cambio en la variable independiente,
manteniendo constantes todas las demás
variables.}\label{representan-el-cambio-esperado-en-la-variable-dependiente-densidad-por-cada-unidad-de-cambio-en-la-variable-independiente-manteniendo-constantes-todas-las-demuxe1s-variables.}

\paragraph{Recordemos que en este caso la interpretación no puede ser
tan directa, ya que además de haberse normalizado, también se le
hicieron trasnformaciones a 2 variables, entonces el cambio es de la
variable transformada y no de la
original}\label{recordemos-que-en-este-caso-la-interpretaciuxf3n-no-puede-ser-tan-directa-ya-que-ademuxe1s-de-haberse-normalizado-tambiuxe9n-se-le-hicieron-trasnformaciones-a-2-variables-entonces-el-cambio-es-de-la-variable-transformada-y-no-de-la-original}

    \subparagraph{const: Es el Intercepto}\label{const-es-el-intercepto}

    \paragraph{std err: El error estándar de los coeficientes estima la
variabilidad. Un error estándar bajo indica mayor precisión de la
estimación del
coeficiente.}\label{std-err-el-error-estuxe1ndar-de-los-coeficientes-estima-la-variabilidad.-un-error-estuxe1ndar-bajo-indica-mayor-precisiuxf3n-de-la-estimaciuxf3n-del-coeficiente.}

    \paragraph{z: Se calcula dividiendo el coeficiente por su error
estándar. Es una medida de cuántas desviaciones estándar está el
coeficiente estimado de
cero.}\label{z-se-calcula-dividiendo-el-coeficiente-por-su-error-estuxe1ndar.-es-una-medida-de-cuuxe1ntas-desviaciones-estuxe1ndar-estuxe1-el-coeficiente-estimado-de-cero.}

    \paragraph{P\textgreater\textbar z\textbar: Valor p asociado a la prueba
estadística z. Se necesita un valor p bajo (menor a 0.05) para rechazar
la hipótesis nula (No existe relación) de que el coeficiente es igual a
cero, indicando que hay un efecto significativo de la variable sobre la
respuesta.(Significancia
estadística)}\label{pz-valor-p-asociado-a-la-prueba-estaduxedstica-z.-se-necesita-un-valor-p-bajo-menor-a-0.05-para-rechazar-la-hipuxf3tesis-nula-no-existe-relaciuxf3n-de-que-el-coeficiente-es-igual-a-cero-indicando-que-hay-un-efecto-significativo-de-la-variable-sobre-la-respuesta.significancia-estaduxedstica}

    \begin{tcolorbox}[breakable, size=fbox, boxrule=1pt, pad at break*=1mm,colback=cellbackground, colframe=cellborder]
\prompt{In}{incolor}{179}{\boxspacing}
\begin{Verbatim}[commandchars=\\\{\}]
\PY{n}{wine\PYZus{}normalized}\PY{p}{[}\PY{l+s+s1}{\PYZsq{}}\PY{l+s+s1}{y\PYZus{}pred}\PY{l+s+s1}{\PYZsq{}}\PY{p}{]} \PY{o}{=} \PY{n}{results}\PY{o}{.}\PY{n}{predict}\PY{p}{(}\PY{n}{X}\PY{p}{)}

\PY{n}{plt}\PY{o}{.}\PY{n}{figure}\PY{p}{(}\PY{n}{figsize}\PY{o}{=}\PY{p}{(}\PY{l+m+mi}{10}\PY{p}{,} \PY{l+m+mi}{6}\PY{p}{)}\PY{p}{)}
\PY{n}{plt}\PY{o}{.}\PY{n}{scatter}\PY{p}{(}\PY{n}{wine\PYZus{}normalized}\PY{p}{[}\PY{l+s+s1}{\PYZsq{}}\PY{l+s+s1}{y\PYZus{}pred}\PY{l+s+s1}{\PYZsq{}}\PY{p}{]}\PY{p}{,} \PY{n}{wine\PYZus{}normalized}\PY{p}{[}\PY{l+s+s1}{\PYZsq{}}\PY{l+s+s1}{Densidad}\PY{l+s+s1}{\PYZsq{}}\PY{p}{]}\PY{p}{,} \PY{n}{alpha}\PY{o}{=}\PY{l+m+mf}{0.5}\PY{p}{,} \PY{n}{color}\PY{o}{=}\PY{l+s+s1}{\PYZsq{}}\PY{l+s+s1}{deepskyblue}\PY{l+s+s1}{\PYZsq{}}\PY{p}{,} \PY{n}{label}\PY{o}{=}\PY{l+s+s1}{\PYZsq{}}\PY{l+s+s1}{Observado vs. Predicho}\PY{l+s+s1}{\PYZsq{}}\PY{p}{)}  \PY{c+c1}{\PYZsh{} Cambiado a deepskyblue}
\PY{n}{plt}\PY{o}{.}\PY{n}{plot}\PY{p}{(}\PY{p}{[}\PY{n}{y}\PY{o}{.}\PY{n}{min}\PY{p}{(}\PY{p}{)}\PY{p}{,} \PY{n}{y}\PY{o}{.}\PY{n}{max}\PY{p}{(}\PY{p}{)}\PY{p}{]}\PY{p}{,} \PY{p}{[}\PY{n}{y}\PY{o}{.}\PY{n}{min}\PY{p}{(}\PY{p}{)}\PY{p}{,} \PY{n}{y}\PY{o}{.}\PY{n}{max}\PY{p}{(}\PY{p}{)}\PY{p}{]}\PY{p}{,} \PY{l+s+s1}{\PYZsq{}}\PY{l+s+s1}{k\PYZhy{}\PYZhy{}}\PY{l+s+s1}{\PYZsq{}}\PY{p}{,} \PY{n}{lw}\PY{o}{=}\PY{l+m+mi}{2}\PY{p}{,} \PY{n}{label}\PY{o}{=}\PY{l+s+s1}{\PYZsq{}}\PY{l+s+s1}{Línea de Predicción Perfecta}\PY{l+s+s1}{\PYZsq{}}\PY{p}{)}  \PY{c+c1}{\PYZsh{} Cambiado a negro con estilo dashed}
\PY{n}{plt}\PY{o}{.}\PY{n}{title}\PY{p}{(}\PY{l+s+s1}{\PYZsq{}}\PY{l+s+s1}{Comparación de Densidad Observada vs. Predicha}\PY{l+s+s1}{\PYZsq{}}\PY{p}{)}
\PY{n}{plt}\PY{o}{.}\PY{n}{xlabel}\PY{p}{(}\PY{l+s+s1}{\PYZsq{}}\PY{l+s+s1}{Valores Predichos de Densidad}\PY{l+s+s1}{\PYZsq{}}\PY{p}{)}
\PY{n}{plt}\PY{o}{.}\PY{n}{ylabel}\PY{p}{(}\PY{l+s+s1}{\PYZsq{}}\PY{l+s+s1}{Valores Observados de Densidad}\PY{l+s+s1}{\PYZsq{}}\PY{p}{)}
\PY{n}{plt}\PY{o}{.}\PY{n}{legend}\PY{p}{(}\PY{p}{)}
\PY{n}{plt}\PY{o}{.}\PY{n}{show}\PY{p}{(}\PY{p}{)}
\end{Verbatim}
\end{tcolorbox}

    \begin{center}
    \adjustimage{max size={0.9\linewidth}{0.9\paperheight}}{punto_3_files/punto_3_127_0.png}
    \end{center}
    { \hspace*{\fill} \\}
    

    % Add a bibliography block to the postdoc
    
    
    
\end{document}
