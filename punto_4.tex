\documentclass[11pt]{article}

    \usepackage[breakable]{tcolorbox}
    \usepackage{parskip} % Stop auto-indenting (to mimic markdown behaviour)
    

    % Basic figure setup, for now with no caption control since it's done
    % automatically by Pandoc (which extracts ![](path) syntax from Markdown).
    \usepackage{graphicx}
    % Maintain compatibility with old templates. Remove in nbconvert 6.0
    \let\Oldincludegraphics\includegraphics
    % Ensure that by default, figures have no caption (until we provide a
    % proper Figure object with a Caption API and a way to capture that
    % in the conversion process - todo).
    \usepackage{caption}
    \DeclareCaptionFormat{nocaption}{}
    \captionsetup{format=nocaption,aboveskip=0pt,belowskip=0pt}

    \usepackage{float}
    \floatplacement{figure}{H} % forces figures to be placed at the correct location
    \usepackage{xcolor} % Allow colors to be defined
    \usepackage{enumerate} % Needed for markdown enumerations to work
    \usepackage{geometry} % Used to adjust the document margins
    \usepackage{amsmath} % Equations
    \usepackage{amssymb} % Equations
    \usepackage{textcomp} % defines textquotesingle
    % Hack from http://tex.stackexchange.com/a/47451/13684:
    \AtBeginDocument{%
        \def\PYZsq{\textquotesingle}% Upright quotes in Pygmentized code
    }
    \usepackage{upquote} % Upright quotes for verbatim code
    \usepackage{eurosym} % defines \euro

    \usepackage{iftex}
    \ifPDFTeX
        \usepackage[T1]{fontenc}
        \IfFileExists{alphabeta.sty}{
              \usepackage{alphabeta}
          }{
              \usepackage[mathletters]{ucs}
              \usepackage[utf8x]{inputenc}
          }
    \else
        \usepackage{fontspec}
        \usepackage{unicode-math}
    \fi

    \usepackage{fancyvrb} % verbatim replacement that allows latex
    \usepackage{grffile} % extends the file name processing of package graphics
                         % to support a larger range
    \makeatletter % fix for old versions of grffile with XeLaTeX
    \@ifpackagelater{grffile}{2019/11/01}
    {
      % Do nothing on new versions
    }
    {
      \def\Gread@@xetex#1{%
        \IfFileExists{"\Gin@base".bb}%
        {\Gread@eps{\Gin@base.bb}}%
        {\Gread@@xetex@aux#1}%
      }
    }
    \makeatother
    \usepackage[Export]{adjustbox} % Used to constrain images to a maximum size
    \adjustboxset{max size={0.9\linewidth}{0.9\paperheight}}

    % The hyperref package gives us a pdf with properly built
    % internal navigation ('pdf bookmarks' for the table of contents,
    % internal cross-reference links, web links for URLs, etc.)
    \usepackage{hyperref}
    % The default LaTeX title has an obnoxious amount of whitespace. By default,
    % titling removes some of it. It also provides customization options.
    \usepackage{titling}
    \usepackage{longtable} % longtable support required by pandoc >1.10
    \usepackage{booktabs}  % table support for pandoc > 1.12.2
    \usepackage{array}     % table support for pandoc >= 2.11.3
    \usepackage{calc}      % table minipage width calculation for pandoc >= 2.11.1
    \usepackage[inline]{enumitem} % IRkernel/repr support (it uses the enumerate* environment)
    \usepackage[normalem]{ulem} % ulem is needed to support strikethroughs (\sout)
                                % normalem makes italics be italics, not underlines
    \usepackage{soul}      % strikethrough (\st) support for pandoc >= 3.0.0
    \usepackage{mathrsfs}
    

    
    % Colors for the hyperref package
    \definecolor{urlcolor}{rgb}{0,.145,.698}
    \definecolor{linkcolor}{rgb}{.71,0.21,0.01}
    \definecolor{citecolor}{rgb}{.12,.54,.11}

    % ANSI colors
    \definecolor{ansi-black}{HTML}{3E424D}
    \definecolor{ansi-black-intense}{HTML}{282C36}
    \definecolor{ansi-red}{HTML}{E75C58}
    \definecolor{ansi-red-intense}{HTML}{B22B31}
    \definecolor{ansi-green}{HTML}{00A250}
    \definecolor{ansi-green-intense}{HTML}{007427}
    \definecolor{ansi-yellow}{HTML}{DDB62B}
    \definecolor{ansi-yellow-intense}{HTML}{B27D12}
    \definecolor{ansi-blue}{HTML}{208FFB}
    \definecolor{ansi-blue-intense}{HTML}{0065CA}
    \definecolor{ansi-magenta}{HTML}{D160C4}
    \definecolor{ansi-magenta-intense}{HTML}{A03196}
    \definecolor{ansi-cyan}{HTML}{60C6C8}
    \definecolor{ansi-cyan-intense}{HTML}{258F8F}
    \definecolor{ansi-white}{HTML}{C5C1B4}
    \definecolor{ansi-white-intense}{HTML}{A1A6B2}
    \definecolor{ansi-default-inverse-fg}{HTML}{FFFFFF}
    \definecolor{ansi-default-inverse-bg}{HTML}{000000}

    % common color for the border for error outputs.
    \definecolor{outerrorbackground}{HTML}{FFDFDF}

    % commands and environments needed by pandoc snippets
    % extracted from the output of `pandoc -s`
    \providecommand{\tightlist}{%
      \setlength{\itemsep}{0pt}\setlength{\parskip}{0pt}}
    \DefineVerbatimEnvironment{Highlighting}{Verbatim}{commandchars=\\\{\}}
    % Add ',fontsize=\small' for more characters per line
    \newenvironment{Shaded}{}{}
    \newcommand{\KeywordTok}[1]{\textcolor[rgb]{0.00,0.44,0.13}{\textbf{{#1}}}}
    \newcommand{\DataTypeTok}[1]{\textcolor[rgb]{0.56,0.13,0.00}{{#1}}}
    \newcommand{\DecValTok}[1]{\textcolor[rgb]{0.25,0.63,0.44}{{#1}}}
    \newcommand{\BaseNTok}[1]{\textcolor[rgb]{0.25,0.63,0.44}{{#1}}}
    \newcommand{\FloatTok}[1]{\textcolor[rgb]{0.25,0.63,0.44}{{#1}}}
    \newcommand{\CharTok}[1]{\textcolor[rgb]{0.25,0.44,0.63}{{#1}}}
    \newcommand{\StringTok}[1]{\textcolor[rgb]{0.25,0.44,0.63}{{#1}}}
    \newcommand{\CommentTok}[1]{\textcolor[rgb]{0.38,0.63,0.69}{\textit{{#1}}}}
    \newcommand{\OtherTok}[1]{\textcolor[rgb]{0.00,0.44,0.13}{{#1}}}
    \newcommand{\AlertTok}[1]{\textcolor[rgb]{1.00,0.00,0.00}{\textbf{{#1}}}}
    \newcommand{\FunctionTok}[1]{\textcolor[rgb]{0.02,0.16,0.49}{{#1}}}
    \newcommand{\RegionMarkerTok}[1]{{#1}}
    \newcommand{\ErrorTok}[1]{\textcolor[rgb]{1.00,0.00,0.00}{\textbf{{#1}}}}
    \newcommand{\NormalTok}[1]{{#1}}

    % Additional commands for more recent versions of Pandoc
    \newcommand{\ConstantTok}[1]{\textcolor[rgb]{0.53,0.00,0.00}{{#1}}}
    \newcommand{\SpecialCharTok}[1]{\textcolor[rgb]{0.25,0.44,0.63}{{#1}}}
    \newcommand{\VerbatimStringTok}[1]{\textcolor[rgb]{0.25,0.44,0.63}{{#1}}}
    \newcommand{\SpecialStringTok}[1]{\textcolor[rgb]{0.73,0.40,0.53}{{#1}}}
    \newcommand{\ImportTok}[1]{{#1}}
    \newcommand{\DocumentationTok}[1]{\textcolor[rgb]{0.73,0.13,0.13}{\textit{{#1}}}}
    \newcommand{\AnnotationTok}[1]{\textcolor[rgb]{0.38,0.63,0.69}{\textbf{\textit{{#1}}}}}
    \newcommand{\CommentVarTok}[1]{\textcolor[rgb]{0.38,0.63,0.69}{\textbf{\textit{{#1}}}}}
    \newcommand{\VariableTok}[1]{\textcolor[rgb]{0.10,0.09,0.49}{{#1}}}
    \newcommand{\ControlFlowTok}[1]{\textcolor[rgb]{0.00,0.44,0.13}{\textbf{{#1}}}}
    \newcommand{\OperatorTok}[1]{\textcolor[rgb]{0.40,0.40,0.40}{{#1}}}
    \newcommand{\BuiltInTok}[1]{{#1}}
    \newcommand{\ExtensionTok}[1]{{#1}}
    \newcommand{\PreprocessorTok}[1]{\textcolor[rgb]{0.74,0.48,0.00}{{#1}}}
    \newcommand{\AttributeTok}[1]{\textcolor[rgb]{0.49,0.56,0.16}{{#1}}}
    \newcommand{\InformationTok}[1]{\textcolor[rgb]{0.38,0.63,0.69}{\textbf{\textit{{#1}}}}}
    \newcommand{\WarningTok}[1]{\textcolor[rgb]{0.38,0.63,0.69}{\textbf{\textit{{#1}}}}}


    % Define a nice break command that doesn't care if a line doesn't already
    % exist.
    \def\br{\hspace*{\fill} \\* }
    % Math Jax compatibility definitions
    \def\gt{>}
    \def\lt{<}
    \let\Oldtex\TeX
    \let\Oldlatex\LaTeX
    \renewcommand{\TeX}{\textrm{\Oldtex}}
    \renewcommand{\LaTeX}{\textrm{\Oldlatex}}
    % Document parameters
    % Document title
    \title{}
    \author{}
    \date{}
    
    
    
    
    
    
    
% Pygments definitions
\makeatletter
\def\PY@reset{\let\PY@it=\relax \let\PY@bf=\relax%
    \let\PY@ul=\relax \let\PY@tc=\relax%
    \let\PY@bc=\relax \let\PY@ff=\relax}
\def\PY@tok#1{\csname PY@tok@#1\endcsname}
\def\PY@toks#1+{\ifx\relax#1\empty\else%
    \PY@tok{#1}\expandafter\PY@toks\fi}
\def\PY@do#1{\PY@bc{\PY@tc{\PY@ul{%
    \PY@it{\PY@bf{\PY@ff{#1}}}}}}}
\def\PY#1#2{\PY@reset\PY@toks#1+\relax+\PY@do{#2}}

\@namedef{PY@tok@w}{\def\PY@tc##1{\textcolor[rgb]{0.73,0.73,0.73}{##1}}}
\@namedef{PY@tok@c}{\let\PY@it=\textit\def\PY@tc##1{\textcolor[rgb]{0.24,0.48,0.48}{##1}}}
\@namedef{PY@tok@cp}{\def\PY@tc##1{\textcolor[rgb]{0.61,0.40,0.00}{##1}}}
\@namedef{PY@tok@k}{\let\PY@bf=\textbf\def\PY@tc##1{\textcolor[rgb]{0.00,0.50,0.00}{##1}}}
\@namedef{PY@tok@kp}{\def\PY@tc##1{\textcolor[rgb]{0.00,0.50,0.00}{##1}}}
\@namedef{PY@tok@kt}{\def\PY@tc##1{\textcolor[rgb]{0.69,0.00,0.25}{##1}}}
\@namedef{PY@tok@o}{\def\PY@tc##1{\textcolor[rgb]{0.40,0.40,0.40}{##1}}}
\@namedef{PY@tok@ow}{\let\PY@bf=\textbf\def\PY@tc##1{\textcolor[rgb]{0.67,0.13,1.00}{##1}}}
\@namedef{PY@tok@nb}{\def\PY@tc##1{\textcolor[rgb]{0.00,0.50,0.00}{##1}}}
\@namedef{PY@tok@nf}{\def\PY@tc##1{\textcolor[rgb]{0.00,0.00,1.00}{##1}}}
\@namedef{PY@tok@nc}{\let\PY@bf=\textbf\def\PY@tc##1{\textcolor[rgb]{0.00,0.00,1.00}{##1}}}
\@namedef{PY@tok@nn}{\let\PY@bf=\textbf\def\PY@tc##1{\textcolor[rgb]{0.00,0.00,1.00}{##1}}}
\@namedef{PY@tok@ne}{\let\PY@bf=\textbf\def\PY@tc##1{\textcolor[rgb]{0.80,0.25,0.22}{##1}}}
\@namedef{PY@tok@nv}{\def\PY@tc##1{\textcolor[rgb]{0.10,0.09,0.49}{##1}}}
\@namedef{PY@tok@no}{\def\PY@tc##1{\textcolor[rgb]{0.53,0.00,0.00}{##1}}}
\@namedef{PY@tok@nl}{\def\PY@tc##1{\textcolor[rgb]{0.46,0.46,0.00}{##1}}}
\@namedef{PY@tok@ni}{\let\PY@bf=\textbf\def\PY@tc##1{\textcolor[rgb]{0.44,0.44,0.44}{##1}}}
\@namedef{PY@tok@na}{\def\PY@tc##1{\textcolor[rgb]{0.41,0.47,0.13}{##1}}}
\@namedef{PY@tok@nt}{\let\PY@bf=\textbf\def\PY@tc##1{\textcolor[rgb]{0.00,0.50,0.00}{##1}}}
\@namedef{PY@tok@nd}{\def\PY@tc##1{\textcolor[rgb]{0.67,0.13,1.00}{##1}}}
\@namedef{PY@tok@s}{\def\PY@tc##1{\textcolor[rgb]{0.73,0.13,0.13}{##1}}}
\@namedef{PY@tok@sd}{\let\PY@it=\textit\def\PY@tc##1{\textcolor[rgb]{0.73,0.13,0.13}{##1}}}
\@namedef{PY@tok@si}{\let\PY@bf=\textbf\def\PY@tc##1{\textcolor[rgb]{0.64,0.35,0.47}{##1}}}
\@namedef{PY@tok@se}{\let\PY@bf=\textbf\def\PY@tc##1{\textcolor[rgb]{0.67,0.36,0.12}{##1}}}
\@namedef{PY@tok@sr}{\def\PY@tc##1{\textcolor[rgb]{0.64,0.35,0.47}{##1}}}
\@namedef{PY@tok@ss}{\def\PY@tc##1{\textcolor[rgb]{0.10,0.09,0.49}{##1}}}
\@namedef{PY@tok@sx}{\def\PY@tc##1{\textcolor[rgb]{0.00,0.50,0.00}{##1}}}
\@namedef{PY@tok@m}{\def\PY@tc##1{\textcolor[rgb]{0.40,0.40,0.40}{##1}}}
\@namedef{PY@tok@gh}{\let\PY@bf=\textbf\def\PY@tc##1{\textcolor[rgb]{0.00,0.00,0.50}{##1}}}
\@namedef{PY@tok@gu}{\let\PY@bf=\textbf\def\PY@tc##1{\textcolor[rgb]{0.50,0.00,0.50}{##1}}}
\@namedef{PY@tok@gd}{\def\PY@tc##1{\textcolor[rgb]{0.63,0.00,0.00}{##1}}}
\@namedef{PY@tok@gi}{\def\PY@tc##1{\textcolor[rgb]{0.00,0.52,0.00}{##1}}}
\@namedef{PY@tok@gr}{\def\PY@tc##1{\textcolor[rgb]{0.89,0.00,0.00}{##1}}}
\@namedef{PY@tok@ge}{\let\PY@it=\textit}
\@namedef{PY@tok@gs}{\let\PY@bf=\textbf}
\@namedef{PY@tok@gp}{\let\PY@bf=\textbf\def\PY@tc##1{\textcolor[rgb]{0.00,0.00,0.50}{##1}}}
\@namedef{PY@tok@go}{\def\PY@tc##1{\textcolor[rgb]{0.44,0.44,0.44}{##1}}}
\@namedef{PY@tok@gt}{\def\PY@tc##1{\textcolor[rgb]{0.00,0.27,0.87}{##1}}}
\@namedef{PY@tok@err}{\def\PY@bc##1{{\setlength{\fboxsep}{\string -\fboxrule}\fcolorbox[rgb]{1.00,0.00,0.00}{1,1,1}{\strut ##1}}}}
\@namedef{PY@tok@kc}{\let\PY@bf=\textbf\def\PY@tc##1{\textcolor[rgb]{0.00,0.50,0.00}{##1}}}
\@namedef{PY@tok@kd}{\let\PY@bf=\textbf\def\PY@tc##1{\textcolor[rgb]{0.00,0.50,0.00}{##1}}}
\@namedef{PY@tok@kn}{\let\PY@bf=\textbf\def\PY@tc##1{\textcolor[rgb]{0.00,0.50,0.00}{##1}}}
\@namedef{PY@tok@kr}{\let\PY@bf=\textbf\def\PY@tc##1{\textcolor[rgb]{0.00,0.50,0.00}{##1}}}
\@namedef{PY@tok@bp}{\def\PY@tc##1{\textcolor[rgb]{0.00,0.50,0.00}{##1}}}
\@namedef{PY@tok@fm}{\def\PY@tc##1{\textcolor[rgb]{0.00,0.00,1.00}{##1}}}
\@namedef{PY@tok@vc}{\def\PY@tc##1{\textcolor[rgb]{0.10,0.09,0.49}{##1}}}
\@namedef{PY@tok@vg}{\def\PY@tc##1{\textcolor[rgb]{0.10,0.09,0.49}{##1}}}
\@namedef{PY@tok@vi}{\def\PY@tc##1{\textcolor[rgb]{0.10,0.09,0.49}{##1}}}
\@namedef{PY@tok@vm}{\def\PY@tc##1{\textcolor[rgb]{0.10,0.09,0.49}{##1}}}
\@namedef{PY@tok@sa}{\def\PY@tc##1{\textcolor[rgb]{0.73,0.13,0.13}{##1}}}
\@namedef{PY@tok@sb}{\def\PY@tc##1{\textcolor[rgb]{0.73,0.13,0.13}{##1}}}
\@namedef{PY@tok@sc}{\def\PY@tc##1{\textcolor[rgb]{0.73,0.13,0.13}{##1}}}
\@namedef{PY@tok@dl}{\def\PY@tc##1{\textcolor[rgb]{0.73,0.13,0.13}{##1}}}
\@namedef{PY@tok@s2}{\def\PY@tc##1{\textcolor[rgb]{0.73,0.13,0.13}{##1}}}
\@namedef{PY@tok@sh}{\def\PY@tc##1{\textcolor[rgb]{0.73,0.13,0.13}{##1}}}
\@namedef{PY@tok@s1}{\def\PY@tc##1{\textcolor[rgb]{0.73,0.13,0.13}{##1}}}
\@namedef{PY@tok@mb}{\def\PY@tc##1{\textcolor[rgb]{0.40,0.40,0.40}{##1}}}
\@namedef{PY@tok@mf}{\def\PY@tc##1{\textcolor[rgb]{0.40,0.40,0.40}{##1}}}
\@namedef{PY@tok@mh}{\def\PY@tc##1{\textcolor[rgb]{0.40,0.40,0.40}{##1}}}
\@namedef{PY@tok@mi}{\def\PY@tc##1{\textcolor[rgb]{0.40,0.40,0.40}{##1}}}
\@namedef{PY@tok@il}{\def\PY@tc##1{\textcolor[rgb]{0.40,0.40,0.40}{##1}}}
\@namedef{PY@tok@mo}{\def\PY@tc##1{\textcolor[rgb]{0.40,0.40,0.40}{##1}}}
\@namedef{PY@tok@ch}{\let\PY@it=\textit\def\PY@tc##1{\textcolor[rgb]{0.24,0.48,0.48}{##1}}}
\@namedef{PY@tok@cm}{\let\PY@it=\textit\def\PY@tc##1{\textcolor[rgb]{0.24,0.48,0.48}{##1}}}
\@namedef{PY@tok@cpf}{\let\PY@it=\textit\def\PY@tc##1{\textcolor[rgb]{0.24,0.48,0.48}{##1}}}
\@namedef{PY@tok@c1}{\let\PY@it=\textit\def\PY@tc##1{\textcolor[rgb]{0.24,0.48,0.48}{##1}}}
\@namedef{PY@tok@cs}{\let\PY@it=\textit\def\PY@tc##1{\textcolor[rgb]{0.24,0.48,0.48}{##1}}}

\def\PYZbs{\char`\\}
\def\PYZus{\char`\_}
\def\PYZob{\char`\{}
\def\PYZcb{\char`\}}
\def\PYZca{\char`\^}
\def\PYZam{\char`\&}
\def\PYZlt{\char`\<}
\def\PYZgt{\char`\>}
\def\PYZsh{\char`\#}
\def\PYZpc{\char`\%}
\def\PYZdl{\char`\$}
\def\PYZhy{\char`\-}
\def\PYZsq{\char`\'}
\def\PYZdq{\char`\"}
\def\PYZti{\char`\~}
% for compatibility with earlier versions
\def\PYZat{@}
\def\PYZlb{[}
\def\PYZrb{]}
\makeatother


    % For linebreaks inside Verbatim environment from package fancyvrb.
    \makeatletter
        \newbox\Wrappedcontinuationbox
        \newbox\Wrappedvisiblespacebox
        \newcommand*\Wrappedvisiblespace {\textcolor{red}{\textvisiblespace}}
        \newcommand*\Wrappedcontinuationsymbol {\textcolor{red}{\llap{\tiny$\m@th\hookrightarrow$}}}
        \newcommand*\Wrappedcontinuationindent {3ex }
        \newcommand*\Wrappedafterbreak {\kern\Wrappedcontinuationindent\copy\Wrappedcontinuationbox}
        % Take advantage of the already applied Pygments mark-up to insert
        % potential linebreaks for TeX processing.
        %        {, <, #, %, $, ' and ": go to next line.
        %        _, }, ^, &, >, - and ~: stay at end of broken line.
        % Use of \textquotesingle for straight quote.
        \newcommand*\Wrappedbreaksatspecials {%
            \def\PYGZus{\discretionary{\char`\_}{\Wrappedafterbreak}{\char`\_}}%
            \def\PYGZob{\discretionary{}{\Wrappedafterbreak\char`\{}{\char`\{}}%
            \def\PYGZcb{\discretionary{\char`\}}{\Wrappedafterbreak}{\char`\}}}%
            \def\PYGZca{\discretionary{\char`\^}{\Wrappedafterbreak}{\char`\^}}%
            \def\PYGZam{\discretionary{\char`\&}{\Wrappedafterbreak}{\char`\&}}%
            \def\PYGZlt{\discretionary{}{\Wrappedafterbreak\char`\<}{\char`\<}}%
            \def\PYGZgt{\discretionary{\char`\>}{\Wrappedafterbreak}{\char`\>}}%
            \def\PYGZsh{\discretionary{}{\Wrappedafterbreak\char`\#}{\char`\#}}%
            \def\PYGZpc{\discretionary{}{\Wrappedafterbreak\char`\%}{\char`\%}}%
            \def\PYGZdl{\discretionary{}{\Wrappedafterbreak\char`\$}{\char`\$}}%
            \def\PYGZhy{\discretionary{\char`\-}{\Wrappedafterbreak}{\char`\-}}%
            \def\PYGZsq{\discretionary{}{\Wrappedafterbreak\textquotesingle}{\textquotesingle}}%
            \def\PYGZdq{\discretionary{}{\Wrappedafterbreak\char`\"}{\char`\"}}%
            \def\PYGZti{\discretionary{\char`\~}{\Wrappedafterbreak}{\char`\~}}%
        }
        % Some characters . , ; ? ! / are not pygmentized.
        % This macro makes them "active" and they will insert potential linebreaks
        \newcommand*\Wrappedbreaksatpunct {%
            \lccode`\~`\.\lowercase{\def~}{\discretionary{\hbox{\char`\.}}{\Wrappedafterbreak}{\hbox{\char`\.}}}%
            \lccode`\~`\,\lowercase{\def~}{\discretionary{\hbox{\char`\,}}{\Wrappedafterbreak}{\hbox{\char`\,}}}%
            \lccode`\~`\;\lowercase{\def~}{\discretionary{\hbox{\char`\;}}{\Wrappedafterbreak}{\hbox{\char`\;}}}%
            \lccode`\~`\:\lowercase{\def~}{\discretionary{\hbox{\char`\:}}{\Wrappedafterbreak}{\hbox{\char`\:}}}%
            \lccode`\~`\?\lowercase{\def~}{\discretionary{\hbox{\char`\?}}{\Wrappedafterbreak}{\hbox{\char`\?}}}%
            \lccode`\~`\!\lowercase{\def~}{\discretionary{\hbox{\char`\!}}{\Wrappedafterbreak}{\hbox{\char`\!}}}%
            \lccode`\~`\/\lowercase{\def~}{\discretionary{\hbox{\char`\/}}{\Wrappedafterbreak}{\hbox{\char`\/}}}%
            \catcode`\.\active
            \catcode`\,\active
            \catcode`\;\active
            \catcode`\:\active
            \catcode`\?\active
            \catcode`\!\active
            \catcode`\/\active
            \lccode`\~`\~
        }
    \makeatother

    \let\OriginalVerbatim=\Verbatim
    \makeatletter
    \renewcommand{\Verbatim}[1][1]{%
        %\parskip\z@skip
        \sbox\Wrappedcontinuationbox {\Wrappedcontinuationsymbol}%
        \sbox\Wrappedvisiblespacebox {\FV@SetupFont\Wrappedvisiblespace}%
        \def\FancyVerbFormatLine ##1{\hsize\linewidth
            \vtop{\raggedright\hyphenpenalty\z@\exhyphenpenalty\z@
                \doublehyphendemerits\z@\finalhyphendemerits\z@
                \strut ##1\strut}%
        }%
        % If the linebreak is at a space, the latter will be displayed as visible
        % space at end of first line, and a continuation symbol starts next line.
        % Stretch/shrink are however usually zero for typewriter font.
        \def\FV@Space {%
            \nobreak\hskip\z@ plus\fontdimen3\font minus\fontdimen4\font
            \discretionary{\copy\Wrappedvisiblespacebox}{\Wrappedafterbreak}
            {\kern\fontdimen2\font}%
        }%

        % Allow breaks at special characters using \PYG... macros.
        \Wrappedbreaksatspecials
        % Breaks at punctuation characters . , ; ? ! and / need catcode=\active
        \OriginalVerbatim[#1,codes*=\Wrappedbreaksatpunct]%
    }
    \makeatother

    % Exact colors from NB
    \definecolor{incolor}{HTML}{303F9F}
    \definecolor{outcolor}{HTML}{D84315}
    \definecolor{cellborder}{HTML}{CFCFCF}
    \definecolor{cellbackground}{HTML}{F7F7F7}

    % prompt
    \makeatletter
    \newcommand{\boxspacing}{\kern\kvtcb@left@rule\kern\kvtcb@boxsep}
    \makeatother
    \newcommand{\prompt}[4]{
        {\ttfamily\llap{{\color{#2}[#3]:\hspace{3pt}#4}}\vspace{-\baselineskip}}
    }
    

    
    % Prevent overflowing lines due to hard-to-break entities
    \sloppy
    % Setup hyperref package
    \hypersetup{
      breaklinks=true,  % so long urls are correctly broken across lines
      colorlinks=true,
      urlcolor=urlcolor,
      linkcolor=linkcolor,
      citecolor=citecolor,
      }
    % Slightly bigger margins than the latex defaults
    
    \geometry{verbose,tmargin=1in,bmargin=1in,lmargin=1in,rmargin=1in}
    
    

\begin{document}
    
    \maketitle
    
    

    
    \section{Punto 4}\label{punto-4---juan-camilo-vergara}

Se tiene un conjunto de datos que registra la cantidad de anuncios
publicitarios en redes sociales que realiza una empresa y su
correspondiente retorno de inversión en ventas. Se desea determinar si
existe una relación lineal significativa entre la cantidad de anuncios
publicitarios y el retorno de inversión.

\begin{itemize}
\item
  El conjunto de datos ``publicidad.csv'' consta de 200 observaciones y
  4 variables que representan los gastos en publicidad (en miles de
  dólares) y las ventas (en miles de unidades) de un producto en un
  mercado específico: - TV: Gasto en publicidad en televisión. - Radio:
  Gasto en publicidad en radio. - Newspaper: Gasto en publicidad en
  periódicos. - Sales: Número de unidades vendidas (en miles)
\item
  Graficar el retorno de inversión (variable ``Sales'') vs la cantidad
  de anuncios publicitarios por canal (``TV'', ``Radio'',
  ``Newspaper''). Para ello use la función scatter\_matrix() del paquete
  pandas e interprete los graficos de las variables dos a dos, teniendo
  en cuenta que nuestra variable respuesta es ``Sales''.
\item
  Calcular el coeficiente de correlación entre todas las variables y
  mediante un mapa de calor represente estas correlaciones. ¿Interprete
  las estructuras de dependencia encontradas?
\item
  Teniedo en cuenta el punto anterior, elija solo una variable
  explicativa (``TV'', ``Radio'', o ``Newspaper''; la más conveniente)
  para modelar las ventas (``Sales''), ajuste el modelo de regresión
  lineal simple y encuentra la ecuación de la recta. ¿Cuál es el valor
  del coeficiente de determinación R2? ¿Cómo se interpreta este valor?
\item
  Realiza una predicción del retorno de inversión esperado cuando se
  realizan 5 anuncios por el canal de la variable escogida en el ítem
  anterior. ¿Cuál es el intervalo de confianza del 95 \% para la
  predicción?
\end{itemize}

    \begin{tcolorbox}[breakable, size=fbox, boxrule=1pt, pad at break*=1mm,colback=cellbackground, colframe=cellborder]
\prompt{In}{incolor}{1}{\boxspacing}
\begin{Verbatim}[commandchars=\\\{\}]
\PY{k+kn}{import} \PY{n+nn}{pandas} \PY{k}{as} \PY{n+nn}{pd}   \PY{c+c1}{\PYZsh{}\PYZsh{} Libreria para manipular y analizar datos}
\PY{k+kn}{import} \PY{n+nn}{matplotlib}\PY{n+nn}{.}\PY{n+nn}{pyplot} \PY{k}{as} \PY{n+nn}{plt}  \PY{c+c1}{\PYZsh{}\PYZsh{} Libreria para visualizar}
\PY{k+kn}{import} \PY{n+nn}{seaborn} \PY{k}{as} \PY{n+nn}{sns}   \PY{c+c1}{\PYZsh{}\PYZsh{} Libreria para visualizar}
\PY{k+kn}{import} \PY{n+nn}{numpy} \PY{k}{as} \PY{n+nn}{np}  \PY{c+c1}{\PYZsh{}\PYZsh{} Libreria para operaciones numéricas}
\PY{k+kn}{import} \PY{n+nn}{plotly}  \PY{c+c1}{\PYZsh{}\PYZsh{} Libreria para visualizaciones interactivas}
\PY{k+kn}{import} \PY{n+nn}{matplotlib}\PY{n+nn}{.}\PY{n+nn}{ticker} \PY{k}{as} \PY{n+nn}{mtick}  \PY{c+c1}{\PYZsh{}\PYZsh{} función para formatear en visualizaciones}
\PY{k+kn}{from} \PY{n+nn}{scipy}\PY{n+nn}{.}\PY{n+nn}{stats} \PY{k+kn}{import} \PY{n}{skew}\PY{p}{,} \PY{n}{kurtosis}  \PY{c+c1}{\PYZsh{}\PYZsh{} funciones estadísticas}
\PY{k+kn}{from} \PY{n+nn}{scipy}\PY{n+nn}{.}\PY{n+nn}{stats} \PY{k+kn}{import} \PY{n}{norm} \PY{c+c1}{\PYZsh{}\PYZsh{} distribución normal}
\PY{k+kn}{from} \PY{n+nn}{scipy}\PY{n+nn}{.}\PY{n+nn}{stats} \PY{k+kn}{import} \PY{n}{poisson} \PY{c+c1}{\PYZsh{}\PYZsh{} distribución Poisson}
\PY{k+kn}{from} \PY{n+nn}{scipy}\PY{n+nn}{.}\PY{n+nn}{stats} \PY{k+kn}{import} \PY{n}{t} \PY{c+c1}{\PYZsh{}\PYZsh{} distribución t}
\PY{k+kn}{from} \PY{n+nn}{scipy}\PY{n+nn}{.}\PY{n+nn}{stats} \PY{k+kn}{import} \PY{n}{f} \PY{c+c1}{\PYZsh{}\PYZsh{} distribución F}
\PY{k+kn}{from} \PY{n+nn}{scipy}\PY{n+nn}{.}\PY{n+nn}{stats} \PY{k+kn}{import} \PY{n}{ttest\PYZus{}1samp}  \PY{c+c1}{\PYZsh{}\PYZsh{} Prueba t una población}
\PY{k+kn}{from} \PY{n+nn}{scipy}\PY{n+nn}{.}\PY{n+nn}{stats} \PY{k+kn}{import} \PY{n}{ttest\PYZus{}ind} \PY{c+c1}{\PYZsh{}\PYZsh{} Prueba t comparación medias}
\PY{k+kn}{from} \PY{n+nn}{scipy}\PY{n+nn}{.}\PY{n+nn}{stats} \PY{k+kn}{import} \PY{n}{shapiro} \PY{c+c1}{\PYZsh{}\PYZsh{} Prueba normalidad Shapiro\PYZhy{}Wilks}
\PY{k+kn}{from} \PY{n+nn}{scipy}\PY{n+nn}{.}\PY{n+nn}{stats} \PY{k+kn}{import} \PY{n}{anderson} \PY{c+c1}{\PYZsh{}\PYZsh{} Prueba normalidad Anderson\PYZhy{}Darling}
\PY{k+kn}{from} \PY{n+nn}{scipy}\PY{n+nn}{.}\PY{n+nn}{stats} \PY{k+kn}{import} \PY{n}{levene} \PY{c+c1}{\PYZsh{}\PYZsh{} Prueba homogeneidad de varianzas Levene}
\PY{k+kn}{from} \PY{n+nn}{scipy}\PY{n+nn}{.}\PY{n+nn}{stats} \PY{k+kn}{import} \PY{n}{mannwhitneyu} \PY{c+c1}{\PYZsh{}\PYZsh{} Prueba Mann\PYZhy{}Whitney\PYZhy{}Wilcoxon (comparación dos poblaciones)}
\PY{k+kn}{from} \PY{n+nn}{scipy}\PY{n+nn}{.}\PY{n+nn}{stats} \PY{k+kn}{import} \PY{n}{f\PYZus{}oneway} \PY{c+c1}{\PYZsh{}\PYZsh{} Prueba ANOVA de una vía}
\PY{k+kn}{from} \PY{n+nn}{scipy}\PY{n+nn}{.}\PY{n+nn}{stats} \PY{k+kn}{import} \PY{n}{chi2\PYZus{}contingency} \PY{c+c1}{\PYZsh{}\PYZsh{} Prueba chi cuadrado de Pearson}
\PY{k+kn}{from} \PY{n+nn}{scipy}\PY{n+nn}{.}\PY{n+nn}{stats} \PY{k+kn}{import} \PY{n}{pearsonr} \PY{c+c1}{\PYZsh{}\PYZsh{} Coeficiente de correlación de Pearson con prueba}
\PY{k+kn}{import} \PY{n+nn}{statsmodels}\PY{n+nn}{.}\PY{n+nn}{stats} \PY{k}{as} \PY{n+nn}{sm}  \PY{c+c1}{\PYZsh{}\PYZsh{} estadísticas}
\PY{k+kn}{import} \PY{n+nn}{statsmodels}\PY{n+nn}{.}\PY{n+nn}{api} \PY{k}{as} \PY{n+nn}{sm1}  \PY{c+c1}{\PYZsh{}\PYZsh{} estadísticas}
\PY{k+kn}{from} \PY{n+nn}{statsmodels}\PY{n+nn}{.}\PY{n+nn}{graphics}\PY{n+nn}{.}\PY{n+nn}{gofplots} \PY{k+kn}{import} \PY{n}{qqplot} \PY{c+c1}{\PYZsh{}\PYZsh{} Gráfico QQ plot}
\PY{k+kn}{import} \PY{n+nn}{pingouin} \PY{k}{as} \PY{n+nn}{pg} \PY{c+c1}{\PYZsh{}\PYZsh{} Librería funciones estadísticas}
\end{Verbatim}
\end{tcolorbox}


    \section{Primer punto}\label{primer-punto}

\begin{itemize}
\tightlist
\item
  El conjunto de datos ``publicidad.csv'' consta de 200 observaciones y
  4 variables que representan los gastos en publicidad (en miles de
  dólares) y las ventas (en miles de unidades) de un producto en un
  mercado específico: - TV: Gasto en publicidad en televisión. - Radio:
  Gasto en publicidad en radio. - Newspaper: Gasto en publicidad en
  periódicos. - Sales: Número de unidades vendidas (en miles)
\end{itemize}

Solo visualizar la información

    \begin{tcolorbox}[breakable, size=fbox, boxrule=1pt, pad at break*=1mm,colback=cellbackground, colframe=cellborder]
\prompt{In}{incolor}{2}{\boxspacing}
\begin{Verbatim}[commandchars=\\\{\}]
\PY{n}{ruta\PYZus{}archivo} \PY{o}{=} \PY{l+s+s2}{\PYZdq{}}\PY{l+s+s2}{datasets/publicidad.csv}\PY{l+s+s2}{\PYZdq{}}
\PY{n}{data\PYZus{}publicidad} \PY{o}{=} \PY{n}{pd}\PY{o}{.}\PY{n}{read\PYZus{}csv}\PY{p}{(}\PY{n}{ruta\PYZus{}archivo}\PY{p}{)}
\PY{n}{data\PYZus{}publicidad} \PY{o}{=} \PY{n}{data\PYZus{}publicidad}\PY{o}{.}\PY{n}{rename}\PY{p}{(}\PY{n}{columns}\PY{o}{=}\PY{p}{\PYZob{}}\PY{l+s+s2}{\PYZdq{}}\PY{l+s+s2}{Unnamed: 0}\PY{l+s+s2}{\PYZdq{}}\PY{p}{:} \PY{l+s+s2}{\PYZdq{}}\PY{l+s+s2}{Índice}\PY{l+s+s2}{\PYZdq{}}\PY{p}{\PYZcb{}}\PY{p}{)}
\PY{n}{data\PYZus{}publicidad} \PY{o}{=} \PY{n}{data\PYZus{}publicidad}\PY{o}{.}\PY{n}{drop}\PY{p}{(}\PY{n}{columns}\PY{o}{=}\PY{p}{\PYZob{}}\PY{l+s+s2}{\PYZdq{}}\PY{l+s+s2}{Índice}\PY{l+s+s2}{\PYZdq{}}\PY{p}{\PYZcb{}}\PY{p}{)}
\PY{n}{data\PYZus{}publicidad}
\end{Verbatim}
\end{tcolorbox}

            \begin{tcolorbox}[breakable, size=fbox, boxrule=.5pt, pad at break*=1mm, opacityfill=0]
\prompt{Out}{outcolor}{2}{\boxspacing}
\begin{Verbatim}[commandchars=\\\{\}]
        TV  Radio  Newspaper  Sales
0    230.1   37.8       69.2   22.1
1     44.5   39.3       45.1   10.4
2     17.2   45.9       69.3    9.3
3    151.5   41.3       58.5   18.5
4    180.8   10.8       58.4   12.9
..     {\ldots}    {\ldots}        {\ldots}    {\ldots}
195   38.2    3.7       13.8    7.6
196   94.2    4.9        8.1    9.7
197  177.0    9.3        6.4   12.8
198  283.6   42.0       66.2   25.5
199  232.1    8.6        8.7   13.4

[200 rows x 4 columns]
\end{Verbatim}
\end{tcolorbox}
        
    \begin{tcolorbox}[breakable, size=fbox, boxrule=1pt, pad at break*=1mm,colback=cellbackground, colframe=cellborder]
\prompt{In}{incolor}{3}{\boxspacing}
\begin{Verbatim}[commandchars=\\\{\}]
\PY{n}{data\PYZus{}publicidad}\PY{o}{.}\PY{n}{info}\PY{p}{(}\PY{p}{)}
\end{Verbatim}
\end{tcolorbox}

    \begin{Verbatim}[commandchars=\\\{\}]
<class 'pandas.core.frame.DataFrame'>
RangeIndex: 200 entries, 0 to 199
Data columns (total 4 columns):
 \#   Column     Non-Null Count  Dtype
---  ------     --------------  -----
 0   TV         200 non-null    float64
 1   Radio      200 non-null    float64
 2   Newspaper  200 non-null    float64
 3   Sales      200 non-null    float64
dtypes: float64(4)
memory usage: 6.4 KB
    \end{Verbatim}

    De esta revisión general de la información. Se puede decir que esta
información consta de variables cuantitativas continuas, tipo float. Que
no hay datos nulos, en su totalidad está completa

    \section{Punto 2}\label{punto-2}

Graficar el retorno de inversión (variable ``Sales'') vs la cantidad de
anuncios publicitarios por canal (``TV'', ``Radio'', ``Newspaper'').
Para ello use la función scatter\_matrix() del paquete pandas e
interprete los graficos de las variables dos a dos, teniendo en cuenta
que nuestra variable respuesta es ``Sales''.

    \begin{tcolorbox}[breakable, size=fbox, boxrule=1pt, pad at break*=1mm,colback=cellbackground, colframe=cellborder]
\prompt{In}{incolor}{4}{\boxspacing}
\begin{Verbatim}[commandchars=\\\{\}]
\PY{n}{pd}\PY{o}{.}\PY{n}{plotting}\PY{o}{.}\PY{n}{scatter\PYZus{}matrix}\PY{p}{(}\PY{n}{data\PYZus{}publicidad}\PY{p}{,} \PY{n}{figsize}\PY{o}{=}\PY{p}{(}\PY{l+m+mi}{10}\PY{p}{,} \PY{l+m+mi}{10}\PY{p}{)}\PY{p}{)}
\end{Verbatim}
\end{tcolorbox}

            \begin{tcolorbox}[breakable, size=fbox, boxrule=.5pt, pad at break*=1mm, opacityfill=0]
\prompt{Out}{outcolor}{4}{\boxspacing}
\end{tcolorbox}
        
    \begin{center}
    \adjustimage{max size={0.9\linewidth}{0.9\paperheight}}{punto_4_files/punto_4_7_1.png}
    \end{center}
    { \hspace*{\fill} \\}
    
    Se puede analizar lo siguiente:

Análisis entre variables Y con X

\begin{enumerate}
\def\labelenumi{\arabic{enumi}.}
\item
  Sales vs TV Al observar el gráfico de dispersión entre las unidades
  vendidas y el gasto en miles de dolares en Televisión, se puede
  observar una relación significativa entre las dos variables. No podría
  decir aún con exactitud que tipo de\\
  relación tienen, si lineal u otra, pareciera ser una relación lineal
  con cierta variación.
\item
  Sales vs Radio Al observar el gráfico de dispersión entre las unidades
  vendidas y el gasto en miles de dolares en radio, se puede observar
  una relación entre ambas variables, sin embargo con una mayor
  dispersión, es decir me hace pensar que tendría una menor correlación
  que el gasto en TV analizado anteriormente. En este caso también
  pensaría que se trata de una relación lineal pero menos fuerte
\item
  Sales vs Newspaper Al observar el gráfico de dispersión entre las
  unidades vendidas y el gasto en miles de dolares en Periódico se
  observa una mayor dispersión de los datos y eso indicaría menor
  correlación entre estas variables. Me hace pensar que la correlación
  más fuerte vendría a estar en las variables de gasto de publicidad en
  miles de dolares en TV y en Radio
\end{enumerate}

Adicional a lo anterior también se puede ver la dependencia de las
variables explicativas X entre ellas. Por ejemplo:

Análisis entre variables X con X

\begin{enumerate}
\def\labelenumi{\arabic{enumi}.}
\setcounter{enumi}{3}
\item
  TV vs Radio No se observa ninguna dependencia de estas variables, la
  una con la otra. los datos se encuentran totalmente dispersos y esto
  me hace pensar que no habría problemas de multicolinealidad entre
  ellas para un futuro modelo de regresión
\item
  TV vs Newspaper No se observa tampoco ninguna dependencia entre estas
  variables. Los datos se encuentran totalmente dispersos y esto me hace
  pensar que tampoco se deberían de presentar problemas de
  multicolinealidad entre estas variables en caso de formar parte de un
  modelo de regresión
\item
  Radio vs Newspaper No se observa tampoco ninguna dependencia entre
  estas variables. Los datos se encuentran totalmente dispersos y esto
  me hace pensar que tampoco se deberían de presentar problemas de
  multicolinealidad entre estas variables en caso de formar parte de un
  modelo de regresión
\end{enumerate}

Finalmente, se puede analizar la distribución de los datos de cada uno
en los histogramas de la diagonal

De esto se puede observar que la variable Y de unidades vendidas sigue
una distribución aparentemente normal, con un leve sesgo negativo.

Las variables X de Gasto en miles de dolares en TV y Radio no
presentan una aparente distribución de los datos

y la variable X de Gasto en miles de dolares en Newspaper aparenta
presentar una distribución F de los datos o al menos una distribución
mas sesgada de la normalidad

    \begin{tcolorbox}[breakable, size=fbox, boxrule=1pt, pad at break*=1mm,colback=cellbackground, colframe=cellborder]
\prompt{In}{incolor}{5}{\boxspacing}
\begin{Verbatim}[commandchars=\\\{\}]
\PY{k}{def} \PY{n+nf}{tabla\PYZus{}descriptivas}\PY{p}{(}\PY{n}{columnas}\PY{p}{:} \PY{n+nb}{list}\PY{p}{)}\PY{p}{:}
  \PY{n}{tabla\PYZus{}descriptivas}\PY{o}{=}\PY{n}{pd}\PY{o}{.}\PY{n}{DataFrame}\PY{p}{(}\PY{n}{columnas}\PY{o}{.}\PY{n}{describe}\PY{p}{(}\PY{p}{)}\PY{p}{)}
  \PY{n}{tabla\PYZus{}descriptivas}\PY{o}{.}\PY{n}{loc}\PY{p}{[}\PY{l+s+s1}{\PYZsq{}}\PY{l+s+s1}{coef. variation}\PY{l+s+s1}{\PYZsq{}}\PY{p}{]}\PY{o}{=}\PY{n}{columnas}\PY{o}{.}\PY{n}{std}\PY{p}{(}\PY{p}{)}\PY{o}{/}\PY{n}{columnas}\PY{o}{.}\PY{n}{mean}\PY{p}{(}\PY{p}{)}
  \PY{n}{tabla\PYZus{}descriptivas}\PY{o}{.}\PY{n}{loc}\PY{p}{[}\PY{l+s+s1}{\PYZsq{}}\PY{l+s+s1}{skew}\PY{l+s+s1}{\PYZsq{}}\PY{p}{]}\PY{o}{=}\PY{n}{skew}\PY{p}{(}\PY{n}{columnas}\PY{p}{)}
  \PY{n}{tabla\PYZus{}descriptivas}\PY{o}{.}\PY{n}{loc}\PY{p}{[}\PY{l+s+s1}{\PYZsq{}}\PY{l+s+s1}{kurtosis}\PY{l+s+s1}{\PYZsq{}}\PY{p}{]}\PY{o}{=}\PY{n}{kurtosis}\PY{p}{(}\PY{n}{columnas}\PY{p}{)}
  \PY{n}{tabla\PYZus{}descriptivas}\PY{o}{.}\PY{n}{loc}\PY{p}{[}\PY{l+s+s1}{\PYZsq{}}\PY{l+s+s1}{mediana}\PY{l+s+s1}{\PYZsq{}}\PY{p}{]}\PY{o}{=}\PY{n}{columnas}\PY{o}{.}\PY{n}{median}\PY{p}{(}\PY{p}{)}

  \PY{c+c1}{\PYZsh{} Fix the indentation for the loop}
  \PY{k}{for} \PY{n}{columna} \PY{o+ow}{in} \PY{n}{columnas}\PY{p}{:}
      \PY{n}{tabla\PYZus{}descriptivas}\PY{o}{.}\PY{n}{loc}\PY{p}{[}\PY{l+s+s1}{\PYZsq{}}\PY{l+s+s1}{coef. variation}\PY{l+s+s1}{\PYZsq{}}\PY{p}{,} \PY{n}{columna}\PY{p}{]} \PY{o}{=} \PY{n}{data\PYZus{}publicidad}\PY{p}{[}\PY{n}{columna}\PY{p}{]}\PY{o}{.}\PY{n}{std}\PY{p}{(}\PY{p}{)} \PY{o}{/} \PY{n}{data\PYZus{}publicidad}\PY{p}{[}\PY{n}{columna}\PY{p}{]}\PY{o}{.}\PY{n}{mean}\PY{p}{(}\PY{p}{)}
      \PY{n}{tabla\PYZus{}descriptivas}\PY{o}{.}\PY{n}{loc}\PY{p}{[}\PY{l+s+s1}{\PYZsq{}}\PY{l+s+s1}{skew}\PY{l+s+s1}{\PYZsq{}}\PY{p}{,} \PY{n}{columna}\PY{p}{]} \PY{o}{=} \PY{n}{skew}\PY{p}{(}\PY{n}{data\PYZus{}publicidad}\PY{p}{[}\PY{n}{columna}\PY{p}{]}\PY{p}{)}
      \PY{n}{tabla\PYZus{}descriptivas}\PY{o}{.}\PY{n}{loc}\PY{p}{[}\PY{l+s+s1}{\PYZsq{}}\PY{l+s+s1}{kurtosis}\PY{l+s+s1}{\PYZsq{}}\PY{p}{,} \PY{n}{columna}\PY{p}{]} \PY{o}{=} \PY{n}{kurtosis}\PY{p}{(}\PY{n}{data\PYZus{}publicidad}\PY{p}{[}\PY{n}{columna}\PY{p}{]}\PY{p}{)}
      \PY{n}{tabla\PYZus{}descriptivas}\PY{o}{.}\PY{n}{loc}\PY{p}{[}\PY{l+s+s1}{\PYZsq{}}\PY{l+s+s1}{mediana}\PY{l+s+s1}{\PYZsq{}}\PY{p}{]}\PY{o}{=}\PY{n}{columnas}\PY{o}{.}\PY{n}{median}\PY{p}{(}\PY{p}{)}
  \PY{k}{return} \PY{n}{tabla\PYZus{}descriptivas}

\PY{n}{tabla\PYZus{}descriptivas}\PY{p}{(}\PY{n}{data\PYZus{}publicidad}\PY{p}{[}\PY{p}{[}\PY{l+s+s1}{\PYZsq{}}\PY{l+s+s1}{TV}\PY{l+s+s1}{\PYZsq{}}\PY{p}{,} \PY{l+s+s1}{\PYZsq{}}\PY{l+s+s1}{Radio}\PY{l+s+s1}{\PYZsq{}}\PY{p}{,} \PY{l+s+s1}{\PYZsq{}}\PY{l+s+s1}{Newspaper}\PY{l+s+s1}{\PYZsq{}}\PY{p}{,} \PY{l+s+s1}{\PYZsq{}}\PY{l+s+s1}{Sales}\PY{l+s+s1}{\PYZsq{}}\PY{p}{]}\PY{p}{]}\PY{p}{)}
\end{Verbatim}
\end{tcolorbox}

            \begin{tcolorbox}[breakable, size=fbox, boxrule=.5pt, pad at break*=1mm, opacityfill=0]
\prompt{Out}{outcolor}{5}{\boxspacing}
\begin{Verbatim}[commandchars=\\\{\}]
                         TV       Radio   Newspaper       Sales
count            200.000000  200.000000  200.000000  200.000000
mean             147.042500   23.264000   30.554000   14.022500
std               85.854236   14.846809   21.778621    5.217457
min                0.700000    0.000000    0.300000    1.600000
25\%               74.375000    9.975000   12.750000   10.375000
50\%              149.750000   22.900000   25.750000   12.900000
75\%              218.825000   36.525000   45.100000   17.400000
max              296.400000   49.600000  114.000000   27.000000
coef. variation    0.583874    0.638188    0.712791    0.372077
skew              -0.069328    0.093467    0.887996    0.404508
kurtosis          -1.225897   -1.258962    0.603527   -0.428570
mediana          149.750000   22.900000   25.750000   12.900000
\end{Verbatim}
\end{tcolorbox}
        
    Adicional a lo anterior, Se puede observar la siguiente información:

\begin{itemize}
\tightlist
\item
  La media y la mediana no son muy diferentes, eso hace pensar que los
  datos no se encuentran tan sesgados
\item
  La kurtosis de todos los valores no es tan grande, lo que puede
  significar una baja dispersión de los datos. Sin embargo para TV,
  Radio y Ventas se observa una kurtosis negativa que indica que la
  distribución de los datos tiene a estar más aplanada, en cambio en el
  Periódico se muestra una distribución más alargada en su centro.
\item
  El skew mide el grado de asimetría de los datos. En este caso para el
  Radio, Periódico y ventas se presenta unos valores positivos que
  indican distribución sesgada hacia la izquierda y el valor negativo de
  TV indica lo contrario, un sesgo hacia la derecha
\item
  El coeficiente de variación de cada variable, vista
  independientemente, indica que tan dispersos se encuentran los datos
  con respecto a su punto central. En este caso se observa que en
  general todas las variables tienen una dispersión alta
\end{itemize}

    \section{Punto 3}\label{punto-3}

Calcular el coeficiente de correlación entre todas las variables y
mediante un mapa de calor represente estas correlaciones. ¿Interprete
las estructuras de dependencia encontradas?

    \begin{tcolorbox}[breakable, size=fbox, boxrule=1pt, pad at break*=1mm,colback=cellbackground, colframe=cellborder]
\prompt{In}{incolor}{6}{\boxspacing}
\begin{Verbatim}[commandchars=\\\{\}]
\PY{n}{data\PYZus{}publicidad}\PY{o}{.}\PY{n}{corr}\PY{p}{(}\PY{n}{method} \PY{o}{=} \PY{l+s+s2}{\PYZdq{}}\PY{l+s+s2}{pearson}\PY{l+s+s2}{\PYZdq{}}\PY{p}{)}
\end{Verbatim}
\end{tcolorbox}

            \begin{tcolorbox}[breakable, size=fbox, boxrule=.5pt, pad at break*=1mm, opacityfill=0]
\prompt{Out}{outcolor}{6}{\boxspacing}
\begin{Verbatim}[commandchars=\\\{\}]
                 TV     Radio  Newspaper     Sales
TV         1.000000  0.054809   0.056648  0.782224
Radio      0.054809  1.000000   0.354104  0.576223
Newspaper  0.056648  0.354104   1.000000  0.228299
Sales      0.782224  0.576223   0.228299  1.000000
\end{Verbatim}
\end{tcolorbox}
        
    \begin{tcolorbox}[breakable, size=fbox, boxrule=1pt, pad at break*=1mm,colback=cellbackground, colframe=cellborder]
\prompt{In}{incolor}{7}{\boxspacing}
\begin{Verbatim}[commandchars=\\\{\}]
\PY{n}{corrmat} \PY{o}{=} \PY{n}{data\PYZus{}publicidad}\PY{o}{.}\PY{n}{corr}\PY{p}{(}\PY{n}{method} \PY{o}{=} \PY{l+s+s2}{\PYZdq{}}\PY{l+s+s2}{pearson}\PY{l+s+s2}{\PYZdq{}}\PY{p}{)}
\PY{n}{hm} \PY{o}{=} \PY{n}{sns}\PY{o}{.}\PY{n}{heatmap}\PY{p}{(}\PY{n}{corrmat}\PY{p}{,} 
                 \PY{n}{cbar}\PY{o}{=}\PY{k+kc}{True}\PY{p}{,} 
                 \PY{n}{annot}\PY{o}{=}\PY{k+kc}{True}\PY{p}{,} 
                 \PY{n}{square}\PY{o}{=}\PY{k+kc}{True}\PY{p}{,} 
                 \PY{n}{fmt}\PY{o}{=}\PY{l+s+s1}{\PYZsq{}}\PY{l+s+s1}{.2f}\PY{l+s+s1}{\PYZsq{}}\PY{p}{,} 
                 \PY{n}{annot\PYZus{}kws}\PY{o}{=}\PY{p}{\PYZob{}}\PY{l+s+s1}{\PYZsq{}}\PY{l+s+s1}{size}\PY{l+s+s1}{\PYZsq{}}\PY{p}{:} \PY{l+m+mi}{10}\PY{p}{\PYZcb{}}\PY{p}{,} 
                 \PY{n}{yticklabels}\PY{o}{=}\PY{n}{data\PYZus{}publicidad}\PY{o}{.}\PY{n}{columns}\PY{p}{,} 
                 \PY{n}{xticklabels}\PY{o}{=}\PY{n}{data\PYZus{}publicidad}\PY{o}{.}\PY{n}{columns}\PY{p}{,} 
                 \PY{n}{cmap}\PY{o}{=}\PY{l+s+s2}{\PYZdq{}}\PY{l+s+s2}{Spectral\PYZus{}r}\PY{l+s+s2}{\PYZdq{}}\PY{p}{)}
\PY{n}{plt}\PY{o}{.}\PY{n}{show}\PY{p}{(}\PY{p}{)}
\end{Verbatim}
\end{tcolorbox}

    \begin{center}
    \adjustimage{max size={0.9\linewidth}{0.9\paperheight}}{punto_4_files/punto_4_13_0.png}
    \end{center}
    { \hspace*{\fill} \\}
    
    \begin{tcolorbox}[breakable, size=fbox, boxrule=1pt, pad at break*=1mm,colback=cellbackground, colframe=cellborder]
\prompt{In}{incolor}{8}{\boxspacing}
\begin{Verbatim}[commandchars=\\\{\}]
\PY{n}{data\PYZus{}publicidad}\PY{o}{.}\PY{n}{corr}\PY{p}{(}\PY{n}{method} \PY{o}{=} \PY{l+s+s2}{\PYZdq{}}\PY{l+s+s2}{spearman}\PY{l+s+s2}{\PYZdq{}}\PY{p}{)}
\end{Verbatim}
\end{tcolorbox}

            \begin{tcolorbox}[breakable, size=fbox, boxrule=.5pt, pad at break*=1mm, opacityfill=0]
\prompt{Out}{outcolor}{8}{\boxspacing}
\begin{Verbatim}[commandchars=\\\{\}]
                 TV     Radio  Newspaper     Sales
TV         1.000000  0.056123   0.050840  0.800614
Radio      0.056123  1.000000   0.316979  0.554304
Newspaper  0.050840  0.316979   1.000000  0.194922
Sales      0.800614  0.554304   0.194922  1.000000
\end{Verbatim}
\end{tcolorbox}
        
    \begin{tcolorbox}[breakable, size=fbox, boxrule=1pt, pad at break*=1mm,colback=cellbackground, colframe=cellborder]
\prompt{In}{incolor}{9}{\boxspacing}
\begin{Verbatim}[commandchars=\\\{\}]
\PY{n}{corrmat} \PY{o}{=} \PY{n}{data\PYZus{}publicidad}\PY{o}{.}\PY{n}{corr}\PY{p}{(}\PY{n}{method} \PY{o}{=} \PY{l+s+s2}{\PYZdq{}}\PY{l+s+s2}{spearman}\PY{l+s+s2}{\PYZdq{}}\PY{p}{)}
\PY{n}{hm} \PY{o}{=} \PY{n}{sns}\PY{o}{.}\PY{n}{heatmap}\PY{p}{(}\PY{n}{corrmat}\PY{p}{,} 
                 \PY{n}{cbar}\PY{o}{=}\PY{k+kc}{True}\PY{p}{,} 
                 \PY{n}{annot}\PY{o}{=}\PY{k+kc}{True}\PY{p}{,} 
                 \PY{n}{square}\PY{o}{=}\PY{k+kc}{True}\PY{p}{,} 
                 \PY{n}{fmt}\PY{o}{=}\PY{l+s+s1}{\PYZsq{}}\PY{l+s+s1}{.2f}\PY{l+s+s1}{\PYZsq{}}\PY{p}{,} 
                 \PY{n}{annot\PYZus{}kws}\PY{o}{=}\PY{p}{\PYZob{}}\PY{l+s+s1}{\PYZsq{}}\PY{l+s+s1}{size}\PY{l+s+s1}{\PYZsq{}}\PY{p}{:} \PY{l+m+mi}{10}\PY{p}{\PYZcb{}}\PY{p}{,} 
                 \PY{n}{yticklabels}\PY{o}{=}\PY{n}{data\PYZus{}publicidad}\PY{o}{.}\PY{n}{columns}\PY{p}{,} 
                 \PY{n}{xticklabels}\PY{o}{=}\PY{n}{data\PYZus{}publicidad}\PY{o}{.}\PY{n}{columns}\PY{p}{,} 
                 \PY{n}{cmap}\PY{o}{=}\PY{l+s+s2}{\PYZdq{}}\PY{l+s+s2}{Spectral\PYZus{}r}\PY{l+s+s2}{\PYZdq{}}\PY{p}{)}
\PY{n}{plt}\PY{o}{.}\PY{n}{show}\PY{p}{(}\PY{p}{)}
\end{Verbatim}
\end{tcolorbox}

    \begin{center}
    \adjustimage{max size={0.9\linewidth}{0.9\paperheight}}{punto_4_files/punto_4_15_0.png}
    \end{center}
    { \hspace*{\fill} \\}
    
    En primer lugar es importante decir que se utilizó dos tipos de
correlaciones: una Pearson o una Spearman. Entendiendo que Pearson
unicamente describe correlación de manera efectiva si se trata de una
relación lineal entre las variables. Mientras que Spearman puede indicar
una relación entre las variables que no necesariamente es lineal.

Cada uno tiene sus valores de correlaciones entre variables y
graficamente un diagrama de calor que indica una correlación fuerte para
colores entre Naranja y Rojo que va desde el 0.75 al 1. y una
correlación moderada en el color amarillo. El resto de colores ya
indican baja o cero correlación.

En este caso puntual tanto para Pearson como Spearman se detecta una
correlación fuerte entre la variable explciativa ``X'' de gasto de
publicidad en miles de dolares en TV con la variable ``Y'' de unidades
vendidas Luego le sigue con una correlación moderada entre la variable
explicativa ``X'' de gasto de publicidad en miles de dolares en Radio y
la variable ``Y'' en unidades vendidas.

El resto de variables realmente no presentan ninguna correlación con la
variable ``Y''

Por lo tanto, se puede asumir que la correlación es lineal entre estas
variables x con Y y que la correlación más fuerte es con ``TV''

    \begin{tcolorbox}[breakable, size=fbox, boxrule=1pt, pad at break*=1mm,colback=cellbackground, colframe=cellborder]
\prompt{In}{incolor}{10}{\boxspacing}
\begin{Verbatim}[commandchars=\\\{\}]
\PY{c+c1}{\PYZsh{} Graficar el diagrama de dispersión con Seaborn}
\PY{n}{sns}\PY{o}{.}\PY{n}{scatterplot}\PY{p}{(}\PY{n}{x}\PY{o}{=}\PY{l+s+s1}{\PYZsq{}}\PY{l+s+s1}{TV}\PY{l+s+s1}{\PYZsq{}}\PY{p}{,} \PY{n}{y}\PY{o}{=}\PY{l+s+s1}{\PYZsq{}}\PY{l+s+s1}{Sales}\PY{l+s+s1}{\PYZsq{}}\PY{p}{,} \PY{n}{data}\PY{o}{=}\PY{n}{data\PYZus{}publicidad}\PY{p}{)}

\PY{c+c1}{\PYZsh{} Añadir título}
\PY{n}{plt}\PY{o}{.}\PY{n}{title}\PY{p}{(}\PY{l+s+s1}{\PYZsq{}}\PY{l+s+s1}{Diagrama de Dispersión}\PY{l+s+s1}{\PYZsq{}}\PY{p}{)}

\PY{c+c1}{\PYZsh{} Mostrar el gráfico}
\PY{n}{plt}\PY{o}{.}\PY{n}{show}\PY{p}{(}\PY{p}{)}
\end{Verbatim}
\end{tcolorbox}

    \begin{center}
    \adjustimage{max size={0.9\linewidth}{0.9\paperheight}}{punto_4_files/punto_4_17_0.png}
    \end{center}
    { \hspace*{\fill} \\}
    
    \section{Punto 4}\label{punto-4}

Teniedo en cuenta el punto anterior, elija solo una variable explicativa
(``TV'', ``Radio'', o ``Newspaper''; la más conveniente) para modelar
las ventas (``Sales''), ajuste el modelo de regresión lineal simple y
encuentra la ecuación de la recta. ¿Cuál es el valor del coeficiente de
determinación R2? ¿Cómo se interpreta este valor?

    \begin{tcolorbox}[breakable, size=fbox, boxrule=1pt, pad at break*=1mm,colback=cellbackground, colframe=cellborder]
\prompt{In}{incolor}{11}{\boxspacing}
\begin{Verbatim}[commandchars=\\\{\}]
\PY{k+kn}{import} \PY{n+nn}{statsmodels}\PY{n+nn}{.}\PY{n+nn}{api} \PY{k}{as} \PY{n+nn}{sm}

\PY{n}{data\PYZus{}modelo} \PY{o}{=} \PY{n}{data\PYZus{}publicidad}\PY{o}{.}\PY{n}{drop}\PY{p}{(}\PY{n}{columns}\PY{o}{=}\PY{p}{\PYZob{}}\PY{l+s+s2}{\PYZdq{}}\PY{l+s+s2}{Radio}\PY{l+s+s2}{\PYZdq{}}\PY{p}{,} \PY{l+s+s2}{\PYZdq{}}\PY{l+s+s2}{Newspaper}\PY{l+s+s2}{\PYZdq{}}\PY{p}{\PYZcb{}}\PY{p}{)}

\PY{n}{X} \PY{o}{=} \PY{n}{sm}\PY{o}{.}\PY{n}{add\PYZus{}constant}\PY{p}{(}\PY{n}{data\PYZus{}modelo}\PY{o}{.}\PY{n}{drop}\PY{p}{(}\PY{l+s+s2}{\PYZdq{}}\PY{l+s+s2}{Sales}\PY{l+s+s2}{\PYZdq{}}\PY{p}{,} \PY{n}{axis} \PY{o}{=} \PY{l+m+mi}{1}\PY{p}{)}\PY{p}{)}
\PY{n}{y} \PY{o}{=} \PY{n}{data\PYZus{}modelo}\PY{p}{[}\PY{p}{[}\PY{l+s+s2}{\PYZdq{}}\PY{l+s+s2}{Sales}\PY{l+s+s2}{\PYZdq{}}\PY{p}{]}\PY{p}{]}

\PY{n}{model} \PY{o}{=} \PY{n}{sm}\PY{o}{.}\PY{n}{OLS}\PY{p}{(}\PY{n}{y}\PY{p}{,} \PY{n}{X}\PY{p}{)}
\PY{n}{result} \PY{o}{=} \PY{n}{model}\PY{o}{.}\PY{n}{fit}\PY{p}{(}\PY{p}{)}

\PY{n+nb}{print}\PY{p}{(}\PY{n}{result}\PY{o}{.}\PY{n}{summary}\PY{p}{(}\PY{p}{)}\PY{p}{)}
\end{Verbatim}
\end{tcolorbox}

    \begin{Verbatim}[commandchars=\\\{\}]
                            OLS Regression Results
==============================================================================
Dep. Variable:                  Sales   R-squared:                       0.612
Model:                            OLS   Adj. R-squared:                  0.610
Method:                 Least Squares   F-statistic:                     312.1
Date:                Tue, 09 Apr 2024   Prob (F-statistic):           1.47e-42
Time:                        20:50:44   Log-Likelihood:                -519.05
No. Observations:                 200   AIC:                             1042.
Df Residuals:                     198   BIC:                             1049.
Df Model:                           1
Covariance Type:            nonrobust
==============================================================================
                 coef    std err          t      P>|t|      [0.025      0.975]
------------------------------------------------------------------------------
const          7.0326      0.458     15.360      0.000       6.130       7.935
TV             0.0475      0.003     17.668      0.000       0.042       0.053
==============================================================================
Omnibus:                        0.531   Durbin-Watson:                   1.935
Prob(Omnibus):                  0.767   Jarque-Bera (JB):                0.669
Skew:                          -0.089   Prob(JB):                        0.716
Kurtosis:                       2.779   Cond. No.                         338.
==============================================================================

Notes:
[1] Standard Errors assume that the covariance matrix of the errors is correctly
specified.
    \end{Verbatim}

    Con este modelo de regresión se obtiene un R2 ajustado de 0.61. Esto
significa que aproximadamente el 61\% de variabilidad en las unidades
vendidas (variable y) se puede explicar por el gasto de publicidad en TV
(variable x)

Adicional se puede observar que el valor P es menor que 0.05 y por ello
se puede concluir que el modelo es estadísticamente significativo.
Adicional el valor P de la variable TV es menor a 0.05

    \section{Prueba de Independencia}\label{prueba-de-independencia}

\begin{itemize}
\tightlist
\item
  Ho = Los Ei son independientes --\textgreater{} que significa que la
  correlación de un Ei con un Ei+1 es igual a cero (O sea que no hay
  autocorrelación)
\item
  Ha = La correlación de un Ei con un Ei+1 NO es igual a cero
\end{itemize}

    \begin{tcolorbox}[breakable, size=fbox, boxrule=1pt, pad at break*=1mm,colback=cellbackground, colframe=cellborder]
\prompt{In}{incolor}{20}{\boxspacing}
\begin{Verbatim}[commandchars=\\\{\}]
\PY{l+s+sd}{\PYZdq{}\PYZdq{}\PYZdq{}Prueba de Independencia\PYZdq{}\PYZdq{}\PYZdq{}}

\PY{c+c1}{\PYZsh{} Añadir una constante a la matriz de predictores para ajustar el intercepto}
\PY{n}{X} \PY{o}{=} \PY{n}{sm}\PY{o}{.}\PY{n}{add\PYZus{}constant}\PY{p}{(}\PY{n}{X}\PY{p}{)}

\PY{c+c1}{\PYZsh{} Ajustar el modelo de regresión}
\PY{n}{model} \PY{o}{=} \PY{n}{sm}\PY{o}{.}\PY{n}{OLS}\PY{p}{(}\PY{n}{y}\PY{p}{,} \PY{n}{X}\PY{p}{)}\PY{o}{.}\PY{n}{fit}\PY{p}{(}\PY{p}{)}

\PY{c+c1}{\PYZsh{} Calcular los residuos del modelo}
\PY{n}{resid} \PY{o}{=} \PY{n}{model}\PY{o}{.}\PY{n}{resid}

\PY{c+c1}{\PYZsh{} Calcular la estadística de Durbin\PYZhy{}Watson}
\PY{n}{durbin\PYZus{}watson\PYZus{}statistic} \PY{o}{=} \PY{n}{sm}\PY{o}{.}\PY{n}{stats}\PY{o}{.}\PY{n}{stattools}\PY{o}{.}\PY{n}{durbin\PYZus{}watson}\PY{p}{(}\PY{n}{resid}\PY{p}{)}

\PY{c+c1}{\PYZsh{} Imprimir el valor de la estadística Durbin\PYZhy{}Watson}
\PY{n+nb}{print}\PY{p}{(}\PY{l+s+s2}{\PYZdq{}}\PY{l+s+s2}{Estadística de Durbin\PYZhy{}Watson:}\PY{l+s+s2}{\PYZdq{}}\PY{p}{,} \PY{n}{durbin\PYZus{}watson\PYZus{}statistic}\PY{p}{)}

\PY{c+c1}{\PYZsh{} Realizar una prueba de hipótesis para determinar la significancia}
\PY{n}{alpha} \PY{o}{=} \PY{l+m+mf}{0.05}

\PY{c+c1}{\PYZsh{} Calcular el valor p para la prueba de Durbin\PYZhy{}Watson}
\PY{n}{dw\PYZus{}test\PYZus{}p\PYZus{}value} \PY{o}{=} \PY{n}{sm}\PY{o}{.}\PY{n}{stats}\PY{o}{.}\PY{n}{durbin\PYZus{}watson}\PY{p}{(}\PY{n}{resid}\PY{p}{,} \PY{n}{axis}\PY{o}{=}\PY{l+m+mi}{0}\PY{p}{)}


\PY{n}{plt}\PY{o}{.}\PY{n}{figure}\PY{p}{(}\PY{n}{figsize}\PY{o}{=}\PY{p}{(}\PY{l+m+mi}{15}\PY{p}{,}\PY{l+m+mi}{5}\PY{p}{)}\PY{p}{)}
\PY{n}{plt}\PY{o}{.}\PY{n}{plot}\PY{p}{(}\PY{n}{model}\PY{o}{.}\PY{n}{resid}\PY{p}{,} \PY{l+s+s1}{\PYZsq{}}\PY{l+s+s1}{.\PYZhy{}}\PY{l+s+s1}{\PYZsq{}}\PY{p}{,} \PY{n}{color} \PY{o}{=}\PY{l+s+s2}{\PYZdq{}}\PY{l+s+s2}{darkblue}\PY{l+s+s2}{\PYZdq{}}\PY{p}{,} \PY{n}{linewidth}\PY{o}{=}\PY{l+m+mf}{0.3}\PY{p}{)}
\PY{n}{plt}\PY{o}{.}\PY{n}{show}\PY{p}{(}\PY{p}{)}
\end{Verbatim}
\end{tcolorbox}

    \begin{Verbatim}[commandchars=\\\{\}]
Estadística de Durbin-Watson: 1.93468853728236
    \end{Verbatim}

    \begin{center}
    \adjustimage{max size={0.9\linewidth}{0.9\paperheight}}{punto_4_files/punto_4_22_1.png}
    \end{center}
    { \hspace*{\fill} \\}
    
    Como el estadístico de durbin watson está entre 1.5 y 2.5, entonces no
habría evidencia suficiente para rechazar la Ho, por lo que se cumple el
supuesto de independencia. Adicional, gráficamente también se puede
observar que el comportamiento de los residuos es completamente
aleatorio y no marca ninguna tendencia.

    \section{Prueba de Normalidad}\label{prueba-de-normalidad}

\begin{itemize}
\tightlist
\item
  Ho = Ei sigue comportamiento normal
\item
  Ha = Ei NO sigue comportamiento normal
\end{itemize}

    \begin{tcolorbox}[breakable, size=fbox, boxrule=1pt, pad at break*=1mm,colback=cellbackground, colframe=cellborder]
\prompt{In}{incolor}{21}{\boxspacing}
\begin{Verbatim}[commandchars=\\\{\}]
\PY{l+s+sd}{\PYZdq{}\PYZdq{}\PYZdq{}Prueba de Normalidad\PYZdq{}\PYZdq{}\PYZdq{}}
\PY{k+kn}{from} \PY{n+nn}{scipy} \PY{k+kn}{import} \PY{n}{stats}

\PY{n}{shapiro\PYZus{}test\PYZus{}statistic}\PY{p}{,} \PY{n}{shapiro\PYZus{}p\PYZus{}value} \PY{o}{=} \PY{n}{stats}\PY{o}{.}\PY{n}{shapiro}\PY{p}{(}\PY{n}{model}\PY{o}{.}\PY{n}{resid}\PY{p}{)}
\PY{n+nb}{print}\PY{p}{(}\PY{l+s+s2}{\PYZdq{}}\PY{l+s+s2}{Estadístico de prueba para Shapiro Wilks: }\PY{l+s+s2}{\PYZdq{}}\PY{p}{,} \PY{n}{shapiro\PYZus{}test\PYZus{}statistic}\PY{p}{)}
\PY{n+nb}{print}\PY{p}{(}\PY{l+s+s2}{\PYZdq{}}\PY{l+s+s2}{P\PYZhy{}Value: }\PY{l+s+s2}{\PYZdq{}}\PY{p}{,} \PY{n}{shapiro\PYZus{}p\PYZus{}value}\PY{p}{)}
\end{Verbatim}
\end{tcolorbox}

    \begin{Verbatim}[commandchars=\\\{\}]
Estadístico de prueba para Shapiro Wilks:  0.9905306696891785
P-Value:  0.21332456171512604
    \end{Verbatim}

    \begin{tcolorbox}[breakable, size=fbox, boxrule=1pt, pad at break*=1mm,colback=cellbackground, colframe=cellborder]
\prompt{In}{incolor}{14}{\boxspacing}
\begin{Verbatim}[commandchars=\\\{\}]
\PY{c+c1}{\PYZsh{} Gráfico QQ Plot }

\PY{n}{fig}\PY{p}{,} \PY{n}{ax} \PY{o}{=} \PY{n}{plt}\PY{o}{.}\PY{n}{subplots}\PY{p}{(}\PY{n}{figsize}\PY{o}{=}\PY{p}{(}\PY{l+m+mi}{8}\PY{p}{,} \PY{l+m+mi}{6}\PY{p}{)}\PY{p}{)}
\PY{n}{qqplot}\PY{p}{(}\PY{n}{model}\PY{o}{.}\PY{n}{resid}\PY{p}{,} \PY{n}{line}\PY{o}{=}\PY{l+s+s1}{\PYZsq{}}\PY{l+s+s1}{s}\PY{l+s+s1}{\PYZsq{}}\PY{p}{,} \PY{n}{ax}\PY{o}{=}\PY{n}{ax}\PY{p}{)}
\PY{n}{ax}\PY{o}{.}\PY{n}{set\PYZus{}title}\PY{p}{(}\PY{l+s+s1}{\PYZsq{}}\PY{l+s+s1}{Gráfico de QQ Plot \PYZhy{} Residuos Modelo Reducido}\PY{l+s+s1}{\PYZsq{}}\PY{p}{)}
\PY{n}{plt}\PY{o}{.}\PY{n}{show}\PY{p}{(}\PY{p}{)}
\end{Verbatim}
\end{tcolorbox}

    \begin{center}
    \adjustimage{max size={0.9\linewidth}{0.9\paperheight}}{punto_4_files/punto_4_26_0.png}
    \end{center}
    { \hspace*{\fill} \\}
    
    Conclusión: Como el valor p es mayor que alpha (0.05), entonces no hay
evidencia suficiente para rechazar Ho, por lo que el comportamiento
sigue una distribución normal y se cumple el supuesto de normalidad.
Adicionalmente, al realizar el gráfico QQ Plot se puede observar que
estos siguen una linea recta lo que también implica que se cumple el
supuesto de normalidad

    \section{Prueba Media Cero}\label{prueba-media-cero}

\begin{itemize}
\tightlist
\item
  Ho = el valor esperado de los Ei es igual a cero (promedio igual a
  cero)
\item
  Ha = el valor esperado de los Ei NO es igual a cero (promedio NO igual
  a cero)
\end{itemize}

    \begin{tcolorbox}[breakable, size=fbox, boxrule=1pt, pad at break*=1mm,colback=cellbackground, colframe=cellborder]
\prompt{In}{incolor}{22}{\boxspacing}
\begin{Verbatim}[commandchars=\\\{\}]
\PY{l+s+sd}{\PYZdq{}\PYZdq{}\PYZdq{}Prueba media cero\PYZdq{}\PYZdq{}\PYZdq{}}

\PY{c+c1}{\PYZsh{} Prueba de One Sample T\PYZhy{}Test}

\PY{n}{t\PYZus{}statistic}\PY{p}{,} \PY{n}{p\PYZus{}value} \PY{o}{=} \PY{n}{stats}\PY{o}{.}\PY{n}{ttest\PYZus{}1samp}\PY{p}{(}\PY{n}{model}\PY{o}{.}\PY{n}{resid}\PY{p}{,} \PY{l+m+mi}{0}\PY{p}{)}


\PY{n+nb}{print}\PY{p}{(}\PY{l+s+s2}{\PYZdq{}}\PY{l+s+s2}{Estadístico t:}\PY{l+s+s2}{\PYZdq{}}\PY{p}{,} \PY{n}{t\PYZus{}statistic}\PY{p}{)}
\PY{n+nb}{print}\PY{p}{(}\PY{l+s+s2}{\PYZdq{}}\PY{l+s+s2}{P\PYZhy{}Value:}\PY{l+s+s2}{\PYZdq{}}\PY{p}{,} \PY{n}{p\PYZus{}value}\PY{p}{)}
\end{Verbatim}
\end{tcolorbox}

    \begin{Verbatim}[commandchars=\\\{\}]
Estadístico t: -1.901234536356806e-14
P-Value: 0.9999999999999849
    \end{Verbatim}

    Conclusión: Como el valor p es mayor que alpha (0.05), entonces no hay
evidencia suficiente para rechazar Ho, por lo que el promedio entonces
es igual a cero y se cumple el supuesto de media cero. El gráfico de los
residuos presentado arriba también aplica para este supuesto e indica
que los residuos siguen un comportamiento aleatorio al rededor del cero
lo que también indica que se cumple este supuesto

    \section{Prueba de Homocedasticidad}\label{prueba-de-homocedasticidad}

\begin{itemize}
\tightlist
\item
  Ho = Varianzas de Ei constante
\item
  Ha = Varianzas de Ei NO constante
\end{itemize}

    \begin{tcolorbox}[breakable, size=fbox, boxrule=1pt, pad at break*=1mm,colback=cellbackground, colframe=cellborder]
\prompt{In}{incolor}{23}{\boxspacing}
\begin{Verbatim}[commandchars=\\\{\}]
\PY{n}{bp\PYZus{}test} \PY{o}{=} \PY{n}{sm}\PY{o}{.}\PY{n}{stats}\PY{o}{.}\PY{n}{diagnostic}\PY{o}{.}\PY{n}{het\PYZus{}breuschpagan}\PY{p}{(}\PY{n}{model}\PY{o}{.}\PY{n}{resid}\PY{p}{,} \PY{n}{model}\PY{o}{.}\PY{n}{model}\PY{o}{.}\PY{n}{exog}\PY{p}{)}
\PY{n+nb}{print}\PY{p}{(}\PY{l+s+s2}{\PYZdq{}}\PY{l+s+s2}{Estadístico de prueba de Breusch\PYZhy{}Pagan:}\PY{l+s+s2}{\PYZdq{}}\PY{p}{,} \PY{n}{bp\PYZus{}test}\PY{p}{[}\PY{l+m+mi}{0}\PY{p}{]}\PY{p}{)}
\PY{n+nb}{print}\PY{p}{(}\PY{l+s+s2}{\PYZdq{}}\PY{l+s+s2}{P\PYZhy{}Value:}\PY{l+s+s2}{\PYZdq{}}\PY{p}{,} \PY{n}{bp\PYZus{}test}\PY{p}{[}\PY{l+m+mi}{1}\PY{p}{]}\PY{p}{)}

\PY{n}{plt}\PY{o}{.}\PY{n}{figure}\PY{p}{(}\PY{n}{figsize}\PY{o}{=}\PY{p}{(}\PY{l+m+mi}{15}\PY{p}{,}\PY{l+m+mi}{5}\PY{p}{)}\PY{p}{)}
\PY{n}{plt}\PY{o}{.}\PY{n}{plot}\PY{p}{(}\PY{n}{model}\PY{o}{.}\PY{n}{resid}\PY{p}{,} \PY{l+s+s1}{\PYZsq{}}\PY{l+s+s1}{.\PYZhy{}}\PY{l+s+s1}{\PYZsq{}}\PY{p}{,} \PY{n}{color} \PY{o}{=}\PY{l+s+s2}{\PYZdq{}}\PY{l+s+s2}{darkblue}\PY{l+s+s2}{\PYZdq{}}\PY{p}{,} \PY{n}{linewidth}\PY{o}{=}\PY{l+m+mf}{0.3}\PY{p}{)}
\PY{n}{plt}\PY{o}{.}\PY{n}{show}\PY{p}{(}\PY{p}{)}
\end{Verbatim}
\end{tcolorbox}

    \begin{Verbatim}[commandchars=\\\{\}]
Estadístico de prueba de Breusch-Pagan: 48.037965662293594
P-Value: 4.180455907755742e-12
    \end{Verbatim}

    \begin{center}
    \adjustimage{max size={0.9\linewidth}{0.9\paperheight}}{punto_4_files/punto_4_32_1.png}
    \end{center}
    { \hspace*{\fill} \\}
    
    Conclusión: En la prueba de homocedasticidad, el valor P es menor que el
aplha (0.05). lo que indica que se rechaza Ho y por lo tanto hay
evidencia suficiente para decir que las varianzas no son constantes.
Esto también se puede evidenciar porque el coeficiente de variación de
todas las variables era muy alto al momento de hacer el analisis
exploratorio de los datos, en la primera sección del ejercicio.

Por este motivo se propone realizar un modelo robusto que permita
penalizar las varianzas y de esta manera forzar la homocedasticidad

    \begin{tcolorbox}[breakable, size=fbox, boxrule=1pt, pad at break*=1mm,colback=cellbackground, colframe=cellborder]
\prompt{In}{incolor}{26}{\boxspacing}
\begin{Verbatim}[commandchars=\\\{\}]
\PY{k+kn}{import} \PY{n+nn}{statsmodels}\PY{n+nn}{.}\PY{n+nn}{api} \PY{k}{as} \PY{n+nn}{sm}

\PY{n}{data\PYZus{}modelo} \PY{o}{=} \PY{n}{data\PYZus{}publicidad}\PY{o}{.}\PY{n}{drop}\PY{p}{(}\PY{n}{columns}\PY{o}{=}\PY{p}{\PYZob{}}\PY{l+s+s2}{\PYZdq{}}\PY{l+s+s2}{Radio}\PY{l+s+s2}{\PYZdq{}}\PY{p}{,} \PY{l+s+s2}{\PYZdq{}}\PY{l+s+s2}{Newspaper}\PY{l+s+s2}{\PYZdq{}}\PY{p}{\PYZcb{}}\PY{p}{)}

\PY{n}{X} \PY{o}{=} \PY{n}{sm}\PY{o}{.}\PY{n}{add\PYZus{}constant}\PY{p}{(}\PY{n}{data\PYZus{}modelo}\PY{o}{.}\PY{n}{drop}\PY{p}{(}\PY{l+s+s2}{\PYZdq{}}\PY{l+s+s2}{Sales}\PY{l+s+s2}{\PYZdq{}}\PY{p}{,} \PY{n}{axis} \PY{o}{=} \PY{l+m+mi}{1}\PY{p}{)}\PY{p}{)}
\PY{n}{y} \PY{o}{=} \PY{n}{data\PYZus{}modelo}\PY{p}{[}\PY{p}{[}\PY{l+s+s2}{\PYZdq{}}\PY{l+s+s2}{Sales}\PY{l+s+s2}{\PYZdq{}}\PY{p}{]}\PY{p}{]}

\PY{n}{model} \PY{o}{=} \PY{n}{sm}\PY{o}{.}\PY{n}{OLS}\PY{p}{(}\PY{n}{y}\PY{p}{,} \PY{n}{X}\PY{p}{)}
\PY{n}{result} \PY{o}{=} \PY{n}{model}\PY{o}{.}\PY{n}{fit}\PY{p}{(}\PY{p}{)}

\PY{n+nb}{print}\PY{p}{(}\PY{l+s+s2}{\PYZdq{}}\PY{l+s+s2}{Modelo Ajustado}\PY{l+s+s2}{\PYZdq{}}\PY{p}{)}

\PY{n+nb}{print}\PY{p}{(}\PY{n}{result}\PY{o}{.}\PY{n}{summary}\PY{p}{(}\PY{p}{)}\PY{p}{)}

\PY{n}{rlm\PYZus{}model} \PY{o}{=} \PY{n}{sm}\PY{o}{.}\PY{n}{RLM}\PY{p}{(}\PY{n}{y}\PY{p}{,} \PY{n}{X}\PY{p}{)}

\PY{n}{rlm\PYZus{}results} \PY{o}{=} \PY{n}{rlm\PYZus{}model}\PY{o}{.}\PY{n}{fit}\PY{p}{(}\PY{p}{)}

\PY{n+nb}{print}\PY{p}{(}\PY{l+s+s2}{\PYZdq{}}\PY{l+s+s2}{Modelo Robusto}\PY{l+s+s2}{\PYZdq{}}\PY{p}{)}

\PY{n}{rlm\PYZus{}results}\PY{o}{.}\PY{n}{summary}\PY{p}{(}\PY{p}{)}
\end{Verbatim}
\end{tcolorbox}

    \begin{Verbatim}[commandchars=\\\{\}]
Modelo Ajustado
                            OLS Regression Results
==============================================================================
Dep. Variable:                  Sales   R-squared:                       0.612
Model:                            OLS   Adj. R-squared:                  0.610
Method:                 Least Squares   F-statistic:                     312.1
Date:                Thu, 11 Apr 2024   Prob (F-statistic):           1.47e-42
Time:                        22:21:02   Log-Likelihood:                -519.05
No. Observations:                 200   AIC:                             1042.
Df Residuals:                     198   BIC:                             1049.
Df Model:                           1
Covariance Type:            nonrobust
==============================================================================
                 coef    std err          t      P>|t|      [0.025      0.975]
------------------------------------------------------------------------------
const          7.0326      0.458     15.360      0.000       6.130       7.935
TV             0.0475      0.003     17.668      0.000       0.042       0.053
==============================================================================
Omnibus:                        0.531   Durbin-Watson:                   1.935
Prob(Omnibus):                  0.767   Jarque-Bera (JB):                0.669
Skew:                          -0.089   Prob(JB):                        0.716
Kurtosis:                       2.779   Cond. No.                         338.
==============================================================================

Notes:
[1] Standard Errors assume that the covariance matrix of the errors is correctly
specified.
Modelo Robusto
    \end{Verbatim}
 
            
\prompt{Out}{outcolor}{26}{}
    
    \begin{center}
\begin{tabular}{lclc}
\toprule
\textbf{Dep. Variable:}  &      Sales       & \textbf{  No. Observations:  } &      200    \\
\textbf{Model:}          &       RLM        & \textbf{  Df Residuals:      } &      198    \\
\textbf{Method:}         &       IRLS       & \textbf{  Df Model:          } &        1    \\
\textbf{Norm:}           &      HuberT      & \textbf{                     } &             \\
\textbf{Scale Est.:}     &       mad        & \textbf{                     } &             \\
\textbf{Cov Type:}       &        H1        & \textbf{                     } &             \\
\textbf{Date:}           & Thu, 11 Apr 2024 & \textbf{                     } &             \\
\textbf{Time:}           &     22:21:02     & \textbf{                     } &             \\
\textbf{No. Iterations:} &        23        & \textbf{                     } &             \\
\bottomrule
\end{tabular}
\begin{tabular}{lcccccc}
               & \textbf{coef} & \textbf{std err} & \textbf{z} & \textbf{P$> |$z$|$} & \textbf{[0.025} & \textbf{0.975]}  \\
\midrule
\textbf{const} &       6.8843  &        0.484     &    14.235  &         0.000        &        5.936    &        7.832     \\
\textbf{TV}    &       0.0491  &        0.003     &    17.278  &         0.000        &        0.044    &        0.055     \\
\bottomrule
\end{tabular}
%\caption{Robust linear Model Regression Results}
\end{center}

If the model instance has been used for another fit with different fit parameters, then the fit options might not be the correct ones anymore .

    

    Se puede observar que el modelo resultante tiene un R2 de 0.612 que es
bueno e indica que los valores que se predigan de unidades vendidas (y)
se pueden explicar en un 61.5\% por la variable de gasto en publicidad
en TV (x).

Adicional no se encuentra valor p grande de la variable por lo que el
modelo se ajusta bien.

    De esta manera entonces se obtiene un modelo con la siguiente función:

Y(unidades vendidas) = 6.8843 + 0.0491X(Gasto en miles de dolares)

    \section{Punto 5}\label{punto-5}

Realiza una predicción del retorno de inversión esperado cuando se
realizan 5 anuncios por el canal de la variable escogida en el ítem
anterior. ¿Cuál es el intervalo de confianza del 95 \% para la
predicción?

    \begin{tcolorbox}[breakable, size=fbox, boxrule=1pt, pad at break*=1mm,colback=cellbackground, colframe=cellborder]
\prompt{In}{incolor}{29}{\boxspacing}
\begin{Verbatim}[commandchars=\\\{\}]
\PY{k+kn}{import} \PY{n+nn}{statsmodels}\PY{n+nn}{.}\PY{n+nn}{api} \PY{k}{as} \PY{n+nn}{sm}
\PY{k+kn}{from} \PY{n+nn}{statsmodels}\PY{n+nn}{.}\PY{n+nn}{tsa}\PY{n+nn}{.}\PY{n+nn}{statespace}\PY{n+nn}{.}\PY{n+nn}{sarimax} \PY{k+kn}{import} \PY{n}{SARIMAX}
\end{Verbatim}
\end{tcolorbox}

    \begin{tcolorbox}[breakable, size=fbox, boxrule=1pt, pad at break*=1mm,colback=cellbackground, colframe=cellborder]
\prompt{In}{incolor}{32}{\boxspacing}
\begin{Verbatim}[commandchars=\\\{\}]
\PY{k+kn}{import} \PY{n+nn}{statsmodels}\PY{n+nn}{.}\PY{n+nn}{api} \PY{k}{as} \PY{n+nn}{sm}

\PY{n}{data\PYZus{}modelo} \PY{o}{=} \PY{n}{data\PYZus{}publicidad}\PY{o}{.}\PY{n}{drop}\PY{p}{(}\PY{n}{columns}\PY{o}{=}\PY{p}{\PYZob{}}\PY{l+s+s2}{\PYZdq{}}\PY{l+s+s2}{Radio}\PY{l+s+s2}{\PYZdq{}}\PY{p}{,} \PY{l+s+s2}{\PYZdq{}}\PY{l+s+s2}{Newspaper}\PY{l+s+s2}{\PYZdq{}}\PY{p}{\PYZcb{}}\PY{p}{)}

\PY{n}{X} \PY{o}{=} \PY{n}{sm}\PY{o}{.}\PY{n}{add\PYZus{}constant}\PY{p}{(}\PY{n}{data\PYZus{}modelo}\PY{o}{.}\PY{n}{drop}\PY{p}{(}\PY{l+s+s2}{\PYZdq{}}\PY{l+s+s2}{Sales}\PY{l+s+s2}{\PYZdq{}}\PY{p}{,} \PY{n}{axis} \PY{o}{=} \PY{l+m+mi}{1}\PY{p}{)}\PY{p}{)}
\PY{n}{y} \PY{o}{=} \PY{n}{data\PYZus{}modelo}\PY{p}{[}\PY{p}{[}\PY{l+s+s2}{\PYZdq{}}\PY{l+s+s2}{Sales}\PY{l+s+s2}{\PYZdq{}}\PY{p}{]}\PY{p}{]}

\PY{n}{model} \PY{o}{=} \PY{n}{sm}\PY{o}{.}\PY{n}{OLS}\PY{p}{(}\PY{n}{y}\PY{p}{,} \PY{n}{X}\PY{p}{)}
\PY{n}{result} \PY{o}{=} \PY{n}{model}\PY{o}{.}\PY{n}{fit}\PY{p}{(}\PY{p}{)}

\PY{n+nb}{print}\PY{p}{(}\PY{l+s+s2}{\PYZdq{}}\PY{l+s+s2}{Modelo Ajustado}\PY{l+s+s2}{\PYZdq{}}\PY{p}{)}

\PY{n+nb}{print}\PY{p}{(}\PY{n}{result}\PY{o}{.}\PY{n}{summary}\PY{p}{(}\PY{p}{)}\PY{p}{)}

\PY{n}{rlm\PYZus{}model} \PY{o}{=} \PY{n}{sm}\PY{o}{.}\PY{n}{RLM}\PY{p}{(}\PY{n}{y}\PY{p}{,} \PY{n}{X}\PY{p}{)}

\PY{n}{rlm\PYZus{}results} \PY{o}{=} \PY{n}{rlm\PYZus{}model}\PY{o}{.}\PY{n}{fit}\PY{p}{(}\PY{p}{)}

\PY{n+nb}{print}\PY{p}{(}\PY{l+s+s2}{\PYZdq{}}\PY{l+s+s2}{Modelo Robusto}\PY{l+s+s2}{\PYZdq{}}\PY{p}{)}

\PY{n}{rlm\PYZus{}results}\PY{o}{.}\PY{n}{summary}\PY{p}{(}\PY{p}{)}
\end{Verbatim}
\end{tcolorbox}

    \begin{Verbatim}[commandchars=\\\{\}]
Modelo Ajustado
                            OLS Regression Results
==============================================================================
Dep. Variable:                  Sales   R-squared:                       0.612
Model:                            OLS   Adj. R-squared:                  0.610
Method:                 Least Squares   F-statistic:                     312.1
Date:                Sat, 13 Apr 2024   Prob (F-statistic):           1.47e-42
Time:                        14:33:28   Log-Likelihood:                -519.05
No. Observations:                 200   AIC:                             1042.
Df Residuals:                     198   BIC:                             1049.
Df Model:                           1
Covariance Type:            nonrobust
==============================================================================
                 coef    std err          t      P>|t|      [0.025      0.975]
------------------------------------------------------------------------------
const          7.0326      0.458     15.360      0.000       6.130       7.935
TV             0.0475      0.003     17.668      0.000       0.042       0.053
==============================================================================
Omnibus:                        0.531   Durbin-Watson:                   1.935
Prob(Omnibus):                  0.767   Jarque-Bera (JB):                0.669
Skew:                          -0.089   Prob(JB):                        0.716
Kurtosis:                       2.779   Cond. No.                         338.
==============================================================================

Notes:
[1] Standard Errors assume that the covariance matrix of the errors is correctly
specified.
Modelo Robusto
    \end{Verbatim}
 
            
\prompt{Out}{outcolor}{32}{}
    
    \begin{center}
\begin{tabular}{lclc}
\toprule
\textbf{Dep. Variable:}  &      Sales       & \textbf{  No. Observations:  } &      200    \\
\textbf{Model:}          &       RLM        & \textbf{  Df Residuals:      } &      198    \\
\textbf{Method:}         &       IRLS       & \textbf{  Df Model:          } &        1    \\
\textbf{Norm:}           &      HuberT      & \textbf{                     } &             \\
\textbf{Scale Est.:}     &       mad        & \textbf{                     } &             \\
\textbf{Cov Type:}       &        H1        & \textbf{                     } &             \\
\textbf{Date:}           & Sat, 13 Apr 2024 & \textbf{                     } &             \\
\textbf{Time:}           &     14:33:28     & \textbf{                     } &             \\
\textbf{No. Iterations:} &        23        & \textbf{                     } &             \\
\bottomrule
\end{tabular}
\begin{tabular}{lcccccc}
               & \textbf{coef} & \textbf{std err} & \textbf{z} & \textbf{P$> |$z$|$} & \textbf{[0.025} & \textbf{0.975]}  \\
\midrule
\textbf{const} &       6.8843  &        0.484     &    14.235  &         0.000        &        5.936    &        7.832     \\
\textbf{TV}    &       0.0491  &        0.003     &    17.278  &         0.000        &        0.044    &        0.055     \\
\bottomrule
\end{tabular}
%\caption{Robust linear Model Regression Results}
\end{center}

If the model instance has been used for another fit with different fit parameters, then the fit options might not be the correct ones anymore .

    

    Con este modelo robusto, entonces tenemos la función para predecir Y: -
Y(unidades vendidas) = 6.8843 + 0.0491X(Gasto en miles de dolares)

Pero además las funciones para los intervalos:

\begin{itemize}
\tightlist
\item
  Valor mínimo = Y(unidades vendidas) = 5.936 + 0.044X(Gasto en miles de
  dolares)
\item
  Valor máximo = Y(unidades vendidas) = 7.832 + 0.055X(Gasto en miles de
  dolares)
\end{itemize}

    \begin{tcolorbox}[breakable, size=fbox, boxrule=1pt, pad at break*=1mm,colback=cellbackground, colframe=cellborder]
\prompt{In}{incolor}{54}{\boxspacing}
\begin{Verbatim}[commandchars=\\\{\}]
\PY{n}{data\PYZus{}publicidad}\PY{p}{[}\PY{p}{[}\PY{l+s+s2}{\PYZdq{}}\PY{l+s+s2}{TV}\PY{l+s+s2}{\PYZdq{}}\PY{p}{,} \PY{l+s+s2}{\PYZdq{}}\PY{l+s+s2}{Sales}\PY{l+s+s2}{\PYZdq{}}\PY{p}{]}\PY{p}{]}
\end{Verbatim}
\end{tcolorbox}

            \begin{tcolorbox}[breakable, size=fbox, boxrule=.5pt, pad at break*=1mm, opacityfill=0]
\prompt{Out}{outcolor}{54}{\boxspacing}
\begin{Verbatim}[commandchars=\\\{\}]
        TV  Sales
0    230.1   22.1
1     44.5   10.4
2     17.2    9.3
3    151.5   18.5
4    180.8   12.9
..     {\ldots}    {\ldots}
195   38.2    7.6
196   94.2    9.7
197  177.0   12.8
198  283.6   25.5
199  232.1   13.4

[200 rows x 2 columns]
\end{Verbatim}
\end{tcolorbox}
        
    \begin{tcolorbox}[breakable, size=fbox, boxrule=1pt, pad at break*=1mm,colback=cellbackground, colframe=cellborder]
\prompt{In}{incolor}{57}{\boxspacing}
\begin{Verbatim}[commandchars=\\\{\}]
\PY{n}{muestra\PYZus{}random} \PY{o}{=} \PY{n}{data\PYZus{}publicidad}\PY{p}{[}\PY{p}{[}\PY{l+s+s2}{\PYZdq{}}\PY{l+s+s2}{TV}\PY{l+s+s2}{\PYZdq{}}\PY{p}{,} \PY{l+s+s2}{\PYZdq{}}\PY{l+s+s2}{Sales}\PY{l+s+s2}{\PYZdq{}}\PY{p}{]}\PY{p}{]}\PY{o}{.}\PY{n}{sample}\PY{p}{(}\PY{l+m+mi}{5}\PY{p}{)}

\PY{c+c1}{\PYZsh{} Print the randomly selected 5 rows}
\PY{n+nb}{print}\PY{p}{(}\PY{l+s+s2}{\PYZdq{}}\PY{l+s+s2}{selección aleatoria de 5 datos:}\PY{l+s+s2}{\PYZdq{}}\PY{p}{)}
\PY{n+nb}{print}\PY{p}{(}\PY{n}{muestra\PYZus{}random}\PY{p}{)}
\end{Verbatim}
\end{tcolorbox}

    \begin{Verbatim}[commandchars=\\\{\}]
selección aleatoria de 5 datos:
        TV  Sales
114   78.2   14.6
52   216.4   22.6
6     57.5   11.8
158   11.7    7.3
93   250.9   22.2
    \end{Verbatim}

    \begin{tcolorbox}[breakable, size=fbox, boxrule=1pt, pad at break*=1mm,colback=cellbackground, colframe=cellborder]
\prompt{In}{incolor}{58}{\boxspacing}
\begin{Verbatim}[commandchars=\\\{\}]
\PY{c+c1}{\PYZsh{}Para responder eligiré los primeros 5 valores de X (\PYZdq{}TV\PYZdq{}) \PYZhy{} Modelo OLS}

\PY{k}{def} \PY{n+nf}{predict\PYZus{}sales}\PY{p}{(}\PY{n}{x}\PY{p}{)}\PY{p}{:}
    
    \PY{n}{sales\PYZus{}prediccion} \PY{o}{=} \PY{l+m+mf}{7.0326} \PY{o}{+} \PY{p}{(}\PY{l+m+mf}{0.0475} \PY{o}{*} \PY{n}{x}\PY{p}{)}
    \PY{k}{return} \PY{n}{sales\PYZus{}prediccion}

\PY{k}{def} \PY{n+nf}{sales\PYZus{}minimo}\PY{p}{(}\PY{n}{x}\PY{p}{)}\PY{p}{:}
    
    \PY{n}{minimo} \PY{o}{=} \PY{l+m+mf}{6.130} \PY{o}{+} \PY{p}{(}\PY{l+m+mf}{0.042} \PY{o}{*} \PY{n}{x}\PY{p}{)}
    \PY{k}{return} \PY{n}{minimo}

\PY{k}{def} \PY{n+nf}{sales\PYZus{}maximo}\PY{p}{(}\PY{n}{x}\PY{p}{)}\PY{p}{:}
    
    \PY{n}{maximo} \PY{o}{=} \PY{l+m+mf}{7.935} \PY{o}{+} \PY{p}{(}\PY{l+m+mf}{0.053} \PY{o}{*} \PY{n}{x}\PY{p}{)}
    \PY{k}{return} \PY{n}{maximo}

\PY{n+nb}{print}\PY{p}{(}\PY{l+s+s2}{\PYZdq{}}\PY{l+s+s2}{Valor 1 (y) predecido: }\PY{l+s+s2}{\PYZdq{}} \PY{o}{+} \PY{n+nb}{str}\PY{p}{(}\PY{n}{predict\PYZus{}sales}\PY{p}{(}\PY{l+m+mf}{78.2}\PY{p}{)}\PY{p}{)} \PY{o}{+} \PY{l+s+s2}{\PYZdq{}}\PY{l+s+s2}{ Intervalo: [}\PY{l+s+s2}{\PYZdq{}} \PY{o}{+} \PY{n+nb}{str}\PY{p}{(}\PY{n}{sales\PYZus{}minimo}\PY{p}{(}\PY{l+m+mf}{78.2}\PY{p}{)}\PY{p}{)} \PY{o}{+} \PY{l+s+s2}{\PYZdq{}}\PY{l+s+s2}{ \PYZhy{} }\PY{l+s+s2}{\PYZdq{}} \PY{o}{+} \PY{n+nb}{str}\PY{p}{(}\PY{n}{sales\PYZus{}maximo}\PY{p}{(}\PY{l+m+mf}{78.2}\PY{p}{)}\PY{p}{)} \PY{o}{+} \PY{l+s+s2}{\PYZdq{}}\PY{l+s+s2}{]}\PY{l+s+s2}{\PYZdq{}} \PY{o}{+} \PY{l+s+s2}{\PYZdq{}}\PY{l+s+s2}{ vs }\PY{l+s+s2}{\PYZdq{}} \PY{o}{+} \PY{l+s+s2}{\PYZdq{}}\PY{l+s+s2}{ y real = 14.6}\PY{l+s+s2}{\PYZdq{}}\PY{p}{)}
\PY{n+nb}{print}\PY{p}{(}\PY{l+s+s2}{\PYZdq{}}\PY{l+s+s2}{Valor 2 (y) predecido: }\PY{l+s+s2}{\PYZdq{}} \PY{o}{+} \PY{n+nb}{str}\PY{p}{(}\PY{n}{predict\PYZus{}sales}\PY{p}{(}\PY{l+m+mf}{216.4}\PY{p}{)}\PY{p}{)} \PY{o}{+} \PY{l+s+s2}{\PYZdq{}}\PY{l+s+s2}{ Intervalo: [}\PY{l+s+s2}{\PYZdq{}} \PY{o}{+} \PY{n+nb}{str}\PY{p}{(}\PY{n}{sales\PYZus{}minimo}\PY{p}{(}\PY{l+m+mf}{216.4}\PY{p}{)}\PY{p}{)} \PY{o}{+} \PY{l+s+s2}{\PYZdq{}}\PY{l+s+s2}{ \PYZhy{} }\PY{l+s+s2}{\PYZdq{}} \PY{o}{+} \PY{n+nb}{str}\PY{p}{(}\PY{n}{sales\PYZus{}maximo}\PY{p}{(}\PY{l+m+mf}{216.4}\PY{p}{)}\PY{p}{)} \PY{o}{+} \PY{l+s+s2}{\PYZdq{}}\PY{l+s+s2}{]}\PY{l+s+s2}{\PYZdq{}}  \PY{o}{+} \PY{l+s+s2}{\PYZdq{}}\PY{l+s+s2}{ vs }\PY{l+s+s2}{\PYZdq{}} \PY{o}{+} \PY{l+s+s2}{\PYZdq{}}\PY{l+s+s2}{ y real = 22.6}\PY{l+s+s2}{\PYZdq{}}\PY{p}{)}
\PY{n+nb}{print}\PY{p}{(}\PY{l+s+s2}{\PYZdq{}}\PY{l+s+s2}{Valor 3 (y) predecido: }\PY{l+s+s2}{\PYZdq{}} \PY{o}{+} \PY{n+nb}{str}\PY{p}{(}\PY{n}{predict\PYZus{}sales}\PY{p}{(}\PY{l+m+mf}{57.5}\PY{p}{)}\PY{p}{)} \PY{o}{+} \PY{l+s+s2}{\PYZdq{}}\PY{l+s+s2}{ Intervalo: [}\PY{l+s+s2}{\PYZdq{}} \PY{o}{+} \PY{n+nb}{str}\PY{p}{(}\PY{n}{sales\PYZus{}minimo}\PY{p}{(}\PY{l+m+mf}{57.5}\PY{p}{)}\PY{p}{)} \PY{o}{+} \PY{l+s+s2}{\PYZdq{}}\PY{l+s+s2}{ \PYZhy{} }\PY{l+s+s2}{\PYZdq{}} \PY{o}{+} \PY{n+nb}{str}\PY{p}{(}\PY{n}{sales\PYZus{}maximo}\PY{p}{(}\PY{l+m+mf}{57.5}\PY{p}{)}\PY{p}{)} \PY{o}{+} \PY{l+s+s2}{\PYZdq{}}\PY{l+s+s2}{]}\PY{l+s+s2}{\PYZdq{}}  \PY{o}{+} \PY{l+s+s2}{\PYZdq{}}\PY{l+s+s2}{ vs }\PY{l+s+s2}{\PYZdq{}} \PY{o}{+} \PY{l+s+s2}{\PYZdq{}}\PY{l+s+s2}{ y real = 11.8}\PY{l+s+s2}{\PYZdq{}}\PY{p}{)}
\PY{n+nb}{print}\PY{p}{(}\PY{l+s+s2}{\PYZdq{}}\PY{l+s+s2}{Valor 4 (y) predecido: }\PY{l+s+s2}{\PYZdq{}} \PY{o}{+} \PY{n+nb}{str}\PY{p}{(}\PY{n}{predict\PYZus{}sales}\PY{p}{(}\PY{l+m+mf}{11.7}\PY{p}{)}\PY{p}{)} \PY{o}{+} \PY{l+s+s2}{\PYZdq{}}\PY{l+s+s2}{ Intervalo: [}\PY{l+s+s2}{\PYZdq{}} \PY{o}{+} \PY{n+nb}{str}\PY{p}{(}\PY{n}{sales\PYZus{}minimo}\PY{p}{(}\PY{l+m+mf}{11.7}\PY{p}{)}\PY{p}{)} \PY{o}{+} \PY{l+s+s2}{\PYZdq{}}\PY{l+s+s2}{ \PYZhy{} }\PY{l+s+s2}{\PYZdq{}} \PY{o}{+} \PY{n+nb}{str}\PY{p}{(}\PY{n}{sales\PYZus{}maximo}\PY{p}{(}\PY{l+m+mf}{11.7}\PY{p}{)}\PY{p}{)} \PY{o}{+} \PY{l+s+s2}{\PYZdq{}}\PY{l+s+s2}{]}\PY{l+s+s2}{\PYZdq{}}  \PY{o}{+} \PY{l+s+s2}{\PYZdq{}}\PY{l+s+s2}{ vs }\PY{l+s+s2}{\PYZdq{}} \PY{o}{+} \PY{l+s+s2}{\PYZdq{}}\PY{l+s+s2}{ y real = 7.3}\PY{l+s+s2}{\PYZdq{}}\PY{p}{)}
\PY{n+nb}{print}\PY{p}{(}\PY{l+s+s2}{\PYZdq{}}\PY{l+s+s2}{Valor 5 (y) predecido: }\PY{l+s+s2}{\PYZdq{}} \PY{o}{+} \PY{n+nb}{str}\PY{p}{(}\PY{n}{predict\PYZus{}sales}\PY{p}{(}\PY{l+m+mf}{250.9}\PY{p}{)}\PY{p}{)} \PY{o}{+} \PY{l+s+s2}{\PYZdq{}}\PY{l+s+s2}{ Intervalo: [}\PY{l+s+s2}{\PYZdq{}} \PY{o}{+} \PY{n+nb}{str}\PY{p}{(}\PY{n}{sales\PYZus{}minimo}\PY{p}{(}\PY{l+m+mf}{250.9}\PY{p}{)}\PY{p}{)} \PY{o}{+} \PY{l+s+s2}{\PYZdq{}}\PY{l+s+s2}{ \PYZhy{} }\PY{l+s+s2}{\PYZdq{}} \PY{o}{+} \PY{n+nb}{str}\PY{p}{(}\PY{n}{sales\PYZus{}maximo}\PY{p}{(}\PY{l+m+mf}{250.9}\PY{p}{)}\PY{p}{)} \PY{o}{+} \PY{l+s+s2}{\PYZdq{}}\PY{l+s+s2}{]}\PY{l+s+s2}{\PYZdq{}} \PY{o}{+} \PY{l+s+s2}{\PYZdq{}}\PY{l+s+s2}{ vs }\PY{l+s+s2}{\PYZdq{}} \PY{o}{+} \PY{l+s+s2}{\PYZdq{}}\PY{l+s+s2}{ y real = 22.2}\PY{l+s+s2}{\PYZdq{}}\PY{p}{)}
\end{Verbatim}
\end{tcolorbox}

    \begin{Verbatim}[commandchars=\\\{\}]
Valor 1 (y) predecido: 10.7471 Intervalo: [9.4144 - 12.0796] vs  y real = 14.6
Valor 2 (y) predecido: 17.3116 Intervalo: [15.218800000000002 - 19.4042] vs  y
real = 22.6
Valor 3 (y) predecido: 9.763850000000001 Intervalo: [8.545 - 10.9825] vs  y real
= 11.8
Valor 4 (y) predecido: 7.58835 Intervalo: [6.6213999999999995 - 8.5551] vs  y
real = 7.3
Valor 5 (y) predecido: 18.95035 Intervalo: [16.6678 - 21.2327] vs  y real = 22.2
    \end{Verbatim}

    \begin{tcolorbox}[breakable, size=fbox, boxrule=1pt, pad at break*=1mm,colback=cellbackground, colframe=cellborder]
\prompt{In}{incolor}{59}{\boxspacing}
\begin{Verbatim}[commandchars=\\\{\}]
\PY{c+c1}{\PYZsh{}Para responder eligiré los primeros 5 valores de X (\PYZdq{}TV\PYZdq{})}

\PY{k}{def} \PY{n+nf}{predict\PYZus{}sales}\PY{p}{(}\PY{n}{x}\PY{p}{)}\PY{p}{:}
    
    \PY{n}{sales\PYZus{}prediccion} \PY{o}{=} \PY{l+m+mf}{6.8843} \PY{o}{+} \PY{p}{(}\PY{l+m+mf}{0.0491} \PY{o}{*} \PY{n}{x}\PY{p}{)}
    \PY{k}{return} \PY{n}{sales\PYZus{}prediccion}

\PY{k}{def} \PY{n+nf}{sales\PYZus{}minimo}\PY{p}{(}\PY{n}{x}\PY{p}{)}\PY{p}{:}
    
    \PY{n}{minimo} \PY{o}{=} \PY{l+m+mf}{5.936} \PY{o}{+} \PY{p}{(}\PY{l+m+mf}{0.044} \PY{o}{*} \PY{n}{x}\PY{p}{)}
    \PY{k}{return} \PY{n}{minimo}

\PY{k}{def} \PY{n+nf}{sales\PYZus{}maximo}\PY{p}{(}\PY{n}{x}\PY{p}{)}\PY{p}{:}
    
    \PY{n}{maximo} \PY{o}{=} \PY{l+m+mf}{7.832} \PY{o}{+} \PY{p}{(}\PY{l+m+mf}{0.055} \PY{o}{*} \PY{n}{x}\PY{p}{)}
    \PY{k}{return} \PY{n}{maximo}

\PY{n+nb}{print}\PY{p}{(}\PY{l+s+s2}{\PYZdq{}}\PY{l+s+s2}{Valor 1 (y) predecido: }\PY{l+s+s2}{\PYZdq{}} \PY{o}{+} \PY{n+nb}{str}\PY{p}{(}\PY{n}{predict\PYZus{}sales}\PY{p}{(}\PY{l+m+mf}{78.2}\PY{p}{)}\PY{p}{)} \PY{o}{+} \PY{l+s+s2}{\PYZdq{}}\PY{l+s+s2}{ Intervalo: [}\PY{l+s+s2}{\PYZdq{}} \PY{o}{+} \PY{n+nb}{str}\PY{p}{(}\PY{n}{sales\PYZus{}minimo}\PY{p}{(}\PY{l+m+mf}{78.2}\PY{p}{)}\PY{p}{)} \PY{o}{+} \PY{l+s+s2}{\PYZdq{}}\PY{l+s+s2}{ \PYZhy{} }\PY{l+s+s2}{\PYZdq{}} \PY{o}{+} \PY{n+nb}{str}\PY{p}{(}\PY{n}{sales\PYZus{}maximo}\PY{p}{(}\PY{l+m+mf}{78.2}\PY{p}{)}\PY{p}{)} \PY{o}{+} \PY{l+s+s2}{\PYZdq{}}\PY{l+s+s2}{]}\PY{l+s+s2}{\PYZdq{}} \PY{o}{+} \PY{l+s+s2}{\PYZdq{}}\PY{l+s+s2}{ vs }\PY{l+s+s2}{\PYZdq{}} \PY{o}{+} \PY{l+s+s2}{\PYZdq{}}\PY{l+s+s2}{ y real = 14.6}\PY{l+s+s2}{\PYZdq{}}\PY{p}{)}
\PY{n+nb}{print}\PY{p}{(}\PY{l+s+s2}{\PYZdq{}}\PY{l+s+s2}{Valor 2 (y) predecido: }\PY{l+s+s2}{\PYZdq{}} \PY{o}{+} \PY{n+nb}{str}\PY{p}{(}\PY{n}{predict\PYZus{}sales}\PY{p}{(}\PY{l+m+mf}{216.4}\PY{p}{)}\PY{p}{)} \PY{o}{+} \PY{l+s+s2}{\PYZdq{}}\PY{l+s+s2}{ Intervalo: [}\PY{l+s+s2}{\PYZdq{}} \PY{o}{+} \PY{n+nb}{str}\PY{p}{(}\PY{n}{sales\PYZus{}minimo}\PY{p}{(}\PY{l+m+mf}{216.4}\PY{p}{)}\PY{p}{)} \PY{o}{+} \PY{l+s+s2}{\PYZdq{}}\PY{l+s+s2}{ \PYZhy{} }\PY{l+s+s2}{\PYZdq{}} \PY{o}{+} \PY{n+nb}{str}\PY{p}{(}\PY{n}{sales\PYZus{}maximo}\PY{p}{(}\PY{l+m+mf}{216.4}\PY{p}{)}\PY{p}{)} \PY{o}{+} \PY{l+s+s2}{\PYZdq{}}\PY{l+s+s2}{]}\PY{l+s+s2}{\PYZdq{}}  \PY{o}{+} \PY{l+s+s2}{\PYZdq{}}\PY{l+s+s2}{ vs }\PY{l+s+s2}{\PYZdq{}} \PY{o}{+} \PY{l+s+s2}{\PYZdq{}}\PY{l+s+s2}{ y real = 22.6}\PY{l+s+s2}{\PYZdq{}}\PY{p}{)}
\PY{n+nb}{print}\PY{p}{(}\PY{l+s+s2}{\PYZdq{}}\PY{l+s+s2}{Valor 3 (y) predecido: }\PY{l+s+s2}{\PYZdq{}} \PY{o}{+} \PY{n+nb}{str}\PY{p}{(}\PY{n}{predict\PYZus{}sales}\PY{p}{(}\PY{l+m+mf}{57.5}\PY{p}{)}\PY{p}{)} \PY{o}{+} \PY{l+s+s2}{\PYZdq{}}\PY{l+s+s2}{ Intervalo: [}\PY{l+s+s2}{\PYZdq{}} \PY{o}{+} \PY{n+nb}{str}\PY{p}{(}\PY{n}{sales\PYZus{}minimo}\PY{p}{(}\PY{l+m+mf}{57.5}\PY{p}{)}\PY{p}{)} \PY{o}{+} \PY{l+s+s2}{\PYZdq{}}\PY{l+s+s2}{ \PYZhy{} }\PY{l+s+s2}{\PYZdq{}} \PY{o}{+} \PY{n+nb}{str}\PY{p}{(}\PY{n}{sales\PYZus{}maximo}\PY{p}{(}\PY{l+m+mf}{57.5}\PY{p}{)}\PY{p}{)} \PY{o}{+} \PY{l+s+s2}{\PYZdq{}}\PY{l+s+s2}{]}\PY{l+s+s2}{\PYZdq{}}  \PY{o}{+} \PY{l+s+s2}{\PYZdq{}}\PY{l+s+s2}{ vs }\PY{l+s+s2}{\PYZdq{}} \PY{o}{+} \PY{l+s+s2}{\PYZdq{}}\PY{l+s+s2}{ y real = 11.8}\PY{l+s+s2}{\PYZdq{}}\PY{p}{)}
\PY{n+nb}{print}\PY{p}{(}\PY{l+s+s2}{\PYZdq{}}\PY{l+s+s2}{Valor 4 (y) predecido: }\PY{l+s+s2}{\PYZdq{}} \PY{o}{+} \PY{n+nb}{str}\PY{p}{(}\PY{n}{predict\PYZus{}sales}\PY{p}{(}\PY{l+m+mf}{11.7}\PY{p}{)}\PY{p}{)} \PY{o}{+} \PY{l+s+s2}{\PYZdq{}}\PY{l+s+s2}{ Intervalo: [}\PY{l+s+s2}{\PYZdq{}} \PY{o}{+} \PY{n+nb}{str}\PY{p}{(}\PY{n}{sales\PYZus{}minimo}\PY{p}{(}\PY{l+m+mf}{11.7}\PY{p}{)}\PY{p}{)} \PY{o}{+} \PY{l+s+s2}{\PYZdq{}}\PY{l+s+s2}{ \PYZhy{} }\PY{l+s+s2}{\PYZdq{}} \PY{o}{+} \PY{n+nb}{str}\PY{p}{(}\PY{n}{sales\PYZus{}maximo}\PY{p}{(}\PY{l+m+mf}{11.7}\PY{p}{)}\PY{p}{)} \PY{o}{+} \PY{l+s+s2}{\PYZdq{}}\PY{l+s+s2}{]}\PY{l+s+s2}{\PYZdq{}}  \PY{o}{+} \PY{l+s+s2}{\PYZdq{}}\PY{l+s+s2}{ vs }\PY{l+s+s2}{\PYZdq{}} \PY{o}{+} \PY{l+s+s2}{\PYZdq{}}\PY{l+s+s2}{ y real = 7.3}\PY{l+s+s2}{\PYZdq{}}\PY{p}{)}
\PY{n+nb}{print}\PY{p}{(}\PY{l+s+s2}{\PYZdq{}}\PY{l+s+s2}{Valor 5 (y) predecido: }\PY{l+s+s2}{\PYZdq{}} \PY{o}{+} \PY{n+nb}{str}\PY{p}{(}\PY{n}{predict\PYZus{}sales}\PY{p}{(}\PY{l+m+mf}{250.9}\PY{p}{)}\PY{p}{)} \PY{o}{+} \PY{l+s+s2}{\PYZdq{}}\PY{l+s+s2}{ Intervalo: [}\PY{l+s+s2}{\PYZdq{}} \PY{o}{+} \PY{n+nb}{str}\PY{p}{(}\PY{n}{sales\PYZus{}minimo}\PY{p}{(}\PY{l+m+mf}{250.9}\PY{p}{)}\PY{p}{)} \PY{o}{+} \PY{l+s+s2}{\PYZdq{}}\PY{l+s+s2}{ \PYZhy{} }\PY{l+s+s2}{\PYZdq{}} \PY{o}{+} \PY{n+nb}{str}\PY{p}{(}\PY{n}{sales\PYZus{}maximo}\PY{p}{(}\PY{l+m+mf}{250.9}\PY{p}{)}\PY{p}{)} \PY{o}{+} \PY{l+s+s2}{\PYZdq{}}\PY{l+s+s2}{]}\PY{l+s+s2}{\PYZdq{}} \PY{o}{+} \PY{l+s+s2}{\PYZdq{}}\PY{l+s+s2}{ vs }\PY{l+s+s2}{\PYZdq{}} \PY{o}{+} \PY{l+s+s2}{\PYZdq{}}\PY{l+s+s2}{ y real = 22.2}\PY{l+s+s2}{\PYZdq{}}\PY{p}{)}
\end{Verbatim}
\end{tcolorbox}

    \begin{Verbatim}[commandchars=\\\{\}]
Valor 1 (y) predecido: 10.72392 Intervalo: [9.3768 - 12.133] vs  y real = 14.6
Valor 2 (y) predecido: 17.50954 Intervalo: [15.4576 - 19.734] vs  y real = 22.6
Valor 3 (y) predecido: 9.70755 Intervalo: [8.466 - 10.9945] vs  y real = 11.8
Valor 4 (y) predecido: 7.4587699999999995 Intervalo: [6.4508 - 8.4755] vs  y
real = 7.3
Valor 5 (y) predecido: 19.20349 Intervalo: [16.9756 - 21.6315] vs  y real = 22.2
    \end{Verbatim}

    Conclusión: Se concluye que el modelo no es capaz de
predecir con precisión dentro del intervalo de confianza del 95\% y esto
puede deberse a que presenta una gran variabilidad justificada en el
alto coeficiente de variación y en que el modelo no cumplió el supuesto
de homocedasticidad.


    % Add a bibliography block to the postdoc
    
    
    
\end{document}
